\documentclass[12pt,a4paper]{report}
\usepackage[utf8]{inputenc}
% \usepackage{graphicx}
\usepackage{amsthm,amsfonts,amssymb}
\usepackage{amsmath}
\usepackage{a4wide}
\usepackage{titlesec}
\usepackage{makeidx}
\usepackage{tikz-cd}
% \usepackage{tikz}

% TODO proper referencing


% \usetikzlibrary{arrows}
\tikzset{commutative diagrams/.cd}
\tikzset{commutative diagrams/row    sep/normal=1.3cm}
\tikzset{commutative diagrams/column sep/normal=1.3cm}
\makeindex

% Formating of theorems, lemmas, ...
\titlelabel{\thetitle.\quad}

\def\afterDotSpace{.5em}

\newcounter{thmCounter}[section]
\renewcommand \thechapter{\Roman{chapter}}
\renewcommand \thesection{\arabic{section}}
\renewcommand{\thethmCounter}{\thesection.\arabic{thmCounter}}

\def\blockHelperA{\par\medskip\noindent}
\def\blockHelperB{\refstepcounter{thmCounter}\arabic{section}.\arabic{thmCounter}.}
\def\blockHelperCA{\hspace{\afterDotSpace}\ignorespaces}
\def\blockHelperCB{\it\hspace{\afterDotSpace}\ignorespaces}

\def\blockEnv#1{\blockHelperA\hbox{\bf {\blockHelperB} #1.}\blockHelperCA}
\def\blockEnvS#1{\blockHelperA\hbox{\bf #1.}\blockHelperCA}
\def\blockPropEnv#1{\blockHelperA\hbox{\bf {\blockHelperB} #1.}\blockHelperCB}
\def\blockPropEnvS#1{\blockHelperA\hbox{\bf #1.}\blockHelperCB}
\def\endBlockEnv{\par\medskip}

% TODO add little space before numbered paragraph
\def\num{\blockHelperA\hbox{\bf {\blockHelperB}}\hspace{\afterDotSpace}}

\newenvironment{block}{\blockEnv}{\endBlockEnv}
\newenvironment{block*}{\blockEnvS}{\endBlockEnv}
\newenvironment{blockProp}{\blockPropEnv}{\endBlockEnv}
\newenvironment{blockProp*}{\blockPropEnvS}{\endBlockEnv}

\newtheoremstyle{newthmstyle}% name of the style to be used
    {3pt}% measure of space to leave above the theorem. E.g.: 3pt
    {3pt}% measure of space to leave below the theorem. E.g.: 3pt
    {\itshape}% name of font to use in the body of the theorem
    {}% measure of space to indent
    {\bfseries}% name of head font
    {.}% punctuation between head and body
    {\afterDotSpace}% space after theorem head; " " = normal interword space
    {\thmnumber{#2.}\thmname{ #1}\thmnote{ \rm [#3]}}% Manually specify head

\newtheoremstyle{newthmstyleNormal}{3pt}{3pt}{}{}{\bfseries}{.}{\afterDotSpace}
    {\thmnumber{#2.}\thmname{ #1}\thmnote{ \rm [#3]}}

\theoremstyle{newthmstyle}
\newtheorem{name}[thmCounter]{Note}
\newtheorem{lemma}[thmCounter]{Lemma}
\newtheorem{theorem}[thmCounter]{Theorem}
\newtheorem{proposition}[thmCounter]{Proposition}
\newtheorem{observation}[thmCounter]{Observation}

\theoremstyle{newthmstyleNormal}
\newtheorem{definition}[thmCounter]{Definition}

% Categories
\newcommand\categoryStyle[1]{\ensuremath{\mathbf{#1}}}
\newcommand\Frm{\categoryStyle{Frm}}
\newcommand\StoneFrm{\categoryStyle{StoneFrm}}
\newcommand\ExtrStoneFrm{\categoryStyle{ExtrStoneFrm}}
\newcommand\Top{\categoryStyle{Top}}
\newcommand\StoneSp{\categoryStyle{StoneSp}}
\newcommand\Bool{\categoryStyle{Bool}}
\newcommand\ComplBool{\categoryStyle{ComplBool}}

% Special symbols
\newcommand\N{\ensuremath{\mathbb{N}}}
\newcommand\R{\ensuremath{\mathfrak{R}}}
\newcommand\J{\ensuremath{\mathfrak{J}}}
\newcommand\p[1]{\ensuremath{\mathcal{ #1 }}}
\newcommand\Bo{\ensuremath{\mathfrak{B}}} % booleanization
\newcommand\Bc{\ensuremath{\mathbb{B}}} % boolean/complemented elements
\newcommand\closure[1]{\overline{#1}}
\newcommand\downset{\mathord{\downarrow}\mkern1mu} % mathord/mathbin/mathrel treats symbol as oordinal math symbol/binary relation/...
\newcommand\rbelow{\prec}
\DeclareMathOperator{\id}{id}

% Function restriction
% http://tex.stackexchange.com/questions/22252/how-to-typeset-function-restrictions
\newcommand\restr[2]{{% we make the whole thing an ordinary symbol
  \left.\kern-\nulldelimiterspace % automatically resize the bar with \right
  #1 % the function
  \vphantom{\big|} % pretend it's a little taller at normal size
  \right|_{#2} % this is the delimiter
}}

% TODO spacing before and after thms and others

% Other stuff
\newcommand\DEF[2][]{\index{#2#1}\emph{#2}}
% example \DEF[|see{null-dimensional}]{zero-dimensional}
\newcommand\DEFSYM[2]{\index{#1@#2}\emph{#2}}

\newenvironment{diagram}{\begin{center}\begin{tikzcd}}{\end{tikzcd}\end{center}}

\title{Some point--free aspects of connectedness}
\author{Tom\'a\v s Jakl}

\begin{document}
\maketitle
\tableofcontents


\chapter{Introduction}
\cite{picado2011frames}

\chapter{Preliminaries}
\section{Partially ordered set}
\begin{itemize}
    \item Joins, meets, order, lattice, distributive lattice
    \item bottom (0) resp. top (1) and $0_S$ resp. $1_S$
    \item Galois correspondence (equivalent definitions)
    \item filters, ideals
    \item pseudocomplements and complements (complemented elements), $(a \wedge b)^{**} = a^{**} \wedge b^{**}$
    \item Boolean algebras and Boolean homomorphisms, \Bool
    \item Heyting algebras
\end{itemize}
\section{Category Theory}
category, functor, natural transformation, natural equivalence, adjunction, units of adjunction
TODO exmplain importance of adjunction by mentioning the limit preserving of adjuction

\section{Topology and point--free Topology}
Basic of \Top{} and \Frm.
frame homomorphisms, localic maps
Subspace and sublocale/subframe.
Spaciality.
Separation axioms (regularity, complete regularity, normality, ...?), rather below (and useful facts about them)
TODO place somewhere StoneSp and StoneFrm and \J.

\chapter{Connectedness and Compactification}
\section{Connectedness and variants of disconnectedness}
DeMorgan/extremally disconnected, null-dimensional.
\section{Compactness and compactification}
\section{Properties of compactification with respect to disconnectedness}
\begin{lemma}
    The following are equivalent:

    \begin{enumerate}
        \item Closure of each open sublocale is open.
        \item $\closure{\mathfrak{o}(a)} = \mathfrak{o}(a^{**})$.
        \item $a^{**} \vee a^* = 1$.
    \end{enumerate}
\end{lemma}

\begin{proposition}\label{p:extrDiscPreserv}
    Let $L$ be a extremally disconnected frame, then $\R L$ is also extremally disconnected.
\end{proposition}

\chapter{Stone duality}
\section{Stone correspondence for \StoneFrm}
\begin{definition}
    We say a frame is \DEF{Stone frame} if it is compact, null--dimensional and regular.
\end{definition}
% TODO define category \StoneFrm

\begin{definition}
    Let $B$ be a Boolean algebra. Define $\DEFSYM{J}{\J} B$ to be the set of all ideals of $B$.
    For $f\colon A \to B$ a Boolean homomorphism, define $\J f\colon \J A \to \J B$ as
    $$(\J f)(I) = \downset f[I].$$
\end{definition}

\begin{lemma}\label{p:complIdeal}
    Let $B$ be a Boolean algebra and $I \in \J B$. Then $I$ is complemented iff $I = \downset b$ for some $b \in B$.
\end{lemma}
\begin{proof}
    Let $I$ be a complemented ideal. Since $I \vee I^c = 1_{\J B}$ there exists $a \in I$ and $b \in I^c$ such that $a \vee b = 1$. From $I \wedge I^c = \{0\}$ we have $a \wedge b = 0$ and $I \wedge \downset b = \{0\}$.
    From uniques of complements we get $I^c = \downset b$, indeed $I \vee \downset b = 1_{\J B}$ and $I \wedge \downset b = \{0\}$. Using the same argument we get $I = \downset a$.

    Converse implication is trivial since $\downset a \vee \downset a^c = 1_{\J B}$ and $\downset a \wedge \downset a^c = \{0\}$.
\end{proof}

\begin{proposition}\label{p:JisFunctor}
    $\J\colon \Bool \to \StoneFrm$ is a functor.
\end{proposition}
\begin{proof}
    $\J B$ is trivially a compact frame. Lemma \ref{p:complIdeal} implies that $\downset a$ is complemented and $\downset a \rbelow \downset a$ for all $a \in B$. Thus for any ideal $I \in \J B$ we obtain
    $$ I = \bigvee \{ \downset a : a \in I \} \subseteq \bigvee \{ J : J \rbelow I \} \subseteq I,$$

\noindent hence $\J B$ is regular and null--dimensional. The rest is easy.
\end{proof}

\begin{definition}
    Let $L$ be a Stone frame. Define $\DEFSYM{Bc}{\Bc} L$ to be the set of all complemented elements of $L$ and for $f\colon L \to M$ a frame homomorphism define
    $$\Bc f = \restr{f}{\Bc L}\colon \Bc L \to \Bc M.$$
\end{definition}
% TODO somehow mention that $f$ is between stone spaces

From the fact that a homomorphic image of a complemented element is a complemented element and join or meet of two complemented elements is again complemented one can see that $\Bc f$ is well--defined.
% TODO prove the `joins/meets are complemented' proposition

% TODO some comment's before this statement
\begin{observation}
    $\Bc\colon \StoneFrm \to \Bool$ is a functor.
\end{observation}

\num For $B \in \Bool$ define \index{istar@$i_*$}$i_B\colon B \to \Bc\J(B)$ as follows
    $$i_B\colon b \mapsto \downset b.$$
    The definition is correct by Lemma \ref{p:complIdeal} and $i_B$ is a Boolean homomorphism, indeed $\downset a \vee \downset b = \downset (a \vee b)$, $\downset a \wedge \downset b = \downset (a \wedge b)$ and $\downset 1 = B$ respectively $\downset 0 = \{0\}$ is top respectively bottom of $\Bc\J(B)$.

From Lemma \ref{p:complIdeal} we also see that $i_B$ is an isomorphism and therefore the following diagram commutes

\begin{diagram}
    A \ar{r}{i_A} \ar{d}[swap]{f} & \Bc\J(A) \ar{d}{\Bc\J(f)}\\
    B \ar{r}{i_B}                 & \Bc\J(B)
\end{diagram}

\noindent for any $f\colon A \to B$ Boolean homomorphism. From previous we have

\begin{blockProp*}{Proposition}
    The collection $i_*$ of Boolean homomorphisms forms a natural equivalence between $\Bc\J$ and the identity functor on \Bool.
\end{blockProp*}

% TODO better wording
\num Similarly for $L \in \StoneFrm$ we have a mapping \index{vstar@$v_*$}$v_L\colon \J\Bc(L) \to L$ defined as
    $$v_L\colon I \mapsto \bigvee I.$$
    And a mapping in the opposite direction $\iota\colon L \to \J\Bc(L)$:
    $$\iota\colon e \mapsto \downset e \cap \Bc L.$$

We can see that both $v_L$ and $\iota$ are monotone maps, $v_L \iota = \id_L$ (by null--dimensionality and regularity of $L$) and $\id_{\J\Bc(L)} \subseteq \iota v_L$, and therefore $v_L$ is the left Galois adjoint to $\iota$, hence $v_L$ preserves all suprema.
Since $\bigvee I_1 \wedge \bigvee I_2 = \bigvee \{a_1 \wedge a_2 : a_i \in I_i\} \leq \bigvee \{ a : a \in I_1 \cap I_2 \} = \bigvee (I_1 \cap I_2) \leq \bigvee I_1 \cap \bigvee I_2$, $v_L$ also preserves finite infima which makes it a frame homomorphism.

Finally $\id_{\J\Bc(L)} = \iota v_L$: Take any $x \in \iota v_L(I)$. From the definitions we immediately see that $x \leq \bigvee I$ and $x$ is complemented in $L$. By the fact that
    $$1 = x \vee x^c \leq \bigvee I \vee x^c$$
    and by compactness of $L$ there is a finite $F \subseteq I$ such that $\bigvee F \vee x^c = 1$. Since $x = 1 \wedge x = (\bigvee F \vee x^c) \wedge x = (\bigvee F \wedge x) \vee (x^c \wedge x) = \bigvee F \wedge x$ we get that $x \leq \bigvee F$ and therefore $x \in I$.

Previous observations give us the fact that $v_L$ is an isomorphism of $L$ and $\J\Bc(L)$ and also that for any $f\colon L \to M$ homomorphism of Stone frames the following diagram commutes

\begin{diagram}
    \J\Bc(L) \ar{d}[swap]{\J\Bc(f)} \ar{r}{v_L} & L \ar{d}{f} \\
    \J\Bc(M) \ar{r}{v_M}    & M
\end{diagram}

\noindent Again as conclusion of previous paragraphs we obtain

\begin{blockProp*}{Proposition}
    The collection $v_*$ of frame homomorphisms is a natural equivalence between $\J\Bc$ and identity functor on \StoneFrm.
\end{blockProp*}

\num Using previous facts we get the main result of this section.

\begin{blockProp*}{Theorem}
    Functor \Bc{} is the right adjoint to the functor \J{} with $i_*$ as unit of adjunction and $v_*$ as counit. Moreover \Bc{} and \J{} constitute an isomorphism of categories \StoneFrm{} and \Bool.
\end{blockProp*}

As we will see in next chapter, \J{} corresponds precisely to compactification of Boolean algebras the same way as the space of ultrafilters is part of Stone duality and compactification for spaces. (TODO better wording)

TODO (note about AC:) As one can check, the whole correspondence is given in constructive setting. No use of Axiom of Choice unlike in classical case, \dots

\section{Stone duality for \StoneSp} % TODO maybe StoneSp


\section{Parts of duality}
\subsection{Complete Boolean algebras}
TODO say something about geometrical/topological meaning of complete Boolean algebras (meaning that ComplBool are frames...).

\begin{proposition}
    Let $B$ be a complete Boolean algebra. The frame $\J B$ is an extremally disconnected Stone frame.
\end{proposition}
\begin{proof}
    In any Boolean algebra the relations $\rbelow$ and $\leq$ coincide. Moreover each complete Boolean algebra is a (completely regular extremally disconnected) Boolean frame. Thus $\J{B}$ equals $\R{B}$.

    From Lemma \ref{p:extrDiscPreserv} we know compactification preserves extremal disconnectedness and therefore \J{B} is also extremally disconnected. From Lemma \ref{p:JisFunctor} we know it is also a Stone frame.
\end{proof}

\num TODO Recall the definition of Booleanization (\Bo). % TODO add \DEF

\begin{definition}
    Let $H$ be a Heyting algebra, by $\Bo H$ denote the set $\{ a^{**} : a \in H \}$.

    And for a Heyting homomorphism $f\colon H \to K$, set $\Bo f\colon \Bo H \to \Bo K$ to be the mapping
    $$\Bo f\colon a \mapsto f(a)^{**}.$$
\end{definition}

\begin{block*}{Facts}
    For any Heyting algebra $H$, its booleanization, the $\Bo H$, is a complete Boolean algebra with joins given by the following formula
    $$ a \sqcup b = (a^* \wedge b^*)^*.$$

    Moreover for a locale $L$, $\Bo L$ is the smallest dense sublocale of $L$ and $ a \sqcup b = (a \vee b)^{**}$.
\end{block*}

\begin{lemma}
    \begin{enumerate}
        \item For all $J \in \J L$: $J^* =\;\downset (\bigvee J)^*$.
        \item For all $a \in L$: $(\downset a)^* = \downset a^*$ in $\J L$.
        \item If $L$ is Boolean, then for all $J \in \J L$: $J^{**} =\;\downset \bigvee J$.
    \end{enumerate}
\end{lemma}
\begin{proof}
    We will prove just the first case, the others follow directly from it. Let $J$ be any ideal on $L$. Observe that $a \wedge \bigvee J = 0 $ iff $\downset a \wedge J = 0_{\Bo L}$. Since
        \begin{align*}
            J^* & = \bigvee \{ L : L \wedge J = 0_{\Bo L} \} = \bigcup \{ \downset a : \downset a \wedge J = 0_{\Bo L} \} \text{ and}\\
            (\bigvee J)^* & = \bigvee \{ a : a \wedge \bigvee J = 0 \},
        \end{align*}

    \noindent we see that $\downset a \subseteq J^*$ iff $a \in J^*$ iff $a \leq (\bigvee J)^*$, hence $J^* = \downset (\bigvee J)^*$.
\end{proof}

\begin{proposition}
    Let $B$ be a Boolean frame, then $B \cong \Bo\J(B)$.
\end{proposition}
\begin{proof}
    From the previous Lemma we see that $\J \in \Bo\J(B)$ iff $J = J^{**} = \downset \bigvee J$. On the other hand, for $a \in B$: $(\downset a)^{**} = \downset a^{**} = \downset a$.

    (FIXME:) Denote $\tilde i\colon B \to \Bo\J(B)$ the mapping $a \mapsto \downset a$. It is a Boolean homomorphism, because $\downset a \vee \downset b = \downset (a \vee b)$, $\downset a \wedge \downset b = \downset (a \wedge b)$, $\downset 0 = \{0\} = 0_{\Bo\J(B)}$ and $\downset 1 = B = 1_{\Bo\J(B)}$. Consequently $\tilde i$ is an isomorphism of $B$ and $\Bo\J(B)$.
    % TODO $\tilde i$ need to preserve big joins, we get that by \bigsqcup I_i = (\bigcup I_i)^{**}
\end{proof}

\begin{proposition}
    Let $L$ be an extremally disconnected Stone frame, then $L \cong \J\Bo(L)$.
\end{proposition}
\begin{proof}
    Similarly to the general case, define $\tilde v_L\colon \J\Bo(L) \to L$ and $\tilde\iota\colon L \to \J\Bo(L)$ as
    $$  \tilde v_L\colon I \mapsto \bigvee I \quad\text{and}\quad \tilde\iota\colon a \mapsto \downset a \cap \Bo L.$$

    Trivially we have $\tilde\iota \tilde v_L \supseteq \id_{\J\Bo(L)}$ and $\tilde v_L \tilde\iota \leq \id_L$. We again have the situation where $\tilde v_L$ is the left Galois adjoint to $\tilde\iota$ and $\bigvee I_1 \wedge \bigvee I_2 = \bigvee (I_1 \wedge I_2)$ for any two $I_1, I_2 \in \J\Bo(L)$ and so $\tilde v_L$ is a frame homomorphism.

    Actually we also have the opposite inequalities: $\tilde v_L \tilde\iota \geq \id_L$ because $a \rbelow x$ implies $a^{**} \rbelow x$ and $L$ is regular. For $\tilde\iota \tilde v_L \subseteq \id_{\J\Bo(L)}$, take any $x \in \tilde\iota \tilde v_L(I)$. We have $x \leq \bigvee I$, $x = x^{**}$ and from extremal disconnectedness $x^{**} \vee x^* = 1$, so
    $$ 1 = x \vee x^* \leq \bigvee I \vee x^*$$
    \noindent holds and from compactness of $L$ there is a finite $F \subseteq I$ such that $\bigvee F \vee x^* = 1$. Now $x \leq \bigvee F$ since $x = 1 \wedge x = (x^* \vee \bigvee F) \wedge x = \bigvee F \wedge x$ and therefore $x \in I$.
\end{proof}

\begin{definition}
    Define $\beta_L\colon L \to \Bo L$ to be the mapping $a \mapsto a^{**}$.
\end{definition}

\begin{lemma}
    $\beta_L$ is a frame homomorphism.
\end{lemma}
\begin{proof}
    $(a \wedge b)^{**} = a^{**} \wedge b^{**}$ is standard equality as stated in Lemma~\ref{??}. For joins see Lemma~\ref{??}, we have $(\bigvee a_i)^{**} = \bigsqcup a_i = \bigsqcup a_i^{**}$.
\end{proof}

\begin{lemma}
    For $L$ be extremally disconnected frame, $a \in L$, and $f\colon L \to M$ a frame homomorphism
    $$ f(a^{**})^* = f(a^*).$$

    Moreover for $a = a^{**}$ we have
    $$ f(a)^* = f(a^*).$$
\end{lemma}
\begin{proof}
    Direct implication of the distributivity of $L$, the fact that $x \leq a^* $ iff $x \wedge a = 0$ and that
    \begin{align*}
        f(a^{**} \vee a^*) & = f(a^{**}) \vee f(a^*) = 1 \\
        f(a^{**} \wedge a^*) & = f(a^{**}) \wedge f(a^*) = 0. &\qedhere
    \end{align*}
\end{proof}
% TODO strange qed symbol, see http://tex.stackexchange.com/questions/27283/how-to-correctly-format-and-align-a-latex-proof

\begin{theorem}
    $\Bo\colon \ExtrStoneFrm \to \ComplBool$ is a functor.
\end{theorem}
\begin{proof}
    TODO $\Bo L$ is a complete Boolean algebra.

    As a direct implication of the previous Lemma we get that for any frame homomorphism $f\colon L \to M$ between two extremally disconnected Stone frames the following diagram commutes.

    \begin{diagram}
        L \ar{r}{\beta_L} \ar{d}{f} & \Bo L \ar{d}{\Bo f}\\
        M \ar{r}{\beta_M}           & \Bo M
    \end{diagram}

    \noindent And therefore for any $f, g$ frame homomorphisms the following diagram also commutes.

    \begin{diagram}
        L \ar{r}{\beta_L}
          \ar{d}{f}
          \ar[bend right]{dd}[swap]{gf} &
        \Bo L \ar{d}[swap]{\Bo f}
              \ar[bend left]{dd}{\Bo (gf)}\\

        M \ar{r}{\beta_M} \ar{d}{g} & \Bo M \ar{d}[swap]{\Bo g}\\
        N \ar{r}{\beta_N}           & \Bo N
    \end{diagram}

    Consequently $\Bo$ is a functor.
\end{proof}

\num TODO By the following isomorphism lemmas we have the natural equivalence between identity functor and \Bo\J{} resp. \J\Bo{} and get an isomorphism of categories \categoryStyle{Heyt} and \categoryStyle{ExtrStoneFrm}, two subcategories of \Frm.

\num TODO Since each Boolean frame is completely regular, \J{} corresponds precisely to compactification of this frame.

\subsection{$\kappa$--complete Boolean algebras}
\begin{definition}
    Let $L$ be a frame, we say $L$ is \DEF{$\kappa$--basically disconnected} if for any subset $M \subseteq L$ of complemented elements such that $|M| \leq \kappa$ and for $m = \bigvee M$, $m^{**} \vee m^* = 1$ holds.
\end{definition}

\begin{definition}
    A lattice is \DEF{$\kappa$--complete}, for some cardinal $\kappa$, if any its subset of cardinality at most $\kappa$ has supremum.
\end{definition}

\begin{lemma}
    If $L$ is a $\kappa$--basically disconnected Stone frame, for some cardinal $\kappa$, then $\Bc L$ is a $\kappa$--complete Boolean algebra.
\end{lemma}
\begin{proof}
    For $M \subseteq \Bc L$ and $|M| \leq \kappa$, set $m = \bigvee M$. From the assumption that $L$ is $\kappa$--basically disconnected we have $m^{**} \vee m^* = 1$ and therefore $m^{**} \in \Bc L$. So $m^{**}$ is an upper bound for $M$ in $\Bc L$.

    Now let $n$ be an arbitrary upper bound for $M$ in $\Bc L$, thus $n$ is an upper bound in $L$ also, but $m \leq n$ since $m$ is the supremum of $M$ in $L$. Which gives us the desired relation $m^{**} \leq n^{**} = n$ and so $m^{**}$ is the supremum of $M$ in $\Bc L$.
\end{proof}

\begin{lemma}
    If $B$ is a $\kappa$--complete Boolean algebra, for some cardinal $\kappa$, then $\J B$ is $\kappa$--basically disconnected Stone frame.
\end{lemma}
\begin{proof}
    By the lemma \ref{p:complIdeal} we know that complemented ideals are precisely principal ideals. To prove the lemma we will show that for any $M \subseteq B$, $|M| \leq \kappa$ and $I = \bigvee \{ \downset a : a \in M \}$, $I^{**} \vee I^* = 1_{\J B}$.

    Define $m = \bigvee M$ and $J = \downset m$. Observe that $I^{**} = J$: Trivially $I^{**} \subseteq J^{**} = J$ and $J \subseteq I^{**}$ follows from
    $$ I^* = \bigcup \{ \downset a : \downset a \cap I = \{0\} \} = \bigcup \{ \downset a : a \wedge m = 0 \} = (\downset m)^* = J^*.$$

    Consequently, $I^{**} \vee I^* = J \vee J^* = \downset m \vee \downset m^c = 1_{\J B}$.
\end{proof}

\begin{theorem}
    The category of $\kappa$--basically disconnected Stone frames and frames homomorphisms is isomorphic to the category of $\kappa$--complete Boolean algebras and Boolean homomorphisms.
\end{theorem}
% TODO better wording

Equivalently in topological spaces, space is $\kappa$--basically disconnectedness iff any union of less than $\kappa$ clopen sets has open closure. In classical setting there is duality between $\kappa$--complete Boolean algebras and $\kappa$--basically disconnected Stone spaces~\cite{monk1989handbook}.

\subsection*{Comments}
TODO picture of correspondence
TODO Mention $\sigma$-Frames.
TODO Compare compactification in classical and in point-free setting. Mention that \J{} correspond to compactification, the topology of Ult is isomorphic to ideal lattice of Boolean algebra and that this was known~\cite{monk1989handbook}.

\chapter{Compactification and Metrizability}
\section{Uniformity and metrizability}
\section{Non--metrizability of compactification}
\begin{lemma}
    For $L$ normal frame: $\sigma(x)\vee\sigma(x) = \sigma(x\vee y)$.
\end{lemma}

\chapter{Conclusion}
TODO or Further work?

\bibliographystyle{plain}
\bibliography{refs.bib}

\clearpage
% \addcontentsline{toc}{chapter}{Index}
\printindex
\end{document}
