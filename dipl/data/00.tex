%%%
%%% MASTER'S THESIS TEMPLATE - ENGLISH
%%%  
%%%  * the title page and front matter
%%%
%%%  AUTHORS:  Martin Mares (mares@kam.mff.cuni.cz)
%%%            Arnost Komarek (komarek@karlin.mff.cuni.cz), 2011
%%%            Michal Kulich (kulich@karlin.mff.cuni.cz), 2013
%%%
%%%  LAST UPDATED: 20130318
%%%  
%%%  ===========================================================================

\pagestyle{empty}
\begin{center}

{\large Charles University in Prague}

\medskip
{\large Faculty of Mathematics and Physics}

\vfill
{\bfseries\Large MASTER THESIS}

\vfill
\centerline{\mbox{\includegraphics[width=60mm]{mfflogo.pdf}}}

\vfill
\vspace{5mm}

{\LARGE Tom\'a\v s Jakl}

\vspace{15mm}

{\LARGE\bfseries Some point--free aspects of connectedness}

\vfill

Department of Applied Mathematics

\vfill

\begin{tabular}{rl}
Supervisor of the master thesis: & Prof. RNDr. Aleš Pultr, DrSc. \\   
\noalign{\vspace{2mm}}
Study programme: & Computer Science\\
\noalign{\vspace{2mm}}
Specialization: & Discrete Models and Algorithms\\
\end{tabular}

\vfill

Prague 2013

\end{center}



\newpage
\vspace*{\stretch{8}}

\noindent
I declare that I carried out this master thesis independently, and only with the cited
sources, literature and other professional sources.

\medskip\noindent
I understand that my work relates to the rights and obligations under the Act No.
121/2000 Coll., the Copyright Act, as amended, in particular the fact that the Charles
University in Prague has the right to conclude a license agreement on the use of this
work as a school work pursuant to Section 60 paragraph 1 of the Copyright Act.

% \vspace{18mm}
% \noindent
% %% Place and date of signature
% In \makebox[4cm]{\dotfill} on \makebox[2.5cm]{\dotfill}
% \hspace*{\fill}
% Author signature
% \hspace*{\fill}
%
% \vspace*{\stretch{1}}

\vspace{18mm}

\hbox{\hbox to 0.5\hsize{%
In Prague date July 26, 2013
\hss}\hbox to 0.5\hsize{%
Tom\'a\v s Jakl
\hss}}

\vspace*{\stretch{1}}

\newpage
%%% Czech and English abstracts

\vbox to 0.5\vsize{
\setlength\parindent{0mm}
\setlength\parskip{5mm}

\textbf{N\'azev pr\'ace:}
N\v ekter\'e bezbodov\' e aspekty souvislosti
\\
\textbf{Autor:}
Tom\'a\v s Jakl
\\
\textbf{Katedra:}
Katedra aplikované matematiky
\\
\textbf{Vedouc\'\i\ diplomov\'e pr\'ace:}
Prof. RNDr. Ale\v s Pultr, DrSc., Katedra aplikované matematiky

\textbf{Abstrakt:}
V této práci ukážeme Stoneovu větu o reprezentaci, která je také známa pod názvem Stoneova dualita, v bezbodovém kontextu.
Předvedený důkaz je bezvýběrový, a protože se nemusíme starat o jednotlivé body, je mnohem jednodušší než původní důkaz.
Ukážeme, že pro každý nekonečný kardinál $\kappa$ jsou protějšky $\kappa$--úplných Booleových algebrer $\kappa$--bazicky nesouvislé Stoneovy framy.
Také předvedeme přesnou charakterizaci morfismů, které jsou v koresponenci s $\kappa$--úplnými Booleovskými homomorfismy.
Ikdyž Booleanizace není obecně funktoriální, v části duality extremálně nesouvislých Stoneových framů funktoriální je a dokonce tvoří ekvivalenci kategorií.
Na konci práce se zaměříme na De Morganovské (respektive extremálně nesouvislé) framy a ukážeme jejich novou charakterizaci pomocí jejich \empty{superhustých} sublokálů.
Naproti tomu jsou metrizovatelné framy, které nemají žádný netriviální superhustý sublokál, a proto nikdy není jejich netriviální \v Cech--Stoneova kompaktifikace metrizovatelná.

\textbf{Kl\'{\i}\v{c}ov\'a slova:}
Stoneova dualita, bezbodová topologie, kompaktifikace, De Morganovské framy, konstruktivní matematika

\vss}

\nobreak\vbox to 0.49\vsize{
\vspace{2em}
\setlength\parindent{0mm}
\setlength\parskip{5mm}

\textbf{Title:}
Some point--free aspects of connectedness
\\
\textbf{Author:}
Tom\'a\v s Jakl
\\
\textbf{Department:}
Department of Applied Mathematics
\\
\textbf{Supervisor of the master thesis:}
Prof. RNDr. Ale\v s Pultr, DrSc., Department of Applied Mathematics

\textbf{Abstract:}
In this thesis we present the Stone representation theorem, generally known as Stone duality in the point--free context.
The proof is choice--free and, since we do not have to be concerned with points, it is by far simpler than the original.
For each infinite cardinal $\kappa$ we show that the counterpart of the $\kappa$--complete Boolean algebras is constituted by the $\kappa$--basically disconnected Stone frames.
We also present a precise characterization of the morphisms which correspond to the $\kappa$--complete Boolean homomorphisms.
Although Booleanization is not functorial in general, in the part of the duality for extremally disconnected Stone frames it is, and constitutes an equivalence of categories.
We finish the thesis by focusing on the De Morgan (or extremally disconnected) frames and present a new characterization of these by their \emph{superdense} sublocales.
We also show that in contrast with this phenomenon, a metrizable frame has no non-trivial superdense sublocale; in other words, a non-trivial \v Cech--Stone compactification of a metrizable frame is never metrizable.

\textbf{Keywords:}
Stone duality, point--free topology, compactification, De Morgan frames, constructive mathematics
\vss}

\newpage
\openright

\pagestyle{plain}
\setcounter{page}{1}

\tableofcontents
