\chapter{Stone duality}
\section{Stone correspondence for \StoneFrm}

\begin{definition}
    We say a frame is \DEF{Stone frame} if it is compact, regular and null--dimensional.
\end{definition}
% TODO define category \StoneFrm
% TODO shifted text ("Definition")

\begin{definition}
    Let $B$ be a Boolean algebra. Define $\DEFSYM{J}{\J} B$ to be the set of all ideals of $B$.
    For a Boolean homomorphism $f\colon A \to B$, define $\J f\colon \J A \to \J B$ as
    $$(\J f)(I) = \downset f[I].$$
\end{definition}

\begin{lemma}\label{p:complIdeal}
    Let $B$ be a Boolean algebra and $I \in \J B$. Then $I$ is complemented iff $I = \downset b$ for some $b \in B$.
\end{lemma}
\begin{proof}
    Let $I$ be a complemented ideal. Since $I \vee I^c = 1_{\J B}$ there exists $a \in I$ and $b \in I^c$ such that $a \vee b = 1$. From $I \wedge I^c = 0_{\J B}$ we have $a \wedge b = 0$ and $I \wedge \downset b = 0_{\J B}$.
    From uniques of complements we get $I^c = \downset b$, indeed $I \vee \downset b = 1_{\J B}$ and $I \wedge \downset b = 0_{\J B}$. Using the same argument we get $I = \downset a$.

    Converse implication is trivial since $\downset a \vee \downset a^c = 1_{\J B}$ and $\downset a \wedge \downset a^c = 0_{\J B}$.
\end{proof}

\begin{proposition}\label{p:JisFunctor}
    $\J\colon \Bool \to \StoneFrm$ is a functor.
\end{proposition}
\begin{proof}
    $\J B$ is trivially a compact frame. Lemma \ref{p:complIdeal} implies that $\downset a$ is complemented and $\downset a \rbelow \downset a$ for all $a \in B$. Thus for any ideal $I \in \J B$ we obtain
    $$ I = \bigvee \Set{ \downset a | a \in I } \subseteq \bigvee \Set{ J | J \rbelow I } \subseteq I,$$

\noindent hence $\J B$ is regular and null--dimensional. The rest is easy.
\end{proof}

\begin{definition}
    Let $L$ be a Stone frame. Define $\DEFSYM{Bc}{\Bc} L$ to be the set of all complemented elements of $L$ and for a frame homomorphism $f\colon L \to M$ define
    $$\Bc f = \restr{f}{\Bc L}\colon \Bc L \to \Bc M.$$
\end{definition}
% TODO somehow mention that $f$ is between stone spaces

From the fact that a homomorphic image of a complemented element is a complemented element and since join or meet of two complemented elements is again complemented (to see that, let $a$ and $b$ be two complemented elements, then
    \begin{align*}
        (a\wedge b) \vee (a^c \vee b^c)
            &= (a\vee (a^c\vee b^c))\wedge (b\vee (a^c\vee b^c))
            = 1 \wedge 1 = 1, \text{ and} \\
        (a\wedge b) \wedge (a^c \vee b^c)
            &= ((a\wedge b)\wedge a^c)\vee ((a\wedge b)\wedge b^c)
            = 0 \vee 0 = 0;
    \end{align*}
and similarly $a^c\wedge b^c$ is complement of $a\vee b$.), one can see that $\Bc f$ is well--defined Boolean homomorphism.

% TODO some comment's before this statement
\begin{observation}
    $\Bc\colon \StoneFrm \to \Bool$ is a functor.
\end{observation}

\num For $B \in \Bool$ define \index{istar@$i_*$}$i_B\colon B \to \Bc\J(B)$ as follows
    $$i_B\colon b \mapsto \downset b.$$
    The definition is sound by Lemma \ref{p:complIdeal} and $i_B$ is a Boolean homomorphism, indeed $\downset a \vee \downset b = \downset (a \vee b)$, $\downset a \wedge \downset b = \downset (a \wedge b)$, and $\downset 1 = B$ respectively $\downset 0 = \{0\}$ is top respectively bottom of $\Bc\J(B)$.

From Lemma \ref{p:complIdeal} we also see that $i_B$ is an isomorphism and therefore the following diagram commutes

\begin{diagram}
    A \ar{r}{i_A} \ar{d}[swap]{f} & \Bc\J(A) \ar{d}{\Bc\J(f)}\\
    B \ar{r}{i_B}                 & \Bc\J(B)
\end{diagram}

\noindent for any Boolean homomorphism $f\colon A \to B$. From previous we have

\begin{proposition*}
    The collection $i_*$ of Boolean homomorphisms forms a natural equivalence between $\Bc\J$ and the identity functor on \Bool.
\end{proposition*}

% TODO better wording
\num Similarly for $L \in \StoneFrm$ we have a mapping \index{vstar@$v_*$}$v_L\colon \J\Bc(L) \to L$ defined as
    $$v_L\colon I \mapsto \bigvee I,$$
    and a mapping in the opposite direction $\iota\colon L \to \J\Bc(L)$:
    $$\iota\colon e \mapsto \downset e \cap \Bc L.$$

We can see that both $v_L$ and $\iota$ are monotone maps, $v_L \iota = \id_L$ (by null--dimensionality and regularity of $L$) and $\id_{\J\Bc(L)} \subseteq \iota v_L$, therefore $v_L$ is the left Galois adjoint to $\iota$, hence $v_L$ preserves all suprema.
Since $\bigvee I_1 \wedge \bigvee I_2 = \bigvee \Set{a_1 \wedge a_2 | a_i \in I_i} \leq \bigvee \Set{ a | a \in I_1 \cap I_2 } = \bigvee (I_1 \cap I_2) \leq \bigvee I_1 \cap \bigvee I_2$, $v_L$ also preserves finite infima which makes it a frame homomorphism.

Finally, $\id_{\J\Bc(L)} = \iota v_L$: take any $x \in \iota v_L(I)$. From the definitions we immediately see that $x \leq \bigvee I$ and $x$ is complemented in $L$. By the fact that
    $$1 = x \vee x^c \leq \bigvee I \vee x^c$$
    and by compactness of $L$ there is a finite $F \subseteq I$ such that $\bigvee F \vee x^c = 1$. Since $x = 1 \wedge x = (\bigvee F \vee x^c) \wedge x = (\bigvee F \wedge x) \vee (x^c \wedge x) = \bigvee F \wedge x$ we get that $x \leq \bigvee F$ and therefore $x \in I$.

Previous observations give us the fact that $v_L$ is an isomorphism of $L$ and $\J\Bc(L)$ and also that for any homomorphism of Stone frames $f\colon L \to M$ the following diagram commutes

\begin{diagram}
    \J\Bc(L) \ar{d}[swap]{\J\Bc(f)} \ar{r}{v_L} & L \ar{d}{f} \\
    \J\Bc(M) \ar{r}{v_M}    & M
\end{diagram}

\noindent Again as conclusion of previous paragraphs we obtain

\begin{proposition*}
    The collection $v_*$ of frame homomorphisms is a natural equivalence between $\J\Bc$ and identity functor on \StoneFrm.
\end{proposition*}

\num Using previous facts we get the main result of this section.

\begin{theorem*}
    Functor \Bc{} is the right adjoint to the functor \J{} with $i_*$ as unit of adjunction and $v_*$ as counit. Moreover, \Bc{} and \J{} constitute an isomorphism of categories \StoneFrm{} and \Bool.
\end{theorem*}

As we will see in next chapter, \J{} corresponds precisely to compactification of Boolean algebras the same way as the space of ultrafilters is part of Stone duality and compactification for spaces. (TODO better wording)

TODO (note about AC:) As one can check, the whole correspondence is given in constructive setting. No use of Axiom of Choice unlike in classical case, \dots

\section{Stone duality for \StoneSp}

\num Similarly to Frame case:
\begin{definition*}
    We say a topological space is \DEF{Stone space} if it is compact, Hausdorff and null--dimensional.
\end{definition*}

\num Let $B$ be a Boolean algebra, define
    $$
    X_B  = \Set{F \subseteq B | F \text{ is an ultrafilter on } B}
    \quad\text{ and }\quad
    \tau_B = \Set{ O_I | I \in \J B},
    $$
\noindent where \DEFSYM{OI}{$O_I$} is the set $\Set{ F \text{ ultrafilter} | F \cap I \neq \emptyset }$. Now $(X_B, \tau_B)$ is a topological space:
\begin{itemize}
    \item $O_I \cap O_J = O_{I\cap J}$: $F \in O_I\cap O_J$ iff $F\cap I \neq \emptyset$ and $F\cap J \neq \emptyset$ iff $F \cap I \cap J \neq \emptyset$ iff $F\in O_{I\cap J}$.
    \item $\bigcup_i O_{J_i} = O_{\bigvee_i J_i}$: $\supseteq$ holds trivially since $\bigcup J_i \subseteq \bigvee J_i$. On the other hand, for an ultrafilter $F$ such that $F\cap \bigvee_i J_i \neq \emptyset$, there is $e = \bigvee E$ where $E \subseteq \bigcup J_i$ finite and $e \in F$. Since $F$ is prime, there is $e' \in E$ such that $e' \in F$ and therefore $F\cap J_j \neq \emptyset$ for some $J_j \ni e'$.
    \item $\tau_B$ contains $O_{\downset 0} = \emptyset$ and $O_{\downset 1} = X_B$.
\end{itemize}

\noindent We will denote \DEFSYM{S B}{$\Sp B$} to be $(X_B, \tau_B)$.

\begin{lemma}\label{p:SpObjects}
    Let $B$ be a Boolean algebra, $\Sp B$ is a Stone space.
\end{lemma}
\begin{proof}
    Compactness and null--dimensionality follows directly from compactness and null--dimensionality of $\J B$.

    To show $X_B$ is also Hausdorff, take any ultrafilters $E \neq F$. Without lost of generality there is some $e \in E \setminus F$. We have $e\vee e^c = 1$, since $F$ is an ultrafilter we have $e^c \in F$. We also have $e\wedge e^c = 0$ and so $O_{\downset e}$ and $O_{\downset e^c}$ separates $E$ and $F$.
\end{proof}

\begin{lemma}
    Let $f\colon A \to B$ be a Boolean homomorphism and let $F$ be an ultrafilter on $B$. Then $f^{-1}[F]$ is an ultrafilter on $A$.
\end{lemma}
\begin{proof}
    Set $E = f^{-1}[F]$. Firstly, we will show $E$ is a filter. Trivially $1 \in E$ and $0 \not\in E$. For $x, y \in E$, there are $x', y' \in F$ such that $x' = f(x)$ and $y' = f(y)$. Hence $x'\wedge y' = f(x) \wedge f(y) = f(x\wedge y)$ and $x\wedge y \in E$. For the upwards closeness, take any $x \in E$ and $y \geq x$. $y \in E$ as $f(y) \geq f(x) \in F$.

    $E$ is also an ultrafilter. For $a, b\in A$ such that $a\vee b \in E$, $f(a)\vee f(b) = f(a\vee b) \in F$ and so $f(a)$ or $f(b)$ is in $F$ and therefor $a$ or $b$ is in $E$.
\end{proof}

\num\label{p:SpMorphisms} Let $f\colon A \to B$ be a Boolean homomorphism, denote $\Sp f\colon \Sp B \to \Sp A$ to be the map defined as
    $$\Sp f\colon F \mapsto f^{-1}[F].$$

\noindent From previous Lemma, we see the definition is sound. We will show $\Sp f$ is also continuous. Take any $O_I \in \tau_B$, we have
\begin{align*}
    (\Sp f)^{-1}[O_I] &= \Set{ (\Sp f)^{-1}(F) | F \cap I \neq \emptyset } \\
                      &= \Set{ E | \exists F \subseteq B \text{ ultrafilter, } (\Sp f)(E)=F, \text{ and } F \cap I \neq \emptyset } \\
                      &= \Set{ E | f^{-1}(E)\cap I \neq \emptyset }
                       = \Set{ E | E\cap f[I] \neq \emptyset } \\
                      &= \Set{ E | E\cap \downset f[I] \neq \emptyset }
                       = O_{\J(h)(I)}.
\end{align*}

\begin{theorem}
    $\Sp\colon \Bool \to \StoneSp$ is a functor.
\end{theorem}
\begin{proof}
    Follows immediately from~\ref{p:SpObjects} and~\ref{p:SpMorphisms}.
\end{proof}

\num Our situation is as follows

\begin{diagram}
    \StoneFrm \ar[bend left=15]{rr}{\Bc} & & \Bool{} \ar[bend left=15]{ll}{\J} \ar{ldd}{\Sp} \\
    \\
    & \StoneSp \ar{uul}{\Omega}
\end{diagram}

\begin{proposition}
    The collection of morphisms $\pi_B\colon \J B \to \Omega\Sp(B)$ defined $I \mapsto O_I$, for $B \in \Bool$, constitutes a natural equivalence $\J \cong \Omega\circ\Sp$.\ACP
\end{proposition}
\begin{proof}
    From the definition of $\Sp B$ we can see that $\pi_A$ is onto frame homomorphism.

    Take any $I \neq J$, $I,J \in \J B$. Without lost of generality take $e \in I \setminus J$. Now the filter $\upset e$ is disjoint with the ideal $J$ and by Boolean Ultrafilter Theorem there exists an ultrafilter $U \supseteq \upset e$ disjoint with $J$. Also $U \cap I$ is not empty, it contains $e$, and therefore $O_I \neq O_J$.

    To prove naturalness of $\pi_*$ the following diagram have to commute

    \begin{diagram}
        \J A \ar{r}{\pi_A} \ar{d}[swap]{\J f} & \Omega\Sp(A) \ar{d}{\Omega\Sp(f)}\\
        \J B \ar{r}{\pi_B}                    & \Omega\Sp(B)
    \end{diagram}

    \noindent for any Boolean homomorphism $f\colon A \to B$. We have
    $$ \Omega\Sp(f)(O_I) = (\Sp f)^{-1}[O_I] = O_{\J(h)(I)},$$

    \noindent where both equalities follows from~\ref{p:SpMorphisms}.
\end{proof}

\begin{conclusion}
    $\Bc\circ\Omega\circ\Sp \cong \Id_\Bool$.\ACP
\end{conclusion}
\begin{proof}
    By previous Proposition and the Stone correspondence for Stone frames and Boolean algebras we have $\Bc\circ\Omega\circ\Sp \cong \Bc\circ\J \cong \Id_\Bool$.
\end{proof}

\begin{proposition}
    The collection of morphisms $\rho_X\colon X \to \Sp\Bc\Omega(X)$ defined $x \mapsto F_x = \Set{U\text{ clopen} | x\in U }$, for $X \in \StoneSp$, constitutes a natural equivalence $\Id_\StoneSp \cong \Sp\circ\Bc\circ\Omega$.
\end{proposition}
\begin{proof}
    First observe $U$ is clopen in $\Sp\Bc\Omega(X)$ iff $U = \downset a$ for some $a \in \Bc\Omega(X)$ iff $a$ is clopen in $\Omega(X)$ iff $a$ is clopen in $\tau_X$.

    We will show $\rho_X$ is homeomorphism.
    \begin{itemize}
        \item $\rho_X$ is one--one: For two points $x_1 \neq x_2$ of $X$, from Hausdorff property there are $U_1, U_2$ such that $U_1\cap U_2 = \emptyset$ and $x_i\in U_i$. From null--dimensionality of $X$ there are two clopen subsets $M_1 \subseteq U_1, M_2 \subseteq U_2$, and $x_i \in M_i$. Hence $F_{x_1} \neq F_{x_2}$.

        % TODO avoid using knowledge about compact spacese?
        \item $\rho_X$ is onto: Take any $F$ ultafilter. From the observation we know, $F$ is also ultrafilter on $X$. Since $\Sp\Bc\Omega(X)$ is compact, there exists convergent point $x$ of $F$. And so $F \subseteq F_x$. But $F$ is an ultrafilter, maximal filter, and therefore $F = F_x$.

        \item $\rho_X$ and $\rho_X^{-1}$ are continuous:
        \begin{align*}
            \rho_X^{-1}[O_I] &= \Set{ \rho_X^{-1}(F_x) = x | F_x \cap I \neq \emptyset} 
                = \Set{ x | x \in M \in I } = \bigcup I \in \tau_X \\
            \rho_X[O] &= \Set{ F_x | x \in M \subseteq O, M \text{ is clopen}} \\
                      &= \Set{ F | F \cap (\downset O \cap \Bc\Omega(X)) \neq \emptyset} = O_{\downset O \cap \Bc\Omega(X)} \in \tau_{\Sp\Bc\Omega(X)}.
        \end{align*}
    \end{itemize}

    And finally $\rho_*$ is natural equivalence. The following diagram commutes
    \begin{diagram}
        X \ar{r}{\rho_X} \ar{d}[swap]{f} & \Sp\Bc\Omega(X) \ar{d}{\Sp\Bc\Omega(f)}\\
        Y \ar{r}{\rho_Y}                 & \Sp\Bc\Omega(Y)
    \end{diagram}
    \noindent for any continuous map $f\colon X \to Y$. Indeed
    \begin{align*}
        \Sp\Bc\Omega(f)(F_x)
            &= \Bc\Omega(f)^{-1}[F_x] = \Omega(f)^{-1}[F_x] \\
            &= \Set{ \Omega(f)^{-1}(M) | M \text{ clopen}, x \in M } \\
            &= \Set{ N | x \in M, M \text{ clopen}, f^{-1}[N] = M } \\
            &= \Set{ N | x \in f^{-1}[N] } = \Set{ N | f(x) \in N } = F_{f(x)} = \rho_Y(f(x))
    \end{align*}
\end{proof}

\begin{theorem}
    Functors $\Omega\circ\Bc$ and \Sp{} are mutually inverse. Specially categories \Bool{} and $\StoneSp^{\text{op}}$ are isomorphic.\ACP
\end{theorem}

\section{Parts of duality}

\subsection{$\kappa$--complete Boolean algebras}
For the rest of this section, let $\kappa$ be some fixed infinite cardinal.

\begin{definition}
    A lattice is \DEF{$\kappa$--complete}, for some cardinal $\kappa$, if any of its subset of cardinality at most $\kappa$ has a supremum and an infimum. We say a homomorphism is $\kappa$--complete if it preserves all supremas and infimas of subsets of cardinality at most $\kappa$.
\end{definition}

Note, our definition is a little nonstandard, what we mean by $\kappa$--completeness is usually called $\kappa^+$--completeness.

\num The following Lemma will be handy for computations.
\begin{lemma*}\label{p:idealsFrame} Let $L$ be a frame. Then:
    \begin{enumerate}
        \item For all $J \in \J L$: $J^* =\;\downset (\bigvee J)^*$.
        \item For all $a \in L$: $(\downset a)^* = \downset a^*$ in $\J L$.
        \item If $L$ is Boolean, then for all $J \in \J L$: $J^{**} =\;\downset \bigvee J$.
    \end{enumerate}
\end{lemma*}
\begin{proof}
    We will prove just the first case, the others follow directly from it. Let $J$ be any ideal on $L$. Observe that $a \wedge \bigvee J = 0 $ iff $\downset a \wedge J = 0_{\Bo L}$. Since
        \begin{align*}
            J^* & = \bigvee \Set{ L | L \wedge J = 0_{\Bo L} } = \bigcup \Set{ \downset a | \downset a \wedge J = 0_{\Bo L} } \text{ and}\\
            (\bigvee J)^* & = \bigvee \Set{ a | a \wedge \bigvee J = 0 },
        \end{align*}

    \noindent we see that $\downset a \subseteq J^*$ iff $a \in J^*$ iff $a \leq (\bigvee J)^*$, hence $J^* = \downset (\bigvee J)^*$.
\end{proof}


\begin{definition}
    Let $L$ be a Stone frame and let $s$ be a element of $L$. We say $s$ is a \DEF{$\kappa$--generated} if $s = \bigvee S$ for some $S \subseteq \Bc L$ of cardinality at most $\kappa$.

    Let $L$ be a Stone frame, we say $L$ is \DEF{$\kappa$--basically disconnected} if $m^{**} \vee m^* = 1$ for all $\kappa$--generated elements $m \in L$.
\end{definition}

\begin{lemma}\label{p:kappaCompleteBA}
    If $B$ is a $\kappa$--complete Boolean algebra, then $\J B$ is $\kappa$--basically disconnected Stone frame.
\end{lemma}
\begin{proof}
    By the Lemma~\ref{p:complIdeal} we know that complemented ideals are precisely principal ideals. Given any $M \subseteq B$, such that $|M| \leq \kappa$ and $I = \bigvee \Set{ \downset a | a \in M }$, we will show that $I^{**} \vee I^* = 1_{\J B}$.

    Define $m = \bigvee M$ and $J = \downset m$. Observe that $I^{**} = J$: Trivially from Lemma~\ref{p:idealsFrame} $I^{**} \subseteq J^{**} = J$ and $J \subseteq I^{**}$ follows from
    $$ I^* = \bigcup \Set{ \downset a | \downset a \wedge I = 0_{\J B}} = \bigcup \Set{ \downset a | a \wedge m = 0 } = (\downset m)^* = J^*.$$

    Consequently, $I^{**} \vee I^* = J \vee J^* = \downset m \vee \downset m^c = 1_{\J B}$.
\end{proof}

\num The following Lemma will be very useful and we will use it without further referencing.
\begin{lemma*}
    Let $B$ be a Boolean algebra such that every of its subset of cardinality at most $\kappa$ has a supremum. Then $B$ is $\kappa$--complete.

    Consequently: Any Boolean homomorphism preserving all $\kappa$--meets (or $\kappa$--joins) is $\kappa$--complete.
\end{lemma*}
\begin{proof}
    Let $S$ be an arbitrary subset of $B$ such that $|S| \leq \kappa$. Set $M = (\bigvee \Set{ b^c | b \in S})^c$, we will show that $M$ is the infimum of $S$.

    Take any $a \in S$. We have $a \wedge M = a^{cc}\wedge (\bigvee \Set{ b^c | b \in S})^c = (a^c\vee \bigvee \Set{ b^c | b \in S})^c = M$. Hence $a \geq M$.

    Now suppose $m \leq a$ for all $a \in S$. Then $m\wedge M = (m^c\vee \bigvee \Set{b^c | b \in S})^c = (m^c)^c = m$. Hence $m \leq M$.
\end{proof}

\begin{lemma}\label{p:kappaComplStoneFrm}
    If $L$ is a $\kappa$--basically disconnected Stone frame, then $\Bc L$ is a $\kappa$--complete Boolean algebra. Moreover, the joins in $\Bc L$ are defined by the following formula

    $$\bigsqcup M = (\bigvee M)^{**}.$$
\end{lemma}
\begin{proof}
    For $M \subseteq \Bc L$ and $|M| \leq \kappa$, set $m = \bigvee M$. Since $L$ is $\kappa$--basically disconnected we have $m^{**} \vee m^* = 1$ and therefore $m^{**} \in \Bc L$. So $m^{**}$ is an upper bound for $M$ in $\Bc L$.

    Now let $n$ be an arbitrary upper bound for $M$ in $\Bc L$, thus $n$ is an upper bound in $L$ also, but $m \leq n$ since $m$ is the supremum of $M$ in $L$. Which gives us the desired relation $m^{**} \leq n^{**} = n$, hence $m^{**}$ is the supremum of $M$ in $\Bc L$.
\end{proof}

\begin{observation}
    Let $f\colon A \to B$ be a $\kappa$--complete Boolean homomorphism and let $S$ be a subset of $A$ such that $|S| \leq \kappa$. Set $I$ to be the ideal generated by $S$, then
    $$(\J f)(I^*) = (\J f)(I)^*.$$
\end{observation}
\begin{proof}
    The result is obtained by simple computation
    \begin{align*}
        (\J f)(I^*) &= (\J f)(\downset (\bigvee I)^*) = \downset f((\bigvee I)^*) \\
                &= \downset f(\bigvee I)^* & \text{($f$ is a Boolean homomorphism)} \\
                &= \downset (\bigvee f[I])^* & \text{($\kappa$--completeness of $f$)} \\
                &= \downset (\bigvee \downset f[I])^* = \downset (\bigvee (\J f)(I))^* \\
                &= (\J f)(I)^*
    \end{align*}
    Where the $\kappa$--completeness in the third step can be used because $I = \bigvee \Set{ \downset s | s \in S }$ and $|S| \leq \kappa$.
\end{proof}
% TODO more elegant

% As we will see after a while, the above Observation characterises morphisms in the image of $\kappa$--complete part of Boolean algebras in Stone correspondence.

\begin{definition}
    Let $f\colon L \to M$ be a homomorphism between Stone frames. We say $f$ is a \DEF{$\kappa$--basically complete} if $f(a^*) = f(a)^*$ holds for all $\kappa$--generated elements $a \in L$.
\end{definition}

In other words the last observation states that \J{} sends any $\kappa$--complete Boolean homomorphism to a $\kappa$--basically complete frame homomorphism. As we will see in the following Lemma, morphisms in the image of $\kappa$--complete part of Boolean algebras in Stone correspondence are characterised precisely this way.

\begin{lemma}
    Let $f\colon L \to M$ be a $\kappa$--basically complete frame homomorphism, then $\Bc f\colon \Bc L \to \Bc M$ is a $\kappa$--complete Boolean homomorphism.
\end{lemma}
\begin{proof}
    Let $A$ be an arbitrary subset of $\Bc L$ such that $|A| \leq \kappa$. We have
    \begin{align*}
        (\Bc f)(\bigsqcup A) &= f((\bigvee A)^{**}) \\
            &= f((\bigvee A)^*)^* & \text{($(\bigvee A)^*$ is complemented)}\\
            &= f(\bigvee A)^{**}  & \text{($\bigvee A$ is $\kappa$--generated)}\\
            &= (\bigvee f[A])^{**} & \text{($f$ is a frame homomorphism)}\\
            &= \bigsqcup f[A] = \bigsqcup (\Bc f)[A].
    \end{align*}
    Therefore $\Bc f$ is $\kappa$--complete.
\end{proof}

\num From the Lemmas~\ref{p:kappaCompleteBA} and~\ref{p:kappaComplStoneFrm} we see that restriction of classical Stone correspondence to subcategories of $\kappa$--complete Boolean algebras on one side and $\kappa$--basically disconnected Stone frames on the other (without any restriction on morphisms) is still a duality of categories. However, we can strengthen the result and get the following

\begin{theorem*}
    The category of $\kappa$--basically disconnected Stone frames and $\kappa$--basically complete frame homomorphisms is isomorphic to the category of $\kappa$--complete Boolean algebras and $\kappa$--complete Boolean homomorphisms.
\end{theorem*}
\begin{proof}
    The only thing we need to show is that the morphisms of natural equivalences for identity functors and functors $\Bc\J$ and $\J\Bc$ are morphisms of our categories.

    For the first part, we will show $i_B\colon B \to \Bc\J(B)$ is a $\kappa$--complete Boolean homomorphisms for any $\kappa$--complete Boolean algebra $B$. Let $(b_j)_{j < \kappa} \subseteq B$, then by simple computation we get

    $$
        \bigsqcup i_B(b_j) = \bigsqcup \downset b_j = (\bigvee \downset b_j)^{**} = \downset (\bigvee b_j) = i_B(\bigvee b_j),
    $$

    \noindent where the third equality follows from~\ref{p:idealsFrame}.

    For the second part, we need to show $v_L\colon \J\Bc(L) \to L$ is $\kappa$--basically complete for any $\kappa$--basically disconnected Stone frame $L$. We will show $v_L$ is $\lambda$--basically complete for any cardinal $\lambda$. Take any $I \in \J\Bc(L)$, we have

    \begin{align*}
        v_L(I^*) &= v_L(\bigvee \Set{ \downset s | \downset s \wedge I = 0 })
                  = \bigvee \Set{ v_L(\downset s) | \downset s \wedge I = 0 } \\
                 &= \bigvee \Set{ s | \downset s \wedge v_L(I) = 0 } = v_L(I)^*,
    \end{align*}

    \noindent which finishes the proof.
\end{proof}

Equivalently in topological spaces, space is $\kappa$--basically disconnectedness iff any union of less than $\kappa$ clopen sets has open closure. In classical setting there is a duality between $\kappa$--complete Boolean algebras and $\kappa$--basically disconnected Stone spaces~\cite{monk1989handbook}.

% However ... (TODO discuss corresponding morphisms in Top)


\subsection{Complete Boolean algebras}
TODO say something about geometrical/topological meaning of complete Boolean algebras (meaning that ComplBool are frames...).

\begin{proposition}
    Let $B$ be a complete Boolean algebra. The frame $\J B$ is an extremally disconnected Stone frame.
\end{proposition}
\begin{proof}
    In any Boolean algebra the relations $\rbelow$ and $\leq$ coincide. Moreover, each complete Boolean algebra is a (completely regular extremally disconnected) Boolean frame. Thus $\J{B}$ equals $\R{B}$.

    From Lemma \ref{p:extrDiscPreserv} we know compactification preserves extremally disconnectedness and therefore \J{B} is also extremally disconnected. From Lemma \ref{p:JisFunctor} we know it is also a Stone frame.
\end{proof}

\num As \DEF{\ComplBool{}} denote the category of all complete Boolean algebras and complete Boolean homomorphisms preserving all meets and joins.

\num TODO Recall the definition of Booleanization (\Bo). % TODO add \DEF

\begin{definition}
    Let $H$ be a Heyting algebra, by $\Bo H$ denote the set $\Set{ a^{**} | a \in H }$.

    And for a Heyting homomorphism $f\colon H \to K$, set $\Bo f\colon \Bo H \to \Bo K$ to be the mapping
    $$\Bo f\colon a \mapsto f(a)^{**}.$$
\end{definition}

\begin{proposition}
    Let $H$ be a Heyting meet--semilattice, then $\Bo H$ is a Boolean algebra with joins and meets defined
    $$ a \sqcup b = (a^* \wedge b^*)^* \quad\text{and}\quad a\sqcap b = a \wedge b.$$
\end{proposition}
\begin{proof}
    First, we will show that $\sqcap$ really is the meet and $\sqcup$ is the join. For $a,b,c \in \Bo H$:

    \begin{itemize}
        \item $a\sqcup b = a^{**}\wedge b^{**} = (a\wedge b)^{**} \in \Bo H$.
        \item Whenever $a,b \leq c$, then $a^*, b^*\geq c^*$ and so $a^*\wedge b^*\geq c^*$, hence $a\sqcap b = (a^*\wedge b^*)^* \leq c^{**} = c$. Trivially $a\sqcap b \in \Bo H$.
    \end{itemize}

    For $a \in \Bo H$, $a\sqcup b = a^{**}\sqcup b = (a^{**}\wedge a^*)^* = 0^* = 1$ and also $a\sqcap a^* = a\wedge a^* = 0$, hence each element is complemented.
\end{proof}

\num\label{p:propertiesBooleanization}
    Let $L$ be a frame, the mapping \DEFSYM{BetaL}{$\beta_L\colon$}$ L \to \Bo L$ defined $a \mapsto a^{**}$ is a nucleus:
    \begin{enumerate}[label=(N\arabic*)]
        \item From the definition of pseudocomplement we have $a \wedge a^* = 0$ iff $a \leq a^{**}$ hence $a \leq a^{**}$;
        \item Pseudocomplement is antitone hence $(a \leq b \implies a^{**} \leq b^{**})$;
        \item $a^* = a^{***}$ implies $a^{**\;**} = a^{**}$; and
        \item $(a \wedge b)^{**} = a^{**} \wedge b^{**}$ is standard equality (Lemma~\ref{??}).
    \end{enumerate}

    Therefore by~\label{p:nuclProp}, $\Bo L$ is a sublocale of $L$ and consequently a complete Boolean algebra. From~\label{p:nuclProp} we also have $\beta_L$ is a frame homomorphism.

\begin{block}{Note}
    $\Bo L$ is the smallest dense sublocale of $L$; and joins are given also by the following formula $(a \vee b)^{**}$, since $a \mapsto a^{**}$ is a nucleus.
\end{block}

\begin{lemma}
    Let $f\colon L \to M$ be a frame homomorphism and $a \in L$ such that $a^{**}\vee a^* = 1$, then
    $$ f(a^{**})^* = f(a^*).$$

    In particular for $a = a^{**}$ we have
    $$ f(a)^* = f(a^*).$$
\end{lemma}
\begin{proof}
    From the assumptions we see the following holds
    \begin{align*}
        f(a^{**} \vee a^*) & = f(a^{**}) \vee f(a^*) = 1, \\
        f(a^{**} \wedge a^*) & = f(a^{**}) \wedge f(a^*) = 0.
    \end{align*}
    Hence $f(a^{**})$ is complement to $f(a^*)$ and distributivity of $M$ gives us $f(a^*) = f(a^{**})^*$.
\end{proof}

\begin{observation}
    $f(a^{**})^* = f(a^*)$ implies $f(a^{**}) = f(a^{**})^{**}$.
\end{observation}
\begin{proof}
    $f((a^*)^*) = f((a^*)^{**})^* = f(a^*)^* = f(a^{**})^{**}$.
\end{proof}

The following lemma is taken from ... (TODO cite Booleanization).

\begin{lemma}
    Let $f\colon L \to M$ be a frame homomorphism, then $\Bo f$ is a frame homomorphism such that the following diagram commutes
    \begin{diagram}
        L \ar{r}{\beta_L} \ar{d}{f} & \Bo L \ar{d}{\Bo f}\\
        M \ar{r}{\beta_M}           & \Bo M
    \end{diagram}
    if and only if $f(a^{**}) \leq f(a)^{**}$.
\end{lemma}
\begin{proof}
    $$
    (\Bo f)(\bigsqcup A) = (\bigvee f[A])^{**} = f(\bigvee A)^{**} \geq f((\bigvee A)^{**}) = f((\bigvee A)^{**})^{**} = (\Bo f)(\bigsqcup A).
    $$


    But
    $$
    (\Bo f)(\bigsqcup A) \leq f(\bigvee A)^{**} \leq f(\bigvee A).
    $$

    TODO: We should have proved $(\Bo f)(\bigsqcup A) = \bigsqcup(\Bo f)[A]$ instead.
\end{proof}

\begin{lemma}
    If $f(a^{**})^* = f(a^*)$ and $f(a^{**}) \leq f(a)^{**}$, then $f(a^*) = f(a)^*$.
\end{lemma}
\begin{proof}
    The $f(a^*) \leq f(a)^*$ is always true and $f(a^*) = f(a^{**})^* \geq f(a)^{***} = f(a)^*$.
\end{proof}

\begin{theorem}
    $\Bo\colon \ExtrStoneFrm \to \ComplBool$ is a functor.
\end{theorem}
\begin{proof}
    From~\ref{p:propertiesBooleanization} we know $\Bo L$ is a complete Boolean algebra.

    TODO $\Bo f$ is a complete Boolean homomorphism.

    As a direct implication of the previous Lemmas we get that for any frame homomorphism between two extremally disconnected Stone frames $f\colon L \to M$ the following diagram commutes.

    \begin{diagram}
        L \ar{r}{\beta_L} \ar{d}{f} & \Bo L \ar{d}{\Bo f}\\
        M \ar{r}{\beta_M}           & \Bo M
    \end{diagram}

    \noindent And therefore for any frame homomorphisms $f, g$ the following diagram also commutes.

    \begin{diagram}
        L \ar{r}{\beta_L}
          \ar{d}{f}
          \ar[bend right]{dd}[swap]{gf} &
        \Bo L \ar{d}[swap]{\Bo f}
              \ar[bend left]{dd}{\Bo (gf)}\\

        M \ar{r}{\beta_M} \ar{d}{g} & \Bo M \ar{d}[swap]{\Bo g}\\
        N \ar{r}{\beta_N}           & \Bo N
    \end{diagram}

    Finally for any identity frame homomorphism $i_L$, $\Bo i_L$ is the identity on $\Bo L$. Consequently $\Bo$ is a functor.
\end{proof}

\begin{proposition}
    Let $B$ be a Boolean frame, then $B \cong \Bo\J(B)$.
\end{proposition}
\begin{proof}
    From the previous Lemma we see that $\J \in \Bo\J(B)$ iff $J = J^{**} = \downset \bigvee J$. On the other hand, for $a \in B$: $(\downset a)^{**} = \downset a^{**} = \downset a$.

    (FIXME:) Denote $\tilde i\colon B \to \Bo\J(B)$ the mapping $a \mapsto \downset a$. It is a Boolean homomorphism, because $\downset a \vee \downset b = \downset (a \vee b)$, $\downset a \wedge \downset b = \downset (a \wedge b)$, $\downset 0 = \{0\} = 0_{\Bo\J(B)}$ and $\downset 1 = B = 1_{\Bo\J(B)}$. Consequently $\tilde i$ is an isomorphism of $B$ and $\Bo\J(B)$.
    % TODO $\tilde i$ need to preserve big joins, we get that by \bigsqcup I_i = (\bigcup I_i)^{**}
\end{proof}

\begin{proposition}
    Let $L$ be an extremally disconnected Stone frame, then $L \cong \J\Bo(L)$.
\end{proposition}
\begin{proof}
    Similarly to the general case, define $\tilde v_L\colon \J\Bo(L) \to L$ and $\tilde\iota\colon L \to \J\Bo(L)$ as
    $$  \tilde v_L\colon I \mapsto \bigvee I \quad\text{and}\quad \tilde\iota\colon a \mapsto \downset a \cap \Bo L.$$

    Trivially we have $\tilde\iota \tilde v_L \supseteq \id_{\J\Bo(L)}$ and $\tilde v_L \tilde\iota \leq \id_L$. We again have the situation where $\tilde v_L$ is the left Galois adjoint to $\tilde\iota$ and $\bigvee I_1 \wedge \bigvee I_2 = \bigvee (I_1 \wedge I_2)$ for any two $I_1, I_2 \in \J\Bo(L)$ and so $\tilde v_L$ is a frame homomorphism.

    Actually we also have the opposite inequalities: $\tilde v_L \tilde\iota \geq \id_L$ because $a \rbelow x$ implies $a^{**} \rbelow x$ and $L$ is regular. For $\tilde\iota \tilde v_L \subseteq \id_{\J\Bo(L)}$, take any $x \in \tilde\iota \tilde v_L(I)$. We have $x \leq \bigvee I$, $x = x^{**}$ and from extremally disconnectedness $x^{**} \vee x^* = 1$, so
    $$ 1 = x \vee x^* \leq \bigvee I \vee x^*$$
    \noindent holds and from the compactness of $L$ there is a finite $F \subseteq I$ such that $\bigvee F \vee x^* = 1$. Now $x \leq \bigvee F$ since $x = 1 \wedge x = (x^* \vee \bigvee F) \wedge x = \bigvee F \wedge x$ and therefore $x \in I$.
\end{proof}

\num The following diagram commutes
    \begin{diagram}
        A \ar{r}{\tilde i_A} \ar{d}[swap]{f} & \Bc\J(A) \ar{d}{\Bc\J(f)}\\
        B \ar{r}{\tilde i_B}                 & \Bc\J(B)
    \end{diagram}

    $$\Bo\J(f)(\downset a) = (\downset f[\downset a])^{**} = \downset f(a)^{**} = \downset f(a)$$

    TODO $\tilde i_A$ is complete Boolean homomorphism.

\num The following also commutes
    \begin{diagram}
        \J\Bc(L) \ar{d}[swap]{\J\Bc(f)} \ar{r}{\tilde v_L} & L \ar{d}{f} \\
        \J\Bc(M) \ar{r}{\tilde v_M}    & M
    \end{diagram}
    $$(\J\Bo)(f)(I) = \downset \Set{ f(x)^{**} | x \in I } = \downset \Set{ f(x) | x \in I} = \downset f[I]$$

    \noindent Therefore $\tilde v_L((\J\Bo)(f)(I)) = \bigvee \downset f[I] = \bigvee f[I] = f(\bigvee I)$.

    TODO $\tilde v_L$ is basically complete frame homomorphism.

\num TODO By the previous isomorphism lemmas we have the natural equivalence between identity functor and \Bo\J{} resp. \J\Bo{} and get an isomorphism of categories \ComplBool{} and \ExtrStoneFrm, two subcategories of \Frm.

\num As conclusion the functors \Bo{} and \J{} form an adjunction, with unit and counit being isomorphism and so we have the following
\begin{theorem*}
    The categories \ExtrStoneFrm{} and \ComplBool{} are isomorphic.
\end{theorem*}

\num TODO note that \Bo{} is part of Stone duality only between \ExtrStoneFrm{} and \ComplBool{}, since for every frame $L$ the frame $\Bo L$ is a complete Boolean algebra.

\num TODO Since each Boolean frame is completely regular, \J{} corresponds precisely to compactification of this frame.

\subsection*{Comments}
TODO Mention $\sigma$-Frames.
TODO Compare compactification in classical and in point-free setting. Mention that \J{} correspond to compactification, the topology of Ult is isomorphic to ideal lattice of Boolean algebra and that this was known~\cite{monk1989handbook}.

\begin{center}
\begin{tikzpicture}
    \draw (1,3.2) node {$\StoneFrm$};
    \draw (7,3.2) node {$\Bool$};

    \draw[semithick] (1,0) ellipse (1.2 and 2.5);
    \draw[semithick] (7,0) ellipse (1.2 and 2.5);

    % Complete part
    \draw[semithick] (1.85,-1.7) arc (30:150:1.0cm);
    \draw[semithick] (7.85,-1.7) arc (30:150:1.0cm);

    % sigma part
    \draw[semithick] (2.12,0.8) arc (20:160:1.2cm);
    \draw[semithick] (8.12,0.8) arc (20:160:1.2cm);

    % kappa-complete part
    \draw[semithick] (2.20,-0.3) arc (25:155:1.32cm);
    \draw[semithick] (8.20,-0.3) arc (25:155:1.32cm);

    % Functors
    \draw[->,semithick] (2.5,2.0) -- (5.5,2.0) node[above,midway] {$\Bc$};
    \draw[->,semithick] (2.5,-0.5) -- (5.5,-0.5) node[above,midway] {$\Bc$};
    \draw[->,semithick] (2.5,-1.9) -- (5.5,-1.9) node[above,midway] {$\Bo$};
    \draw[<-,semithick] (2.5,-2.1) -- (5.5,-2.1) node[below,midway] {$\J$};
    \draw[<-,semithick] (2.5,0.8) -- (5.5,0.8) node[above,midway] {$\J$};

    % Categories captions
    \node at (-1, 1.2) {$\sigma$--\categoryStyle{BDStoneFrm}};
    \node at (1, 0.9) {$\vdots$};
    \node at (-1,-0.2) {$\kappa$--\categoryStyle{BDStoneFrm}};
    \node at (1,-0.6) {$\vdots$};
    \node at (-0.5,-2.0) {\ExtrStoneFrm};

    \node at (8.2, 1.2) {$\sigma$--\ComplBool};
    \node at (7, 0.9) {$\vdots$};
    \node at (8.2,-0.2) {$\kappa$--\ComplBool};
    \node at (7,-0.6) {$\vdots$};
    \node at (8,-2.0) {\ComplBool};
\end{tikzpicture}
\end{center}

\noindent Where $\kappa$--\ComplBool{} denotes the category of $\kappa$--complete Boolean algebras and $\kappa$--complete Boolean homomorphisms and $\kappa$--\categoryStyle{BDStoneFrm} denotes the category of $\kappa$--basically disconnected Stone frames and $\kappa$--basically complete frame homomorphisms.
