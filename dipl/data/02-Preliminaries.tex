\chapter{Preliminaries}
    We assume basic knowledge of Set Theory, mainly basic knowledge about cardinals and cardinalities, "axiom of choice".

\section{Partially ordered set}
\begin{itemize}
    \item Joins, meets, order, lattice, distributive lattice (unique solution to $a\vee x = b$ and $c\wedge x = d$)
    \item $\kappa$--complete and complete lattices and lattice homomorphisms % TODO check, we have already defined it in "Parts of duality"
    \item meet/join semilattice
    \item bottom (0) resp. top (1) and $0_S$ resp. $1_S$
    \item Galois correspondence (equivalent definitions)
    \item filters, ideals, principal filter/ideal
    \item pseudocomplements and complements (complemented elements), $(a \wedge b)^{**} = a^{**} \wedge b^{**}$, $a \leq b \implies a^* \geq b^*$.
    \item Boolean algebras and Boolean homomorphisms, \Bool, prime/maximal/ultrafilter on BA
    \item Heyting algebras, $\_^* = \_ \to 0$
    \item Axiom of Choice/Zorn's Lemma -- its importance/controversy
\end{itemize}
\section{Category Theory}
category, functor (important example: identity functor), natural transformation, natural equivalence, adjunction, units of adjunction
TODO exmplain importance of adjunction by mentioning the limit preserving of adjuction

\section{Topology and point--free Topology}
\begin{itemize}
    \item Basic of \Top{} and \Frm.
    \item frame homomorphisms, localic maps, continuous map, homeomorphism
    \item Subspace and sublocale/subframe, $\iota_S\colon L \to S$ as subframe inclusion and as onto frame homomorphisms? (the next section needs it)
    \item Spatiality.
    \item Separation axioms (regularity, complete regularity, normality, ...?), rather below (and useful facts about them)
    \item TODO place somewhere StoneSp and StoneFrm and \J.
\end{itemize}

Notation.
$\downset a^*$ means $\downset (a^*)$ (rightmost unary operator applies first)

For the definition of frames, we see that $\wedge$ is the left/right Galois adjoint, therefore every frame is a complete Heyting algebra, moreover also a pseudocomplemented lattice. The pseudocomplement operation for open set in spacial frames corresponds precisely to the complement of the closure of that set, or the interior of the complement.

\begin{definition}
    Let $L$ be a locale, \DEF{nucleus} $\nu$ on $L$ is a monotone map $\nu\colon L \to L$ having the following four properties:
    \begin{enumerate}[(N1)]
        \item $a \leq \nu(a)$,
        \item $a \leq b \implies \nu(a) \leq \nu(b)$,
        \item $\nu\nu(a) = \nu(a)$, and
        \item $\nu(a \wedge b) = \nu(a) \wedge \nu(b)$.
    \end{enumerate}
\end{definition}

\num\label{p:nuclProp} \textbf{Properties of nuclei.} For a locale $L$ and a nucleus $\nu\colon L \to L$, set $S = \nu(L)$. Then the set $S$ together with suprema and infima defined
    $$ \bigsqcup a_i = \nu(\bigvee a_i)\quad\text{ and }\quad a \sqcap b = a \wedge b$$

\noindent is a sublocale of $L$. Indeed, we have
$$ (\bigsqcup a_i)\sqcap b = \nu(\bigvee a_i) \wedge \nu(b) = \nu(\bigvee(a_i \wedge b)) = \bigsqcup (a_i \sqcap b).$$

\noindent From (N4) we know, $\sqcap$ is the infimum and from (N2) we know $\bigsqcup$ is the supremum. Hence $\nu(L)$ is a locale, to show that $\nu(L)$ is a sublocale of $L$ for $\nu^*\colon L \to \nu(L)$, defined $a \mapsto \nu(a)$, we will show it is an onto frame homomorphism. It is enough to show
    $$\nu(\bigvee a_i) = \nu(\bigvee \nu(a_i)) \quad( = \bigsqcup \nu(a_i)).$$

    \noindent We have $\bigvee a_i \leq \bigvee \nu(a_i)$ by (N1) and $\nu(\bigvee a_i) \leq \nu(\bigvee \nu(a_i))$ by (N2). On the other hand $\nu(a_i) \leq \nu(\bigvee a_i)$ by (N2). Hence $\bigvee \nu(a_i) \leq \nu(\bigvee a_i)$ and $\nu(\bigvee \nu(a_i)) \leq \nu\nu(\bigvee a_i) = \nu(\bigvee a_i)$ by (N2) and (N3).
