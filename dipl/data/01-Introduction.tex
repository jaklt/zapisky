\chapter{Introduction}

It was a breaktrough both in algebra and in topology when Marshall Stone published in 1936 his paper ``The Theory of Representation of Boolean Algebras''~\cite{stone1936theory}. In this article the author presented a structural isomorphism
between Boolean algebras and a certain class of topological spaces. Futhermore, the spaces we encounter there,
unlike those typically studied before, were not motivated by geometry (subsets of Euclidean spaces and spaces
from analysis, and similar) but arised from algebraic structures~\cite{johnstone1986stone}.

Frames (locales), the basic structure of point--free topology are a natural generalization of  the concept of a topological space. This generalization is based on the properties of the lattices of open sets of topological
spaces (and typically contains all the information one needs). The results in thus extended context often bring new insight into the classical facts. In contrast with the choice dependence of many of them, a large part of the theory of frames can be treated constructively. This phenomenon is aptly expressed by the famous slogan
in Banaschewski's~\cite{banaschewski1990proving}:

\smallskip

\centerline{\em Choice--free localic argument + suitable choice principle = }

\centerline{\em  classical result in spaces.}

\smallskip

\noindent (It should be noted, however, that there are numerous results going far beyond the classical context.)

\medskip

In this thesis we present the Stone representation theorem, generally known as Stone duality in the point--free context.
The proof is choice--free, and the Banaschewski's slogan is valid again: using the Boolean Ultrafilter Theorem we then can derive the classical spatial Stone duality. Not surprisingly, the Stone duality (to be more precise, equivalence)
is by far simpler than the original. The fact that we do not have to be concerned with points helps a lot.

\medskip

The point--free approach allows us to analyze easily some particular parts of the Stone duality. For each infinite cardinal $\kappa$ we show that the counterpart of the $\kappa$--complete Boolean algebras is constituted by the $\kappa$--basically disconnected Stone frames. We also present a precise characterization of the morphisms which correspond to the $\kappa$--complete Boolean homomorphisms.

\medskip

Booleanization is the construction of a Boolean algebra from a Heyting algebra by taking the set of elements of the form $a=a^{**}$ (for a topological space the Booleanization of its frame of open sets is the system of all regular open subsets; it forms a complete Boolean algebra). In point--free topology it has a useful and somewhat surprising property: it is the smallest dense sublocale of the original frame \cite{isbell1972atomless}. This fact has no counterpart in classical topology. 

Analogously with the Stone duality for topological spaces, the frames corresponding to the complete Boolean algebras are precisely the extremally disconnected Stone frames. Hence we obtain a proper class of variants of disconnectedness in between the zero-dimensionality and the extremal disconnectedness. Note that although the Booleanization is not functorial in general, in this part of the duality it is, and constitutes an equivalence of the parts in question.

\medskip 

 We finish the thesis by focusing on the De Morgan (or extremally disconnected) frames and present a new
 characterization of these. Let  us call a dense  sublocale $S$ of a frame $L$ {\em superdense} if the \v Cech--Stone compactifications of $S$ and $L$ are isomorphic. Now, a completely regular frame $L$ is De Morgan  if and only if
 each dense sublocale of $L$ is superdense. We also show  that in contrast with this phenomenon, a metrizable frame has no non-trivial superdense sublocale; in other words, a non-trivial  \v Cech--Stone compactification of a metrizable frame is never metrizable.


\subsection*{Organization of the thesis}

In Chapter II we present the necessary facts from order theory, category theory and topology which we will need in the subsequent chapters. In Chapter III several disconnectedness properties and the \v Cech--Stone compactification are discussed. In Chapter IV we present the Stone duality in both the point--free and the space setting. Chapter  V is devoted to particular parts (fragments) of the Stone duality. In Chapter VI we show that Booleanization is an equivalence of the``complete parts'' of the Stone duality. Finally, in Chapter VII we present a new characterization of De Morgan frames and prove the non-metrizability of non-trivial \v Cech--Stone compactification.


\subsection*{Acknowledgement}

My sincere gratitude goes to my supervisor Ale\v s Pultr for leading my work, for his kindness and patience,
for inspiring talks and lectures,
for invaluable proof--reading of the text (although, all remaining errors are mine),
and for helping with some of the proofs, in particular in the final parts of the thesis.

I would also like to thank my family for their support and my colleagues for creating an inspiring environment.

% for ready advice
% immense knowledge.

% TODO Consider:  So compact regular frm is de morgan <=> compactification of Booleanization is the original frame?
