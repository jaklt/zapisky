\chapter{Introduction}

In 1936 Marshall Stone made a great breakthrough in algebra and topology by publishing a paper called ``The Theory of Representations of Boolean Algebras''~\cite{stone1936theory}.
In this paper he described structural similarities between Boolean algebras and a certain class of topological spaces.
Besides that, he was also the first who constructed an example of a space which did not have the origin in geometry (as a subset of Euclidean space) and was constructed from an algebraic structure~\cite{johnstone1986stone}.

% The Stone representation theorem, or in other words the Stone duality, had a great impact in topology, functional analysis or representation theory of rings~\cite{johnstone1986stone}.
% TODO introduce Stone duality and Stone type dualities -- as an important example of a Stone type duality is Hofmann--Lawson's duality/... between .../spatial frames and sober? topological spaces

Frames and locales in point--free topology are natural generalizations of the notion of topological space.
These generalized spaces come out from an algebraical description of properties of the open set lattice.
Results from point--free topology often brings a new insight to classical results from general topology.
In contrast to general (set--theoretical) topology, a large part of the theory of frames and locales is given constructively.
This phenomenon is nicely expressed by the famous slogan of Banaschewski~\cite{banaschewski1990proving}:
\begin{center}\em
    Choice--free localic argument + suitable choice principle = classical result in spaces.
\end{center}


In this thesis, we present the Stone representation theorem, or later known as Stone duality, in point--free contex.
% In our case, it is a correspondence between Boolean algebras and Stone frames.
The slogan of Banaschewski has been fulfilled again even though we present the theorem in the language of frames instead of locales.
With Boolean Ultrafilter Theorem, we obtain the classical Stone duality for topological spaces.
It is not a surprise that the Stone duality in point--free setting happened to be a lot simpler than the original.
We do not have to be concerned about points.
% We work only with purely algebraic structures.

The point--free approach allow us to easily analyse particular parts of Stone duality.
For each infinite regular cardinal $\kappa$, we show that the counterpart to the $\kappa$--complete Boolean algebras on the frame side are the $\kappa$--basically disconnected Stone frames.
We also give a precise characterisation of the morphisms which correspond to the $\kappa$--complete Boolean homomorphisms.

Booleanization is the construction of Boolean algebra from Heyting algebra by taking the set of elements of the form $a = a^{**}$.
For a topological space, Booleanization of its frame of open sets is the set of all regular open subsets and it forms a complete Boolean algebra.
In point--free topology, one has an another useful property.
Booleanization of a locale is the smallest dense sublocale of that locale~\cite{banaschewski1996booleanization}.

Analogously to Stone duality for topological spaces, the frames corresponding to the complete Boolean algebras are precisely the extremally disconnected Stone frames.
Hence, we have a proper class of variants of disconnectedness in between zero--dimensionality and extremal disconnectedness. % TODO
Interestingly, Booleanization is functorial for this part of duality and it is an isomorphism of the complete parts of the duality.

At the end of the thesis we focus on De Morgan (or extremally disconnected) frames.
We show a new characterisation of completely regular De Morgan frames.
A completely regular frame is De Morgan if and only if each dense sublocale of the frame is superdense.
Counter to that are the compact metric frames, they have no non-trivial superdense sublocale.
Consequently, a non-trivial \v{C}ech--Stone compactification of metrizabile frame is never metrizabile.

% TODO Consider:  So compact regular frm is de morgan <=> compactification of Booleanization is the original frame?

\subsection*{Organization of the thesis}

In Chapter II we present necessary facts from order theory, category theory and topology which we will need in the subsequent chapters.
In Chapter III are discussed several disconnected properties and \v{C}ech--Stone compactification. % TODO
In Chapter IV we present the Stone duality in both point--free and set--theoretical setting.
Chapter V is devoted to particular parts of Stone duality.
In Chapter VI we show that Booleanization is an isomorphisms of the complete parts of Stone duality.
Finally, in Chapter VII we present a new characterisation of De Morgan frames and non--metrizability of non--trivial \v{C}ech--Stone compactification.

% Chapters II and III consist of well--known facts and I am not an author of any of them.
The theorems in the last chapter as well as the idea to take a look at Booleanization are due to Professor Pultr.

We assume from the reader experiences with abstract algebra, topology and set theory.
A familiarity with the basic notions of category theory is an advantage but not necessity.

The propositions which rely on Axiom of Choice or Boolean Ultrafilter Theorem are marked by  \ACPStar{}.
The chapter numbers are omitted whenever we refer to the same chapter.
