\chapter{Parts of duality}

\section{$\kappa$--complete Boolean algebras}
For the rest of this section, let $\kappa$ be some fixed infinite cardinal.

\begin{definition}
    A lattice is \DEFSYM[complete]{}{$\kappa$--complete}, if any of its subset of cardinality less than $\kappa$ has a supremum and an infimum. We say a homomorphism is $\kappa$--complete if it preserves all supremas and infimas of subsets of cardinality less than $\kappa$.
\end{definition}

\num The following Lemma will be handy for computations.
\begin{lemma*}\label{p:idealsFrame} Let $L$ be a frame. Then:
    \begin{enumerate}
        \item For all $J \in \J L$: $J^* =\;\downset (\bigvee J)^*$.
        \item For all $a \in L$: $(\downset a)^* = \downset a^*$ in $\J L$.
        \item If $L$ is Boolean, then for all $J \in \J L$: $J^{**} =\;\downset \bigvee J$.
    \end{enumerate}
\end{lemma*}
\begin{proof}
    We will prove just the first case, the others follow directly from it. Let $J$ be any ideal on $L$. Observe that $a \wedge \bigvee J = 0 $ iff $\downset a \wedge J = 0_{\Bo L}$. Since
        \begin{align*}
            J^* & = \bigvee \set{ L | L \wedge J = 0_{\Bo L} } = \bigcup \Set{ \downset a | \downset a \wedge J = 0_{\Bo L} } \text{ and}\\
            (\bigvee J)^* & = \bigvee \set{ a | a \wedge \bigvee J = 0 },
        \end{align*}

    \noindent we see $\downset a \subseteq J^*$ iff $a \in J^*$ iff $a \leq (\bigvee J)^*$, hence $J^* = \downset (\bigvee J)^*$.
\end{proof}


\begin{definition}
    Let $L$ be a Stone frame and let $s$ be an element of $L$. We say $s$ is a \DEFSYM[generated]{}{$\kappa$--generated} if $s = \bigvee S$ for some $S \subseteq \BcO L$ of cardinality less than $\kappa$.

    Let $L$ be a Stone frame, we say $L$ is \DEFSYM[basically disconnected]{}{$\kappa$--basically disconnected} if $m^{**} \vee m^* = 1$ for all $\kappa$--generated elements $m \in L$.
\end{definition}

\begin{lemma}\label{p:kappaCompleteBA}
    If $B$ is a $\kappa$--complete Boolean algebra, then $\JO B$ is a $\kappa$--basically disconnected Stone frame.
\end{lemma}
\begin{proof}
    By the Lemma~\ref{p:complIdeal} we know, complemented ideals are precisely principal ideals. Given any $M \subseteq B$, such that $|M| \leq \kappa$ and $I = \bigvee \Set{ \downset a | a \in M }$, we will show that $I^{**} \vee I^* = 1_{\JO B}$.

    Define $m = \bigvee M$ and $J = \downset m$. Observe that $I^{**} = J$: Trivially from Lemma~\ref{p:idealsFrame} $I^{**} \subseteq J^{**} = J$ and $J \subseteq I^{**}$ follows from
    $$ I^* = \bigcup \Set{ \downset a | \downset a \wedge I = 0_{\JO B}} = \bigcup \Set{ \downset a | a \wedge m = 0 } = (\downset m)^* = J^*.$$

    Consequently, $I^{**} \vee I^* = J \vee J^* = \downset m \vee \downset m^c = 1_{\JO B}$.
\end{proof}

\num The following Lemma will be very useful and we will use it without further referencing.
\begin{lemma*}
    Let $B$ be a Boolean algebra such that every of its subset of cardinality at most $\kappa$ has a supremum. Then $B$ is $\kappa$--complete.

    Consequently: Any Boolean homomorphism preserving all $\kappa$--meets (or $\kappa$--joins) is $\kappa$--complete.
\end{lemma*}
\begin{proof}
    Let $S$ be an arbitrary subset of $B$ such that $|S| \leq \kappa$. Set $M = (\bigvee \Set{ b^c | b \in S})^c$, we will show that $M$ is the infimum of $S$.

    Take any $a \in S$. We have $a \wedge M = a^{cc}\wedge (\bigvee \Set{ b^c | b \in S})^c = (a^c\vee \bigvee \Set{ b^c | b \in S})^c = M$. Hence $a \geq M$.

    Now suppose $m \leq a$ for all $a \in S$. Then $m\wedge M = (m^c\vee \bigvee \Set{b^c | b \in S})^c = (m^c)^c = m$. Hence $m \leq M$.
\end{proof}

\begin{lemma}\label{p:kappaComplStoneFrm}
    If $L$ is a $\kappa$--basically disconnected Stone frame, then $\BcO L$ is a $\kappa$--complete Boolean algebra. Moreover, the joins in $\BcO L$ are defined by the following formula

    $$\bigsqcup M = (\bigvee M)^{**}.$$
\end{lemma}
\begin{proof}
    For $M \subseteq \BcO L$ and $|M| \leq \kappa$, set $m = \bigvee M$. Since $L$ is $\kappa$--basically disconnected we have $m^{**} \vee m^* = 1$ and therefore $m^{**} \in \BcO L$. So $m^{**}$ is an upper bound for $M$ in $\BcO L$.

    Now let $n$ be an arbitrary upper bound for $M$ in $\BcO L$, thus $n$ is an upper bound in $L$ also, but $m \leq n$ since $m$ is the supremum of $M$ in $L$. Which gives us the desired relation $m^{**} \leq n^{**} = n$, hence $m^{**}$ is the supremum of $M$ in $\BcO L$.
\end{proof}

\begin{observation}
    Let $f\colon A \to B$ be a $\kappa$--complete Boolean homomorphism and let $I$ be a $\kappa$--generated ideal of $A$, then
    $$(\JO f)(I^*) = (\JO f)(I)^*.$$
\end{observation}
\begin{proof}
    The result is obtained by simple computation,
    \begin{align*}
        (\JO f)(I^*) &= (\JO f)(\downset (\bigvee I)^*)
                &= \downset f[\downset (\bigvee I)^*] = \downset f((\bigvee I)^*) \\
                &= \downset f(\bigvee I)^* & \text{($f$ is a Boolean homomorphism)} \\
                &= \downset (\bigvee f[I])^* & \text{($\kappa$--completeness of $f$)} \\
                &= \downset (\bigvee \downset f[I])^* = \downset (\bigvee (\JO f)(I))^* \\
                &= (\JO f)(I)^*.
    \end{align*}
    Where the $\kappa$--completeness in the fourth step can be used because $I$ is $\kappa$--generated. Or in other words, $I = \bigvee \Set{ \downset s | s \in S }$ for some $S$ subset of $A$ of the cardinality less then $\kappa$.
\end{proof}
% TODO more elegant

% As we will see after a while, the above Observation characterises morphisms in the image of $\kappa$--complete part of Boolean algebras in Stone correspondence.

\begin{definition}
    Let $f\colon L \to M$ be a homomorphism between Stone frames. We say $f$ is a \DEFSYM[basically complete]{}{$\kappa$--basically complete} if $f(a^*) = f(a)^*$ holds for all $\kappa$--generated elements $a \in L$.
\end{definition}

In other words, the last observation states that \JO{} sends any $\kappa$--complete Boolean homomorphism to a $\kappa$--basically complete frame homomorphism. And as we will see in the following Lemma, morphisms in the image of $\kappa$--complete part of Boolean algebras in Stone correspondence are characterised precisely this way.
% TODO or: precisely characterises morphisms?

\begin{lemma}
    Let $f\colon L \to M$ be a $\kappa$--basically complete frame homomorphism, then $\BcO f\colon \BcO L \to \BcO M$ is a $\kappa$--complete Boolean homomorphism.
\end{lemma}
\begin{proof}
    Let $A$ be an arbitrary subset of $\BcO L$ such that $|A| \leq \kappa$. We have
    \begin{align*}
        (\BcO f)(\bigsqcup A) &= f((\bigvee A)^{**}) \\
            &= f((\bigvee A)^*)^* & \text{($(\bigvee A)^*$ is complemented)}\\
            &= f(\bigvee A)^{**}  & \text{($\bigvee A$ is $\kappa$--generated)}\\
            &= (\bigvee f[A])^{**} & \text{($f$ is a frame homomorphism)}\\
            &= \bigsqcup f[A] = \bigsqcup (\BcO f)[A].
    \end{align*}
    Therefore $\BcO f$ is $\kappa$--complete.
\end{proof}

\num\label{p:kappaCompleteThm} From the Lemmas~\ref{p:kappaCompleteBA} and~\ref{p:kappaComplStoneFrm} we see, the restriction of classical Stone correspondence to subcategories of $\kappa$--complete Boolean algebras on one side and $\kappa$--basically disconnected Stone frames on the other (without any restriction on morphisms) is still a duality of categories.

However, we can set \DEF{\kComplBool{}} to be the category of $\kappa$--complete Boolean algebras and $\kappa$--complete Boolean homomorphisms and set \DEF{\kBDStoneFrm} as the category of $\kappa$--basically disconnected Stone frames and $\kappa$--basically complete frame homomorphisms.

Then, it is correct to define two functors (by the previous two Lemmas)

\begin{align*}
    \JK&\colon \kComplBool \to \kBDStoneFrm, \text{ and} \\
    \BcK&\colon \kBDStoneFrm \to \kComplBool,
\end{align*}

\noindent as a restriction of \JO{} and \BcO{} to the corresponding categories. And we can strengthen the last reasoning and get the following

\begin{theorem*}\label{p:kappaDuality}
    The functors \JK{} and \BcK{} constitute an isomorphism between categories \kComplBool{} and \kBDStoneFrm.
\end{theorem*}
\begin{proof}
    The only thing we need to show is that the morphisms of natural equivalences for identity functors and functors $\Bc\J$ and $\J\Bc$ are morphisms of our categories.

    For the first part, we will show $i_B\colon B \to \Bc\J(B)$ is a $\kappa$--complete Boolean homomorphisms for any $\kappa$--complete Boolean algebra $B$. Let $(b_j)_{j < \kappa} \subseteq B$, then by simple computation we get

    $$
        \bigsqcup_{j < \kappa} i_B(b_j) = \bigsqcup_{j < \kappa} \downset b_j = (\bigvee_{j < \kappa} \downset b_j)^{**} = \downset (\bigvee_{j < \kappa} b_j) = i_B(\bigvee_{j < \kappa} b_j),
    $$
    % TODO instead of j < \kappa, use some indexing set of cardinality less than \kappa

    \noindent where the third equality follows from~\ref{p:idealsFrame}.

    For the second part, we need to show $v_L\colon \J\Bc(L) \to L$ is $\kappa$--basically complete for any $\kappa$--basically disconnected Stone frame $L$. We will show $v_L$ is $\lambda$--basically complete for any infinite cardinal $\lambda$. Take any $I \in \J\Bc(L)$, we have

    \begin{align*}
        v_L(I^*) &= v_L(\bigvee \Set{ \downset s | \downset s \wedge I = 0 })
                  = \bigvee \Set{ v_L(\downset s) | \downset s \wedge I = 0 } \\
                 &= \bigvee \Set{ s | s \wedge v_L(I) = 0 } = v_L(I)^*,
    \end{align*}

    \noindent which finishes the proof.
\end{proof}

Equivalently in topological spaces, space is $\kappa$--basically disconnectedness iff any union of less than $\kappa$ clopen sets has open closure. In classical setting there is a duality between $\kappa$--complete Boolean algebras and $\kappa$--basically disconnected Stone spaces~\cite{monk1989handbook}.

% However ... (TODO discuss corresponding morphisms in Top)

\section{Complete Boolean algebras}

Since, there is no limitation or upper bound for the cardinal $\kappa$ in the Theorem~\ref{p:kappaDuality}, we can look at the part of the correspondence, where $\kappa$ is unbounded.

Very interesting is how the Stone frame part of the correspondence look like. For objects, from zero--dimensionality we know, each element is $\lambda$--generated for some infinite cardinal $\lambda$, hence

$$a^{**}\vee a^* = 1$$

\noindent holds for any element $a$; and the following condition for morphisms is fulfilled
\begin{align}
    f(a^*) = f(a)^*,\label{e:no0}
\end{align}

\noindent for any morphism $f$ and any element $a$. Thus the resulting category is the category of extremally disconnected Stone frames and morphisms satisfying (\oldref{e:no0}), we will denote it \DEF{\ExtrStoneFrm}. On the side of Boolean algebras we simply have the category of complete Boolean algebras and complete Boolean homomorphisms which we will denote \DEF{\ComplBool}.

We can define functors
\begin{align*}
    \JI&\colon \ComplBool \to \ExtrStoneFrm, \text{ and} \\
    \BcI&\colon \ExtrStoneFrm \to \ComplBool,
\end{align*}

\noindent as a restriction of \JO{} and \BcO{} to the corresponding categories. And the whole situation is depicted in the next diagram of categories


\begin{diagram}[row sep=0.7cm]
    \StoneFrm
        \ar[yshift=0.2em]{r}{\scalebox{1.5}\BcO}
    & \Bool
        \ar[yshift=-0.2em]{l}{\scalebox{1.5}\JO} \\
    \\
    \sigma\text{--}\categoryStyle{BDStoneFrm}
        \ar[yshift=0.2em]{r}{\scalebox{1.5}{$\Bc_\sigma$}}
        \ar{uu}{\InclUp}
    & \sigma\text{--}\ComplBool
        \ar{uu}{\InclUp}
        \ar[yshift=-0.2em]{l}{\scalebox{1.5}{$\J_\sigma$}} \\
    \DotsUp
        \ar{u}{\InclUp}
    & \DotsUp
        \ar{u}{\InclUp} \\
    \kBDStoneFrm
        \ar[yshift=0.2em]{r}{\scalebox{1.5}\BcK}
        \ar{u}{\InclUp}
    & \kComplBool
        \ar{u}{\InclUp}
        \ar[yshift=-0.2em]{l}{\scalebox{1.5}\JK} \\
    \DotsUp
        \ar{u}{\InclUp}
    & \DotsUp
        \ar{u}{\InclUp} \\
    \ExtrStoneFrm
        \ar[yshift=0.2em]{r}{\scalebox{1.5}{\BcI}}
        \ar{u}{\InclUp}
    & \ComplBool
        \ar{u}{\InclUp}
        \ar[yshift=-0.2em]{l}{\scalebox{1.5}\JI} \\
\end{diagram}

\noindent Where $\kappa$ iterates over all infinite cardinals greater or equal to $\omega_1$, $\sigma$--\ComplBool{} denotes the category $\omega_1$--\ComplBool{} and $\sigma$--\categoryStyle{BDStoneFrm} denotes the category $\omega_1$--\categoryStyle{BDStoneFrm}.

\begin{proposition}
    Let $B$ be a complete Boolean algebra. Then the frame $\J B$ is an extremally disconnected Stone frame.
\end{proposition}
\begin{proof}
    In any Boolean algebra the relations $\rbelow$ and $\leq$ coincide. Moreover, each complete Boolean algebra is a (completely regular extremally disconnected) Boolean frame. Thus $\J{B}$ equals $\C{B}$.

    From Lemma \ref{p:extrDiscPreserv} we know compactification preserves extremally disconnectedness and therefore \J{B} is also extremally disconnected. From Lemma \ref{p:JisFunctor} we know it is also a Stone frame.
\end{proof}

This Proposition is no surprise ("in the light/point of view of the previous paragraphs"), by the Theorem~\ref{p:kappaCompleteThm} we know, there is an isomorphism between the category of $\kappa$--completele Boolean algebras and the category $\kappa$--basically disconnected Stone frames. When we let $\kappa$ to be arbitrary large we obtain exactly an isomorphism of category of completele Boolean algebras and category of extremally disconnected Stone frames.

More important is what we got from its proof. As $\J B$ is the proper compactification of Boolean frame, it shows that this part of duality is of geometrical/topological nature.

\dotfill

As we see, the "last part" of the correspondence is purely geometrical. On both sides we have categories of frames (and with all frame homomorphisms in the \Bool{} part).

Moreover $\sigma$--frames are lattices having joins for all subsets of cardinality at most $\omega_0$ such that finite meets distribute over such joins.
They are also studied as some kind of form of geometrical abstraction.

Generally for $\kappa$--frames ... is in some sence also purely geometrical.
