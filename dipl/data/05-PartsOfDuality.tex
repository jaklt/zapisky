\chapter{Parts of duality}

\section{$\kappa$--complete Boolean algebras}
For the rest of this section, let $\kappa$ be some fixed infinite cardinal.

\begin{definition}
    A lattice is \DEF{$\kappa$--complete}, if any of its subset of cardinality at most $\kappa$ has a supremum and an infimum. We say a homomorphism is $\kappa$--complete if it preserves all supremas and infimas of subsets of cardinality at most $\kappa$.
\end{definition}

Note, our definition is a little nonstandard, what we mean by $\kappa$--completeness is usually called $\kappa^+$--completeness.

\num The following Lemma will be handy for computations.
\begin{lemma*}\label{p:idealsFrame} Let $L$ be a frame. Then:
    \begin{enumerate}
        \item For all $J \in \J L$: $J^* =\;\downset (\bigvee J)^*$.
        \item For all $a \in L$: $(\downset a)^* = \downset a^*$ in $\J L$.
        \item If $L$ is Boolean, then for all $J \in \J L$: $J^{**} =\;\downset \bigvee J$.
    \end{enumerate}
\end{lemma*}
\begin{proof}
    We will prove just the first case, the others follow directly from it. Let $J$ be any ideal on $L$. Observe that $a \wedge \bigvee J = 0 $ iff $\downset a \wedge J = 0_{\Bo L}$. Since
        \begin{align*}
            J^* & = \bigvee \Set{ L | L \wedge J = 0_{\Bo L} } = \bigcup \Set{ \downset a | \downset a \wedge J = 0_{\Bo L} } \text{ and}\\
            (\bigvee J)^* & = \bigvee \Set{ a | a \wedge \bigvee J = 0 },
        \end{align*}

    \noindent we see that $\downset a \subseteq J^*$ iff $a \in J^*$ iff $a \leq (\bigvee J)^*$, hence $J^* = \downset (\bigvee J)^*$.
\end{proof}


\begin{definition}
    Let $L$ be a Stone frame and let $s$ be a element of $L$. We say $s$ is a \DEF{$\kappa$--generated} if $s = \bigvee S$ for some $S \subseteq \Bc L$ of cardinality at most $\kappa$.

    Let $L$ be a Stone frame, we say $L$ is \DEF{$\kappa$--basically disconnected} if $m^{**} \vee m^* = 1$ for all $\kappa$--generated elements $m \in L$.
\end{definition}

\begin{lemma}\label{p:kappaCompleteBA}
    If $B$ is a $\kappa$--complete Boolean algebra, then $\J B$ is $\kappa$--basically disconnected Stone frame.
\end{lemma}
\begin{proof}
    By the Lemma~\ref{p:complIdeal} we know that complemented ideals are precisely principal ideals. Given any $M \subseteq B$, such that $|M| \leq \kappa$ and $I = \bigvee \Set{ \downset a | a \in M }$, we will show that $I^{**} \vee I^* = 1_{\J B}$.

    Define $m = \bigvee M$ and $J = \downset m$. Observe that $I^{**} = J$: Trivially from Lemma~\ref{p:idealsFrame} $I^{**} \subseteq J^{**} = J$ and $J \subseteq I^{**}$ follows from
    $$ I^* = \bigcup \Set{ \downset a | \downset a \wedge I = 0_{\J B}} = \bigcup \Set{ \downset a | a \wedge m = 0 } = (\downset m)^* = J^*.$$

    Consequently, $I^{**} \vee I^* = J \vee J^* = \downset m \vee \downset m^c = 1_{\J B}$.
\end{proof}

\num The following Lemma will be very useful and we will use it without further referencing.
\begin{lemma*}
    Let $B$ be a Boolean algebra such that every of its subset of cardinality at most $\kappa$ has a supremum. Then $B$ is $\kappa$--complete.

    Consequently: Any Boolean homomorphism preserving all $\kappa$--meets (or $\kappa$--joins) is $\kappa$--complete.
\end{lemma*}
\begin{proof}
    Let $S$ be an arbitrary subset of $B$ such that $|S| \leq \kappa$. Set $M = (\bigvee \Set{ b^c | b \in S})^c$, we will show that $M$ is the infimum of $S$.

    Take any $a \in S$. We have $a \wedge M = a^{cc}\wedge (\bigvee \Set{ b^c | b \in S})^c = (a^c\vee \bigvee \Set{ b^c | b \in S})^c = M$. Hence $a \geq M$.

    Now suppose $m \leq a$ for all $a \in S$. Then $m\wedge M = (m^c\vee \bigvee \Set{b^c | b \in S})^c = (m^c)^c = m$. Hence $m \leq M$.
\end{proof}

\begin{lemma}\label{p:kappaComplStoneFrm}
    If $L$ is a $\kappa$--basically disconnected Stone frame, then $\Bc L$ is a $\kappa$--complete Boolean algebra. Moreover, the joins in $\Bc L$ are defined by the following formula

    $$\bigsqcup M = (\bigvee M)^{**}.$$
\end{lemma}
\begin{proof}
    For $M \subseteq \Bc L$ and $|M| \leq \kappa$, set $m = \bigvee M$. Since $L$ is $\kappa$--basically disconnected we have $m^{**} \vee m^* = 1$ and therefore $m^{**} \in \Bc L$. So $m^{**}$ is an upper bound for $M$ in $\Bc L$.

    Now let $n$ be an arbitrary upper bound for $M$ in $\Bc L$, thus $n$ is an upper bound in $L$ also, but $m \leq n$ since $m$ is the supremum of $M$ in $L$. Which gives us the desired relation $m^{**} \leq n^{**} = n$, hence $m^{**}$ is the supremum of $M$ in $\Bc L$.
\end{proof}

\begin{observation}
    Let $f\colon A \to B$ be a $\kappa$--complete Boolean homomorphism and let $S$ be a subset of $A$ such that $|S| \leq \kappa$. Set $I$ to be the ideal generated by $S$, then
    $$(\J f)(I^*) = (\J f)(I)^*.$$
\end{observation}
\begin{proof}
    The result is obtained by simple computation
    \begin{align*}
        (\J f)(I^*) &= (\J f)(\downset (\bigvee I)^*) = \downset f((\bigvee I)^*) \\
                &= \downset f(\bigvee I)^* & \text{($f$ is a Boolean homomorphism)} \\
                &= \downset (\bigvee f[I])^* & \text{($\kappa$--completeness of $f$)} \\
                &= \downset (\bigvee \downset f[I])^* = \downset (\bigvee (\J f)(I))^* \\
                &= (\J f)(I)^*
    \end{align*}
    Where the $\kappa$--completeness in the third step can be used because $I = \bigvee \Set{ \downset s | s \in S }$ and $|S| \leq \kappa$.
\end{proof}
% TODO more elegant

% As we will see after a while, the above Observation characterises morphisms in the image of $\kappa$--complete part of Boolean algebras in Stone correspondence.

\begin{definition}
    Let $f\colon L \to M$ be a homomorphism between Stone frames. We say $f$ is a \DEF{$\kappa$--basically complete} if $f(a^*) = f(a)^*$ holds for all $\kappa$--generated elements $a \in L$.
\end{definition}

In other words the last observation states that \J{} sends any $\kappa$--complete Boolean homomorphism to a $\kappa$--basically complete frame homomorphism. As we will see from the following Lemma, morphisms in the image of $\kappa$--complete part of Boolean algebras in Stone correspondence are characterised precisely this way.
% TODO or: precisely characterises morphisms?

\begin{lemma}
    Let $f\colon L \to M$ be a $\kappa$--basically complete frame homomorphism, then $\Bc f\colon \Bc L \to \Bc M$ is a $\kappa$--complete Boolean homomorphism.
\end{lemma}
\begin{proof}
    Let $A$ be an arbitrary subset of $\Bc L$ such that $|A| \leq \kappa$. We have
    \begin{align*}
        (\Bc f)(\bigsqcup A) &= f((\bigvee A)^{**}) \\
            &= f((\bigvee A)^*)^* & \text{($(\bigvee A)^*$ is complemented)}\\
            &= f(\bigvee A)^{**}  & \text{($\bigvee A$ is $\kappa$--generated)}\\
            &= (\bigvee f[A])^{**} & \text{($f$ is a frame homomorphism)}\\
            &= \bigsqcup f[A] = \bigsqcup (\Bc f)[A].
    \end{align*}
    Therefore $\Bc f$ is $\kappa$--complete.
\end{proof}

\num From the Lemmas~\ref{p:kappaCompleteBA} and~\ref{p:kappaComplStoneFrm} we see that restriction of classical Stone correspondence to subcategories of $\kappa$--complete Boolean algebras on one side and $\kappa$--basically disconnected Stone frames on the other (without any restriction on morphisms) is still a duality of categories. However, we can strengthen the result and get the following

\begin{theorem*}
    The category of $\kappa$--basically disconnected Stone frames and $\kappa$--basically complete frame homomorphisms is isomorphic to the category of $\kappa$--complete Boolean algebras and $\kappa$--complete Boolean homomorphisms.
\end{theorem*}
\begin{proof}
    The only thing we need to show is that the morphisms of natural equivalences for identity functors and functors $\Bc\J$ and $\J\Bc$ are morphisms of our categories.

    For the first part, we will show $i_B\colon B \to \Bc\J(B)$ is a $\kappa$--complete Boolean homomorphisms for any $\kappa$--complete Boolean algebra $B$. Let $(b_j)_{j < \kappa} \subseteq B$, then by simple computation we get

    $$
        \bigsqcup i_B(b_j) = \bigsqcup \downset b_j = (\bigvee \downset b_j)^{**} = \downset (\bigvee b_j) = i_B(\bigvee b_j),
    $$

    \noindent where the third equality follows from~\ref{p:idealsFrame}.

    For the second part, we need to show $v_L\colon \J\Bc(L) \to L$ is $\kappa$--basically complete for any $\kappa$--basically disconnected Stone frame $L$. We will show $v_L$ is $\lambda$--basically complete for any cardinal $\lambda$. Take any $I \in \J\Bc(L)$, we have

    \begin{align*}
        v_L(I^*) &= v_L(\bigvee \Set{ \downset s | \downset s \wedge I = 0 })
                  = \bigvee \Set{ v_L(\downset s) | \downset s \wedge I = 0 } \\
                 &= \bigvee \Set{ s | \downset s \wedge v_L(I) = 0 } = v_L(I)^*,
    \end{align*}

    \noindent which finishes the proof.
\end{proof}

Equivalently in topological spaces, space is $\kappa$--basically disconnectedness iff any union of less than $\kappa$ clopen sets has open closure. In classical setting there is a duality between $\kappa$--complete Boolean algebras and $\kappa$--basically disconnected Stone spaces~\cite{monk1989handbook}.

% However ... (TODO discuss corresponding morphisms in Top)


\section{Complete Boolean algebras}
TODO say something about geometrical/topological meaning of complete Boolean algebras (meaning that ComplBool are frames...).

\begin{proposition}
    Let $B$ be a complete Boolean algebra. Then the frame $\J B$ is an extremally disconnected Stone frame.
\end{proposition}
\begin{proof}
    In any Boolean algebra the relations $\rbelow$ and $\leq$ coincide. Moreover, each complete Boolean algebra is a (completely regular extremally disconnected) Boolean frame. Thus $\J{B}$ equals $\R{B}$.

    From Lemma \ref{p:extrDiscPreserv} we know compactification preserves extremally disconnectedness and therefore \J{B} is also extremally disconnected. From Lemma \ref{p:JisFunctor} we know it is also a Stone frame.
\end{proof}

\num As \DEF{\ComplBool{}} denote the category of all complete Boolean algebras and complete Boolean homomorphisms preserving all meets and joins.

\begin{definition}
    Let $H$ be a Heyting algebra, by \DEF{Booleanization} of $H$ we mean the set \DEFSYM{Booleanization}{$\Bo H$} $= \Set{ a^{**} | a \in H }$.
\end{definition}

\begin{proposition}
    Let $H$ be a Heyting meet--semilattice, then $\Bo H$ is a Boolean algebra with joins and meets defined
    $$ a \sqcup b = (a^* \wedge b^*)^* \quad\text{and}\quad a\sqcap b = a \wedge b.$$
\end{proposition}
\begin{proof}
    First, we will show that $\sqcap$ really is the meet and $\sqcup$ is the join. For $a,b,c \in \Bo H$:

    \begin{itemize}
        \item $a\sqcup b = a^{**}\wedge b^{**} = (a\wedge b)^{**} \in \Bo H$.
        \item Whenever $a,b \leq c$, then $a^*, b^*\geq c^*$ and so $a^*\wedge b^*\geq c^*$, hence $a\sqcap b = (a^*\wedge b^*)^* \leq c^{**} = c$. Trivially $a\sqcap b \in \Bo H$.
    \end{itemize}

    For $a \in \Bo H$, $a\sqcup b = a^{**}\sqcup b = (a^{**}\wedge a^*)^* = 0^* = 1$ and also $a\sqcap a^* = a\wedge a^* = 0$, hence each element is complemented.
\end{proof}

\num\label{p:propertiesBooleanization}
    Let $L$ be a frame, the mapping \DEFSYM{BetaL}{$\beta_L\colon$}$ L \to \Bo L$ defined $a \mapsto a^{**}$ is a nucleus:
    \begin{enumerate}[label=(N\arabic*)]
        \item From the definition of pseudocomplement we have $a \wedge a^* = 0$ iff $a \leq a^{**}$ hence $a \leq a^{**}$;
        \item Pseudocomplement is antitone hence $(a \leq b \implies a^{**} \leq b^{**})$;
        \item $a^* = a^{***}$ implies $a^{**\;**} = a^{**}$; and
        \item $(a \wedge b)^{**} = a^{**} \wedge b^{**}$ is standard equality (Lemma~\ref{??}).
    \end{enumerate}

    Therefore by~\label{p:nuclProp}, $\Bo L$ is a sublocale of $L$ and consequently a complete Boolean algebra. From~\label{p:nuclProp} we also have $\beta_L$ is a frame homomorphism.

\begin{block}{Note}
    $\Bo L$ is the smallest dense sublocale of $L$; and joins are given also by the following formula $(a \vee b)^{**}$, since $a \mapsto a^{**}$ is a nucleus.
\end{block}

\begin{lemma}\label{p:basicalMorphs}
    Let $f\colon L \to M$ be a frame homomorphism and $a \in L$ such that $a^{**}\vee a^* = 1$, then
    $$ f(a^{**})^* = f(a^*).$$

    In particular for $a = a^{**}$ we have
    $$ f(a)^* = f(a^*).$$
\end{lemma}
\begin{proof}
    From the assumptions we see the following holds
    \begin{align*}
        f(a^{**} \vee a^*) & = f(a^{**}) \vee f(a^*) = 1, \\
        f(a^{**} \wedge a^*) & = f(a^{**}) \wedge f(a^*) = 0.
    \end{align*}
    Hence $f(a^{**})$ is complement to $f(a^*)$ and distributivity of $M$ gives us $f(a^*) = f(a^{**})^*$.
\end{proof}


\begin{definition}
    For a frame homomorphism $f\colon H \to K$, set $\Bo f\colon \Bo H \to \Bo K$ to be the mapping
    $$\Bo f\colon a \mapsto f(a)^{**}.$$
\end{definition}

\num\label{p:boolenizationMorph} The following Proposition is taken from~\cite{banaschewski1996booleanization}.
\begin{proposition*}
    Let $f\colon L \to M$ be a frame homomorphism, then $\Bo f$ is a frame homomorphism such that the following diagram commutes
    \begin{diagram}
        L \ar{r}{\beta_L} \ar{d}{f} & \Bo L \ar{d}{\Bo f}\\
        M \ar{r}{\beta_M}           & \Bo M
    \end{diagram}
    if and only if $f(a^{**}) \leq f(a)^{**}$ for all $a \in L$.
\end{proposition*}
\begin{proof}
    We will not prove the ``only if'' part, because it will not be needed in the rest of the text.

    First observe that
    $$ f(a^{**})^{**} = f(a)^{**} \iff f(a^{**}) \leq f(a)^{**}. $$
    $\Leftarrow$ is straightforward and the $\Rightarrow$ implication follows simply from $f(a^{**})^{**} \leq f(a)^{**}$. Hence the diagram commutes.

    Then by simple computation we get
    $$ (\Bo f)(\bigsqcup A) = f((\bigvee A)^{**})^{**} = f(\bigvee A)^{**} = (\bigvee f[A])^{**}. $$

    Now, take any $a\in A$, from $f(a) \leq \bigvee f[A]$ we have $f(a)^{**} \leq (\bigvee f[A])^{**}$. Since $a$ was chosen arbitrary, we also have $\bigvee_{a\in A} f(a)^{**} \leq (\bigvee f[A])^{**}$ and therefore $(\bigvee_{a\in A} f(a)^{**})^{**} \leq (\bigvee f[A])^{**}$. Hence $\bigsqcup (\Bo f)[A] = (\bigvee f[A])^{**}$.

    To sum up everything, we have
    $$ \bigsqcup (\Bo f)[A] = (\bigvee f[A])^{**} = (\Bo f)(\bigsqcup A),$$
    \noindent which is what we wanted.
\end{proof}

\num As we see from~\ref{p:basicalMorphs} and~\ref{p:boolenizationMorph} and from the fact that
    \begin{align}
        f(a^{**})^* = f(a^*)\text{ and }f(a^{**}) \leq f(a)^{**} \text{ iff } f(a^*) = f(a)^*\label{e:1001}
    \end{align}
    (Indeed, The $f(a^*) \leq f(a)^*$ is always true and $f(a^*) = f(a^{**})^* \geq f(a)^{***} = f(a)^*$)
, in order for \Bo{} to behave functorially, we need to restrict category of extremally disconnected Stone frames only to morphisms satisfying~(\ref{e:1001}) -- basically complete frame homomorphisms. We will denote this category \ExtrStoneFrm{}.

\begin{proposition}
    $\Bo\colon \ExtrStoneFrm \to \ComplBool$ is a functor.
\end{proposition}
\begin{proof}
    From~\ref{p:propertiesBooleanization} we know $\Bo L$ is a complete Boolean algebra.

    As a direct implication of Proposition~\ref{p:boolenizationMorph} we get that for any basically complete frame homomorphism between two extremally disconnected Stone frames $f\colon L \to M$ the following diagram commutes.

    \begin{diagram}
        L \ar{r}{\beta_L} \ar{d}{f} & \Bo L \ar{d}{\Bo f}\\
        M \ar{r}{\beta_M}           & \Bo M
    \end{diagram}

    \noindent And therefore for any frame homomorphisms $f, g$ the following diagram also commutes.

    \begin{diagram}
        L \ar{r}{\beta_L}
          \ar{d}{f}
          \ar[bend right]{dd}[swap]{gf} &
        \Bo L \ar{d}[swap]{\Bo f}
              \ar[bend left]{dd}{\Bo (gf)}\\

        M \ar{r}{\beta_M} \ar{d}{g} & \Bo M \ar{d}[swap]{\Bo g}\\
        N \ar{r}{\beta_N}           & \Bo N
    \end{diagram}

    Finally for any identity frame homomorphism $i_L$, $\Bo i_L$ is the identity on $\Bo L$. Consequently, $\Bo$ is a functor.
\end{proof}

\num One can check the proof and see that we do not have to assume anything more about category of frames than to have morphisms satysfying $f(a^{**}) \leq f(a)^{**}$ for all $a$, in order to have \Bo{} functorial.

\begin{proposition}
    Let $B$ be a Boolean frame, then $B \cong \Bo\J(B)$.
\end{proposition}
\begin{proof}
    From the previous Lemma we see that $\J \in \Bo\J(B)$ iff $J = J^{**} = \downset \bigvee J$. On the other hand, for $a \in B$: $(\downset a)^{**} = \downset a^{**} = \downset a$.

    (FIXME:) Denote $\tilde i\colon B \to \Bo\J(B)$ the mapping $a \mapsto \downset a$. It is a Boolean homomorphism, because $\downset a \vee \downset b = \downset (a \vee b)$, $\downset a \wedge \downset b = \downset (a \wedge b)$, $\downset 0 = \{0\} = 0_{\Bo\J(B)}$ and $\downset 1 = B = 1_{\Bo\J(B)}$. Consequently $\tilde i$ is an isomorphism of $B$ and $\Bo\J(B)$.
    % TODO $\tilde i$ need to preserve big joins, we get that by \bigsqcup I_i = (\bigcup I_i)^{**}
\end{proof}

\begin{proposition}
    Let $L$ be an extremally disconnected Stone frame, then $L \cong \J\Bo(L)$.
\end{proposition}
\begin{proof}
    Similarly to the general case, define $\tilde v_L\colon \J\Bo(L) \to L$ and $\tilde\iota\colon L \to \J\Bo(L)$ as
    $$  \tilde v_L\colon I \mapsto \bigvee I \quad\text{and}\quad \tilde\iota\colon a \mapsto \downset a \cap \Bo L.$$

    Trivially we have $\tilde\iota \tilde v_L \supseteq \id_{\J\Bo(L)}$ and $\tilde v_L \tilde\iota \leq \id_L$. We again have the situation where $\tilde v_L$ is the left Galois adjoint to $\tilde\iota$ and $\bigvee I_1 \wedge \bigvee I_2 = \bigvee (I_1 \wedge I_2)$ for any two $I_1, I_2 \in \J\Bo(L)$ and so $\tilde v_L$ is a frame homomorphism.

    Actually we also have the opposite inequalities: $\tilde v_L \tilde\iota \geq \id_L$ because $a \rbelow x$ implies $a^{**} \rbelow x$ and $L$ is regular. For $\tilde\iota \tilde v_L \subseteq \id_{\J\Bo(L)}$, take any $x \in \tilde\iota \tilde v_L(I)$. We have $x \leq \bigvee I$, $x = x^{**}$ and from extremally disconnectedness $x^{**} \vee x^* = 1$, so
    $$ 1 = x \vee x^* \leq \bigvee I \vee x^*$$
    \noindent holds and from the compactness of $L$ there is a finite $F \subseteq I$ such that $\bigvee F \vee x^* = 1$. Now $x \leq \bigvee F$ since $x = 1 \wedge x = (x^* \vee \bigvee F) \wedge x = \bigvee F \wedge x$ and therefore $x \in I$.
\end{proof}

\num The following diagram commutes
    \begin{diagram}
        A \ar{r}{\tilde i_A} \ar{d}[swap]{f} & \Bc\J(A) \ar{d}{\Bc\J(f)}\\
        B \ar{r}{\tilde i_B}                 & \Bc\J(B)
    \end{diagram}

    $$\Bo\J(f)(\downset a) = (\downset f[\downset a])^{**} = \downset f(a)^{**} = \downset f(a)$$

    TODO $\tilde i_A$ is complete Boolean homomorphism.

\num The following also commutes
    \begin{diagram}
        \J\Bc(L) \ar{d}[swap]{\J\Bc(f)} \ar{r}{\tilde v_L} & L \ar{d}{f} \\
        \J\Bc(M) \ar{r}{\tilde v_M}    & M
    \end{diagram}
    $$(\J\Bo)(f)(I) = \downset \Set{ f(x)^{**} | x \in I } = \downset \Set{ f(x) | x \in I} = \downset f[I]$$

    \noindent Therefore $\tilde v_L((\J\Bo)(f)(I)) = \bigvee \downset f[I] = \bigvee f[I] = f(\bigvee I)$.

    TODO $\tilde v_L$ is basically complete frame homomorphism.

\num TODO By the previous isomorphism lemmas we have the natural equivalence between identity functor and \Bo\J{} resp. \J\Bo{} and get an isomorphism of categories \ComplBool{} and \ExtrStoneFrm, two subcategories of \Frm.

\num As conclusion the functors \Bo{} and \J{} form an adjunction, with unit and counit being isomorphism and so we have the following
\begin{theorem*}
    The categories \ExtrStoneFrm{} and \ComplBool{} are isomorphic.
\end{theorem*}

\num TODO note that \Bo{} is part of Stone duality only between \ExtrStoneFrm{} and \ComplBool{}, since for every frame $L$ the frame $\Bo L$ is a complete Boolean algebra.

\num TODO Since each Boolean frame is completely regular, \J{} corresponds precisely to compactification of this frame.

\section*{Comments (on $\sigma$--Frames)}
TODO Mention $\sigma$-Frames.
TODO Compare compactification in classical and in point-free setting. Mention that \J{} correspond to compactification, the topology of Ult is isomorphic to ideal lattice of Boolean algebra and that this was known~\cite{monk1989handbook}.

\begin{center}
\begin{tikzpicture}
    \draw (1,3.2) node {$\StoneFrm$};
    \draw (7,3.2) node {$\Bool$};

    \draw[semithick] (1,0) ellipse (1.2 and 2.5);
    \draw[semithick] (7,0) ellipse (1.2 and 2.5);

    % Complete part
    \draw[semithick] (1.85,-1.7) arc (30:150:1.0cm);
    \draw[semithick] (7.85,-1.7) arc (30:150:1.0cm);

    % sigma part
    \draw[semithick] (2.12,0.8) arc (20:160:1.2cm);
    \draw[semithick] (8.12,0.8) arc (20:160:1.2cm);

    % kappa-complete part
    \draw[semithick] (2.20,-0.3) arc (25:155:1.32cm);
    \draw[semithick] (8.20,-0.3) arc (25:155:1.32cm);

    % Functors
    \draw[->,semithick] (2.5,2.0) -- (5.5,2.0) node[above,midway] {$\Bc$};
    \draw[->,semithick] (2.5,-0.5) -- (5.5,-0.5) node[above,midway] {$\Bc$};
    \draw[->,semithick] (2.5,-1.9) -- (5.5,-1.9) node[above,midway] {$\Bo$};
    \draw[<-,semithick] (2.5,-2.1) -- (5.5,-2.1) node[below,midway] {$\J$};
    \draw[<-,semithick] (2.5,0.8) -- (5.5,0.8) node[above,midway] {$\J$};

    % Categories captions
    \node at (-1, 1.2) {$\sigma$--\categoryStyle{BDStoneFrm}};
    \node at (1, 0.9) {$\vdots$};
    \node at (-1,-0.2) {$\kappa$--\categoryStyle{BDStoneFrm}};
    \node at (1,-0.6) {$\vdots$};
    \node at (-0.5,-2.0) {\ExtrStoneFrm};

    \node at (8.2, 1.2) {$\sigma$--\ComplBool};
    \node at (7, 0.9) {$\vdots$};
    \node at (8.2,-0.2) {$\kappa$--\ComplBool};
    \node at (7,-0.6) {$\vdots$};
    \node at (8,-2.0) {\ComplBool};
\end{tikzpicture}
\end{center}

\noindent Where $\kappa$--\ComplBool{} denotes the category of $\kappa$--complete Boolean algebras and $\kappa$--complete Boolean homomorphisms and $\kappa$--\categoryStyle{BDStoneFrm} denotes the category of $\kappa$--basically disconnected Stone frames and $\kappa$--basically complete frame homomorphisms.
