\chapter{Construction of \R}

\section{Summarization}

In one part of the diagram, we have compactification -- by taking frames of regular ideals of the previous frames.

On the side of Stone correspondence, we take a Boolean algebra and take a frame of its ideals. However, this is the same as taking the frame of regular ideals.

There is nothing new in this observation, in the classical Stone duality for topological spaces, in direction from Boolean algebras we take a space of all ultrafilters of a Boolean algebra. One of the many ways, how compactification for topological spaces is constructed, is again as the space of all ultrafilters on the previous space.


As we see, the construction is completely the same on objects of corresponding categories. However, the Stone correspondence is a duality of categories. The natural question is , when we switched from an isomorphism of categories to coreflection.


As we see from the next picture, we cannot extend the duality to frames more than we actually have.

\begin{center}
\begin{tikzpicture}[descr/.style={fill=white,inner sep=2.5pt}]
    \tikzset{
      mymx/.style={
        matrix of nodes,
        % nodes=block,
        row sep=3.5em,
        column sep=3.5em,
      },
      lbl/.style={
        % above,
        below,
        auto,
        % pos=0.15,
        % sloped, % make the text follow the path
        %execute at begin node={$}, % begin math mode, for the minus signs etc.
        %execute at end node={$}, % end math mode
      }
    }
    \tikzstyle{bigbox} = [draw=gray, thick, rounded corners, rectangle]

    \pgfmathsetmacro{\H}{1.0}
    \pgfmathsetmacro{\W}{0.6}

    \matrix (m) [mymx]
        { KRFrm & & CRFrm \\
          StoneFrm & Bool & \\
          KDMRFrm & CBool & DMFrm \\
        };
    \path[->,font=\scriptsize,every node/.style=lbl]
        (m-2-1)     edge node {\InclUp} (m-1-1)
        (m-3-1)     edge node {\InclUp} (m-2-1)
        (m-3-3)     edge node {\InclUp} (m-1-3)
        (m-3-2)     edge node {\InclUp} (m-2-2)
        (m-3-2)     edge node {$ \subseteq^{\text{obj}}$} (m-3-3)

        % Stone duality
        (m-2-2) edge[transform canvas={yshift=1.5pt}]  node[swap] {Ideals}  (m-2-1)
        (m-2-1) edge[transform canvas={yshift=-1.5pt}] node[swap] {Boolean} (m-2-2)

        % Compact. Reflection
        (m-1-1.355) edge node[swap] {Forget} (m-1-3.185)
        (m-1-3.175) edge node[swap] {Comp} (m-1-1.5);

    \path[<->, thick]
        (m-3-1) edge node {} (m-3-2);


    \node[bigbox, dotted] [fit = (m-2-1) (m-3-2)] {};
    \draw[gray, rounded corners, dashed]
           ($(m-1-1) + (-\H,   0)$)
        -- ($(m-1-1) + (-\H, +\W)$)
        -- ($(m-1-3) + (+\H, +\W)$)
        -- ($(m-3-3) + (+\H, -\W)$)
        -- ($(m-3-3) + (-\H, -\W)$)
        -- ($(m-1-3) + (-\H, -\W)$)
        -- ($(m-1-1) + (-\H, -\W)$)
        -- ($(m-1-1) + (-\H,   0)$);
\end{tikzpicture}
\end{center}

New categories in the picture: CBool denotes the category of complete Boolean algebras and \emph{all} Boolean homomorphisms. The category KDMRFrm is the category of all extremally disconnected Stone spaces, or in other words all compact De Morgan regular frames, and \emph{all} frame homomorphisms. KRFrm denotes the category of compact regular frames, CRFrm the category of completely regular frames.

The dotted rectangle of the picture is pictured the Stone duality. In the dashed area is depicted the coreflection given by compactification of completely regular frames (and their subcategories).

% TODO Compare compactification in classical and in point-free setting. Mention that \J{} correspond to compactification, the topology of Ult is isomorphic to ideal lattice of Boolean algebra and that this was known~\cite{monk1989handbook}.
% (note: Taken from the big picture of Stone duality)

\R{} is a general construction of compact frames from pseudocomplemented bounded lattices and frame homomorphisms from homomorphisms between those lattices.


\begin{diagram}
    KDMRFrm \ar[<->, thick]{r}{} & CBool \ar{r}{\subseteq^{\text{obj}}} & DMRFrm \ar[bend right=20,swap]{ll}{\restr{Comp}{DMRFrm}} \\
    ExtrDStoneFrm \ar[yshift=0.2em]{r}{\BcI}\ar{u}{\InclUp} & ComplBool \ar[yshift=-0.2em]{l}{\JI} \ar{u}{\InclUp} \ar{ru}{\subseteq}
\end{diagram}

As long as we have in a category all (regular) De Morgan frames as objects, we cannot expect from \R{} to form an isomorphism, because we know, the category of compact (regular) De Morgan frames is (non--isomorphically) coreflexive in the category of (regular) De Morgan frames~\ref{p:extrDiscPreserv} (certainly, there exists a non--compact (regular) De Morgan frame).


\section{Booleanization}

\begin{definition}\label{d:Booleanization}
    Let $H$ be a Heyting algebra, by \DEF{Booleanization} of $H$ we mean the set
    $$
    \DEFSYM{Booleanization}{\Bo H} = \Set{ a^{**} | a \in H }.
    $$
\end{definition}

\begin{proposition}
    Let $H$ be a Heyting meet--semilattice, then $\Bo H$ is a Boolean algebra with joins and meets defined
    $$ a \sqcup b = (a^* \wedge b^*)^* \quad\text{and}\quad a\sqcap b = a \wedge b.$$
\end{proposition}
\begin{proof}
    First, we will show $\sqcap$ really is the meet, $\sqcup$ is the join and both operations are well--defined. For $a,b,c \in \Bo H$:

    \begin{itemize}
        \item From~\ref{p:pseudcomplProperties}, $a\sqcap b = a^{**}\wedge b^{**} = (a\wedge b)^{**} \in \Bo H$.
        \item Whenever $a,b \leq c$, then $a^*, b^*\geq c^*$ and so $a^*\wedge b^*\geq c^*$, hence $a\sqcup b = (a^*\wedge b^*)^* \leq c^{**} = c$. And trivially, $a\sqcup b \in \Bo H$.
    \end{itemize}

    For $a \in \Bo H$: $a^* = a^{***} \in \Bo H$ too, $a\sqcup a^* = (a^*\wedge a^{**})^* = 0^* = 1$ and also $a\sqcap a^* = a\wedge a^* = 0$. Hence, each element is complemented and $\Bo H$ is a Boolean algebra.
\end{proof}

\num\label{p:propertiesBooleanization}
    Let $L$ be a frame, the mapping defined as $a \mapsto a^{**}$ is a nucleus:
    \begin{enumerate}[label=(N\arabic*)]
        \item $a \leq a^{**}$: From the definition of pseudocomplement we have $a \wedge a^* = 0$ iff $a \leq a^{**}$ and the first is always true (again, from the definition);
        \item $(a \leq b \implies a^{**} \leq b^{**})$: holds, since taking pseudocomplements is antitone;
        \item $a^{**\;**} = a^{**}$: simply from $a^* = a^{***}$; and
        \item $(a \wedge b)^{**} = a^{**} \wedge b^{**}$: it is a standard equality (Lemma~\ref{p:pseudcomplProperties}).
    \end{enumerate}

    Therefore, by~\ref{p:nuclProp}, $\Bo L$ is a sublocale of $L$ and consequently a complete Boolean algebra. As a drirect consequence of~\ref{p:nuclProp}, the mapping
    $$\text{\DEFSYM{BetaL}{$\beta_L$}}\colon L \to \Bo L,\quad a \mapsto a^{**},$$

    is a frame homomorphism.

\begin{block}{Note}
    $\Bo L$ is the smallest dense sublocale of $L$; and joins are also given by the following formula $(a \vee b)^{**}$ (since $a \mapsto a^{**}$ is a nucleus).
\end{block}


\begin{definition}\label{d:BooleanizationMorph}
    For a frame homomorphism $f\colon H \to K$, set \DEFSYM{BooleanizationMorph}{$\Bo f$}$\colon \Bo H \to \Bo K$ to be the mapping
    $$\Bo f\colon a \mapsto f(a)^{**}.$$
\end{definition}

\num\label{p:boolenizationCond} The following Proposition is taken from~\cite{banaschewski1996booleanization}.
\begin{proposition*}
    Let $f\colon L \to M$ be a frame homomorphism, then $\Bo f$ is a frame homomorphism such that the following diagram commutes
    \begin{diagram}
        L \ar{r}{\beta_L} \ar{d}{f} & \Bo L \ar{d}{\Bo f}\\
        M \ar{r}{\beta_M}           & \Bo M
    \end{diagram}
    if and only if $f(a^{**}) \leq f(a)^{**}$ for all $a \in L$.
\end{proposition*}
\begin{proof}
    First observe that
    \begin{align}
        f(a^{**})^{**} = f(a)^{**} \iff f(a^{**}) \leq f(a)^{**},\quad\text{for all } a\in L.\label{e:2202iff20leq02}\tag{W.O.}
    \end{align}

    $\Leftarrow$ is straightforward and the $\Rightarrow$ implication follows simply from $f(a^{**}) \leq f(a^{**})^{**} $. And the diagram commutes iff $f(a^{**})^{**} = f(a)^{**}$.

    The last thing we need to show, is that $f(a^{**}) \leq f(a)^{**}$ implies $\Bo f$ is a frame homomorphism. It is straightforward to see $\Bo f$ preserves $0$, $1$ and $\sqcap$. For $\sqcup$--preserving, take $a, b \in \Bo L$ and infer
    $$(\Bo f)(a)\sqcup (\Bo f)(b) = (f(a)^{**}\vee f(b)^{**})^{**} \geq f(a^{**}\vee b^{**})^{**} = f(a\vee b)^{**} = (\Bo f)(a\sqcup b),$$

    \noindent where the last equality holds thanks to~(\oldref{e:2202iff20leq02}). The opposite inequality is trivial, thus $\Bo f$ is a Boolean homomorphism.

    Let's show $\Bo f$ preserves big joins. Again using~(\oldref{e:2202iff20leq02}), we get
    $$ (\Bo f)(\bigsqcup A) = f((\bigvee A)^{**})^{**} = f(\bigvee A)^{**} = (\bigvee f[A])^{**}. $$

    Now, take any $a\in A$, from $f(a) \leq \bigvee f[A]$ we have $f(a)^{**} \leq (\bigvee f[A])^{**}$. Since $a$ was chosen arbitrary, we also have $\bigvee_{a\in A} f(a)^{**} \leq (\bigvee f[A])^{**}$ and therefore $(\bigvee_{a\in A} f(a)^{**})^{**} \leq (\bigvee f[A])^{**}$. The opposite inequality is trivial, hence $\bigsqcup (\Bo f)[A] = (\bigvee f[A])^{**}$.

    To sum up everything, we have
    $$ \bigsqcup (\Bo f)[A] = (\bigvee f[A])^{**} = (\Bo f)(\bigsqcup A),$$
    \noindent which is what we wanted.
\end{proof}

\begin{proposition}
    Let $\p C$ be a subcategory of the category of frames and frame homomorphisms satisfying (\oldref{e:2202iff20leq02}). Then

    $$\Bo\colon \p C \to \ComplBool,$$

    \noindent defined on objects as in~\ref{d:Booleanization} and on morphisms as in~\ref{d:BooleanizationMorph}, is a functor.
    % TODO do not reference inside a proof -- ref{e:2202iff20leq02}
\end{proposition}
\begin{proof}
    From~\ref{p:propertiesBooleanization} we know $\Bo L$ is a complete Boolean algebra. As a direct implication of Proposition~\ref{p:boolenizationCond} we get that for any morphism $f\colon L \to M$ in $\p C$, the following diagram commutes.

    \begin{diagram}
        L \ar{r}{\beta_L} \ar{d}{f} & \Bo L \ar{d}{\Bo f}\\
        M \ar{r}{\beta_M}           & \Bo M
    \end{diagram}

    \noindent $\Bo f$ is a frame homomorphisms, but it is also a complete Boolean homomorphisms, as we know from~\ref{p:kappaCompleteBAfromMeets}. Since the following diagram also commutes, for any morphisms $f, g$, we can see $\Bo$ respect morphisms composition.

    \begin{diagram}
        L \ar{r}{\beta_L}
          \ar{d}{f}
          \ar[bend right]{dd}[swap]{gf} &
        \Bo L \ar{d}[swap]{\Bo f}
              \ar[bend left]{dd}{\Bo (gf)}\\

        M \ar{r}{\beta_M} \ar{d}{g} & \Bo M \ar{d}[swap]{\Bo g}\\
        N \ar{r}{\beta_N}           & \Bo N
    \end{diagram}

    Finally, for any identity frame homomorphism $i_L$, $\Bo i_L$ is the identity on $\Bo L$. Consequently, $\Bo$ is a functor.
\end{proof}

\begin{lemma}\label{p:basicalMorphs}
    Let $f\colon L \to M$ be a frame homomorphism and $a \in L$ such that $a^{**}\vee a^* = 1$, then
    $$ f(a^{**})^* = f(a^*).$$

    In particular, for $a = a^{**}$ we have
    $$ f(a)^* = f(a^*).$$
\end{lemma}
\begin{proof}
    From the assumptions we see the following holds
    \begin{align*}
        f(a^{**}) \vee f(a^*) &= f(a^{**} \vee a^*) = 1, \\
        f(a^{**}) \wedge f(a^*) &= f(a^{**} \wedge a^*)  =  0.
    \end{align*}

    Hence $f(a^{**})$ is complemented and $f(a^*)$ is its complement. From distributivity of $M$ we know, complements are unique, hence $f(a^*) = f(a^{**})^*$.
\end{proof}

\num For any frame homomorphism between two extremally disconnected Stone frames, by Lemma~\ref{p:basicalMorphs}, $f(a^{**})^* = f(a^*)$ for all $a$. As we see from~\ref{p:boolenizationCond} and from the fact that
    \begin{align*}
        f(a^{**})^* = f(a^*)\text{ and }f(a^{**}) \leq f(a)^{**} \text{ iff } f(a^*) = f(a)^*\label{e:1001}\tag{N.O.}
    \end{align*}

    hold for all $a$ (indeed, the $f(a^*) \leq f(a)^*$ is always true and $f(a^*) = f(a^{**})^* \geq f(a)^{***} = f(a)^*$)
, in order for \Bo{} to behave functorially, we need to restrict category of extremally disconnected Stone frames only to morphisms satisfying~(\oldref{e:1001}).

However, this is exactly the case for the category \ExtrStoneFrm{}. Basically complete frame homomorphisms suffices the condition (\oldref{e:1001}), and \ExtrStoneFrm{} is the only subcategory of \StoneFrm{} we presented here which satisfies it.

\begin{conclusion}
    $\Bo\colon \ExtrStoneFrm \to \ComplBool$ is a functor.
\end{conclusion}

In literature frame homomorphisms satisfying (\oldref{e:2202iff20leq02}) are called \DEF{weakly open} and frame homomorphisms satisfying (\oldref{e:1001}) are called \DEF{nearly open}~\cite{banaschewski1994variants}.
% TODO topological interpretation

\begin{observation}\label{p:boEQbc}
    $\Bo L = \Bc L$ for any extremally disconnected Stone frame $L$.
\end{observation}
\begin{proof}
    $\Bo L \supseteq \Bc L$ holds always. Let $x \in \Bo L$, then $x = x^{**}$ and from extremal disconnectedness also $x^{**}\vee x^*=1$, hence $x \in \Bc L$.
\end{proof}


Until the end of this section, we will use the Lemma~\ref{p:idealsFrame} frequently without further referencing.

\begin{lemma}
    Let $B$ be a Boolean frame, then $B \cong \Bo\J(B)$ in \ComplBool.
\end{lemma}
\begin{proof}
    We see $J \in \Bo\J(B)$ iff $J = J^{**} = \downset \bigvee J$. On the other hand, for $a \in B$: $(\downset a)^{**} = \downset a^{**} = \downset a$.

    Denote \DEFSYM{istar2}{$\tilde i_B$}$\colon B \to \Bo\J(B)$ to be the mapping $a \mapsto \downset a$. From the previous Observation we know the definitions of $\tilde i_B$ and $i_B$ coincide, therefore by~\ref{p:BoolEquivalence} $\tilde i_B$ is a Boolean homomorphism.

    Now, take any $\Set{ a_i | i\in I}$ subset of $B$, then
    $$
    \bigsqcup_{i\in I} \downset a_i = (\bigvee_{i\in I} \downset a_i)^{**} = \downset (\bigvee_{i\in I} a_i)^{**} = \downset (\bigvee_{i\in I} a_i).
    $$

    \noindent Consequently, $\tilde i_B$ is a complete Boolean homomorphism, morphisms of \ComplBool{} and an isomorphism of $B$ and $\Bo\J(B)$.
\end{proof}

\num Now, for any complete Boolean algebras $A$ and $B$, for any complete Boolean homomorphism $f\colon A \to B$ and for $\tilde i_B$ from the previous Lemma, the following diagram commutes
    \begin{diagram}
        A \ar{r}{\tilde i_A} \ar{d}[swap]{f} & \Bo\J(A) \ar{d}{\Bo\J(f)}\\
        B \ar{r}{\tilde i_B}                 & \Bo\J(B)
    \end{diagram}

    Indeed, we have
    $$
    \Bo\J(f)(\downset a) = (\downset f[\downset a])^{**} = \downset f(a)^{**} = \downset f(a).
    $$

    We can conclude the following
\begin{proposition*}
    The collection $\tilde i_*$ of complete Boolean homomorphisms forms a natural equivalence between $\Bo\JI$ and the identity functor on \ComplBool.
\end{proposition*}

\begin{lemma}
    Let $L$ be an extremally disconnected Stone frame, then $L \cong \J\Bo(L)$ in \ExtrStoneFrm.
\end{lemma}
\begin{proof}
    Similarly to the general case, define \DEFSYM{vstar2}{$\tilde v_L$}$\colon \J\Bo(L) \to L$ as
    $$ \tilde v_L\colon I \mapsto \bigvee I.$$

    From the Observation~\ref{p:boEQbc}, we know the definitions of $\tilde v_L$ and $v_L$ are the same, and $\tilde v_L$ is a frame isomorphism by~\ref{p:StoneFrmEquivalence}.

    From the proof of Theorem~\ref{p:kappaCompleteThm} we know $v_L$ is $\lambda$--basically complete frame homomorphism for all regular cardinals $\lambda$, therefore $\tilde v_L$ is also basically complete and is a~morphisms of $\ExtrStoneFrm$.
\end{proof}

\num We can show naturalness of $\tilde v_*$ defined in the previous Lemma. Let $L$ and $M$ be extremally disconnected Stone frames and let $f\colon L \to M$ be a basically complete frame homomorphism. The following diagram commutes
    \begin{diagram}
        \J\Bo(L) \ar{d}[swap]{\J\Bo(f)} \ar{r}{\tilde v_L} & L \ar{d}{f} \\
        \J\Bo(M) \ar{r}{\tilde v_M}    & M
    \end{diagram}

    To show that, let $I \in \J\Bo(L)$. First observe, for $x\in I$, $x = x^{**}$ and since $L$ is extremally disconnected, $x\vee x^* = 1$. Therefore, $f(x^*) = f(x)^*$ by Lemma~\ref{p:basicalMorphs}. Now, we can compute
    $$
    (\J\Bo)(f)(I) = \downset \Set{ f(x)^{**} | x \in I } = \downset \Set{ f(x) | x \in I} = \downset f[I].
    $$

    And from that, we see
    $$
    \tilde v_M((\J\Bo)(f)(I)) = \bigvee \downset f[I] = \bigvee f[I] = f(\bigvee I) = f(\tilde v_L(I)).
    $$

    And finally, from the commutativity of the diagram above we can conclude

\begin{proposition*}
    The collection $\tilde v_*$ of basically complete frame homomorphisms is a natural equivalence between $\JI\Bo$ and the identity functor on \ExtrStoneFrm.
\end{proposition*}

\num By the Theorem~\ref{p:kappaCompleteThm} we know we have an isomorphism between categories of $\kappa$--complete Boolean algebras and $\kappa$--basically disconnected Stone frames witnessed by functors \J{} and \Bc{}. When we restrict $\J{}$ to the category \ComplBool{}, morphisms on side of Stone frames are precisely basically complete frame homomorphisms.

% TODO refs
By 2.13, 2.14, 2.15 and 2.16 we have the natural equivalence between identity functors and \Bo\J{} resp. \J\Bo{}.
% By the previous isomorphism lemmas we have the natural equivalence between identity functor and \Bo\J{} resp. \J\Bo{} and get an isomorphism of categories \ComplBool{} and \ExtrStoneFrm (two subcategories of \Frm).
As conclusion the functors \Bo{} and \J{} form an adjunction, with units and counits being natural equivalence and we obtain the main result of this section

\begin{theorem*}
    The categories \ExtrStoneFrm{} and \ComplBool{} are isomorphic.
\end{theorem*}


\num One may ask, if \Bo{} works also in other parts of the correspondence. As $\Bo L$ is always a complete Boolean algebra and so $\J\Bo(L)$ is extremally disconnected, there is no hope for non--extremally disconnected Stone frames.
% TODO and by \ref{e:1001} we cannot hope to ommit condition to have weakly open frame homomorphismsa?
