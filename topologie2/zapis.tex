\documentclass[12pt,a4paper]{article}
\usepackage{../basic}
\usepackage{../math}

\usepackage{fancyhdr}
\usepackage{xspace}
\pagestyle{fancy}
\renewcommand{\headrulewidth}{0.0pt} % podtrzeni titulku
\newcommand{\head}[1]{\medskip\noindent {\bf #1}}

\newcommand\BX{\ensuremath{\beta X}}
\newcommand\Tich{$T_{3.5}$\xspace}
\newcommand\Hausd{$T_2$\xspace}
\newcommand\close{\overline}
\newcommand{\betaClose}[1]{\close #1 ^{\beta X}}

% \usepackage{verbatim}
% \renewcommand\dukaz{\begin{comment}}
% \renewcommand\qed{\end{comment}}

% Hand made theorem
% \newenvironment{proof}[1][Proof]{\begin{trivlist} \item[\hskip \labelsep {\bfseries #1}]}{\end{trivlist}}

% \newcommand{\qedd}{\nobreak \ifvmode \relax \else
% 	  \ifdim\lastskip<1.5em \hskip-\lastskip
% 	  \hskip1.5em plus0em minus0.5em \fi \nobreak
% 	  \vrule height0.75em width0.5em depth0.25em\fi}

\begin{document}
%% Zapisky zakladnich veci:
\section{Základy}
\definice Nechť $(X,\sigma)$, $(X,\tau)$ jsou dvě topologie na X. Řekneme, že topologie $\sigma$ je {\bf jemnější (silnější)} než topologie $\tau$ (popř. $\tau$ je {\bf hrubší (slabší)}) než topologie $\sigma$, pokud $\sigma \supseteq \tau$.

Pozn: Diskrétní topologie je nejjemnější, indiskrétní je nejhrubší.

\definice Buď $(X,\sigma)$ topologický prostor, $\{(X_\alpha,\sigma_\alpha) : \alpha \in A\}$ topologické prostory a $f_\alpha : X \to X_\alpha$ je zobrazení pro každé $\alpha \in A$. Řekneme, že topologie $\sigma$ je {\bf projektivně vytvořena} (resp. topologický prostor $(X,\sigma)$ je projektivně vytvořen) souborem $\{A_\alpha : \alpha \in A\}$, pokud platí:
\begin{enumerate}
\item $(\forall \alpha \in A)~f_\alpha : (X,\sigma) \to (X_\alpha,
	\sigma_\alpha)$ je spojité,
\item $\sigma$ je nejhrubší topologie na množině $X$ splňující (1), tj. pokud
	je $\sigma'$ topologie na $X$, $\sigma' \not\supseteq \sigma$, pak existuje
	$\alpha \in A$ takové, že $f_\alpha : (X,\sigma') \to (X_\alpha,
	\sigma_\alpha)$ není spojité.
\end{enumerate}

\tvrzeni Topologický prostor $(X,\sigma)$ je projektivně generován souborem
	zobrazení $\{f_\alpha : X \to (X_\alpha,\sigma)~;~\alpha \in A\} \iff \{
	f^{-1}[O] : \alpha \in A, O \in \sigma_\alpha\}$ tvoří subbási topologie
	$\sigma$.
\dukaz Buď $\p R$ topologie na $X$ o subbási $\{f^{-1}[O] : \alpha \in A, O \in \sigma_\alpha\}$. Z charakteristiky spojitosti je pro každé $\alpha \in A$ zobrazení $f_\alpha : (X,\sigma) \to (X_\alpha,\sigma_\alpha)$ spojité. Protože $\p R$ splňuje (1) a je nejhrubší topologie splňující (1), má platit $\p R \supset \sigma$. Nechť $\tau$ je libovolná topologie na $X$, při které všechna zobrazení $f_\alpha : (X,\tau) \to (X_\alpha,\sigma)$ jsou spojitá. Potom $\{f^{-1}[O] : \alpha \in A, O \in \sigma_\alpha\} \subseteq \tau$, tedy $\p R \subseteq \tau$. Volbou $\tau = \sigma$ dostaneme, že $\p R \subseteq \sigma$.
\qed

\veta Topologický prostor $(X, \sigma)$ je projektivně vytvořen souborem zobrazení $\{f_\alpha : (X,\sigma) \to (X_\alpha, \sigma_\alpha)~;~\alpha \in A\} \iff$ platí následující podmínka:

\begin{description}
\item[($\star$)] Zobrazení $g : (Y,\p R) \to (X, \sigma)$ kde $(Y, \p R)$ je libovolný topologický prostor, je spojité $\iff$ pro každé $\alpha \in A$ je zobrazení $f_\alpha \circ g: (Y,\p R) \to (X_\alpha, \sigma_\alpha)$ spojité.
\end{description}

\dukaz Předpokládejme, že topolgoie $\sigma$ je projektivně generována souborem
zobrazení $\{f_\alpha : (X,\sigma) \to (X_\alpha, \sigma_\alpha)~;~\alpha \in
A\}$:

($\Rightarrow$) Nechť $g : (Y, \p R) \to (X,\sigma)$ je spojité. Potom
$f_\alpha \circ g$ je složením dvou spojitých zobrazení a tedy je spojité.

($\Leftarrow$) Víme, že všechna zobrazení $f_\alpha \circ g$ jsou spojitá.
Protože soubor $\{f^{-1}[O] : \alpha \in A, O \in \sigma_\alpha\}$ je otevřená
subbáse topologie $\sigma$, neboť $(X, \sigma)$ je tímto souborem projektivně
generována. Stačí ověřit, že $g^{-1}[f^{-1}[O]]$ je otevřená pro každé $\alpha
\in A$ a $O \in \sigma_\alpha$, už postačí ke spojitosti $g$ (charakterizace
spojitosti). Platí, že $O\in\sigma_\alpha$ je otevřená v
$(X_\alpha,\sigma_\alpha)$ a zobrazení $f_\alpha \circ g$ je spojité, tedy
$g^{-1}[f^{-1}[O]]$ je otevřená v $(Y, \p R)$. Tedy $g$ je spojité.

Nechť platí ($\star$). Označme $\tau$ topologii $X$, při které je splněna
($\star$). Chceme ukázat, že $\tau = \sigma$. Zobrazení $id : (X, \tau) \to
(X,\tau)$ je spojité, čili podle ($\star$) $f_\alpha \circ id : (X,\tau) \to
(X_\alpha, \sigma_\alpha)$ je spojité zobrazení pro všechna $\alpha \in A$.
Tedy $f_\alpha : (X,\tau) \to (X_\alpha, \sigma_\alpha)$ je spojité pro všechna
$\alpha \in A$ a přitom $(X,\sigma)$ je projektivně generovaná soustavou
$\{f_\alpha : \alpha \in A\}$. Dostáváme $\tau \supseteq \sigma$. Uvažujeme $id
: (X, \sigma) \to (X,\sigma)$. Víme, že $f_\alpha \circ id : (X,\sigma) \to
(X_\alpha, \sigma_\alpha)$ je spojité zobrazení pro všechna $\alpha \in A$,
neboť $\sigma$ je projektivně generovaná. Podle ($\star$) je $id : (X,\sigma)
\to (X,\tau)$ je spojitá a proto $\tau \subseteq \sigma$. Dohromady $\sigma =
\tau$.
\qed

\definice Buď $(X,\sigma)$ topologický prostor, $\{(X_\alpha,\sigma_\alpha) : \alpha \in A\}$ soubor topologických prosotrů a $f_\alpha : (X_\alpha,\sigma_\alpha) \to X$, $\alpha \in A$, zobrazení. Řekneme, že topologie $\sigma$ na $X$ je {\bf induktivně generována} souborem zobrazení ${f_\alpha : \alpha \in A}$, jestliže:
\begin{enumerate}
\item $(\forall \in A)~f_\alpha : (X_\alpha,\sigma_\alpha) \to (X,\sigma)$ je spojité,
\item $\sigma$ je nejjemnější topologie na $X$ splňující (1), tj. je-li $\sigma'$ topologie na $X$ splňující (1), pak $\sigma \supseteq \sigma'$.
\end{enumerate}

\tvrzeni Buďte $f_\alpha : (X_\alpha,\sigma_\alpha) \to (X,\sigma)$, $\alpha
\in A$, zobrazení. Topologie $\sigma$ je induktivně generována souborem
zobrazení $\{f_\alpha : \alpha \in A\}  \iff (\forall \p U \subseteq X)~\p U
\in \sigma \leftrightarrow (\forall \alpha \in A)~f_\alpha^{-1}[\p
U]\in\sigma_\alpha$.

\veta Topologický prostor $(X,\sigma)$ je induktivně generován zobrazeními $f_\alpha : (X_\alpha,\sigma_\alpha) \to (X,\sigma)$, $\alpha \in A$ $\iff$ platí následující ekvivalence:

\begin{description}
\item{($\star$)} zobrazení $g : (X,\sigma) \to (Y, \p R)$ je spojité $\iff$ pro všechna $\alpha \in A$ je $g \circ f_\alpha : (X_\alpha, \sigma_\alpha) \to (Y,\p R)$ spojité.
\end{description}

%% 1. hodina
\section{Čechovsky úplné prostory}
\lemma
	$bX$ kompaktifikace úplně regulárního prostoru X, $f: \beta X \to
	bX$, $f$ je spojité rzšíření $id: X \to X$. Potom $f[\beta
	X\setminus X] = bX \setminus X$.
\dukaz
	$f[X]$ je hustý v $bX$, tedy $f[\beta X] \supseteq \close X^{bX} = bX$,
	$f$ je na.

	Sporem: nechť $p \in \beta X \setminus X$ takový, že $f(p) \in
	X$. Označme $x = f(p)$. \BX je hausdorfův, tedy existuje (v \BX) $U$, $V$,
	že $p \in U, x \in V, U \cap V = \emptyset$.

	$V \cap X$ je množina otevřená v $X$, existuje tedy $W \subseteq bX$,
	otevřené v $bX$, že $W \cap X = V \cap X$. $f$ je spojité, $f(p) = x$, $W$
	okolí x v $bX$, existuje $U_1$ otevřené okolí $p$ v $\beta X$, že $f[U_1]
	\subseteq W$.
	
	$U \cap V$ otevřené v \BX, $f[U \cap U_1] \subset W$, $X$ hustý v
	$\beta X$, existuje $z \in U \cap U_1 \cap X$, $f(z) \in W, f(z) \neq z,
	V \cap U_1 \cap U = \emptyset$, takže $f$ nerozšiřuje $id: X \to
	X$, spor.

	\qed

\veta $X$ je \Tich. Pak následující je ekvivalentní:
\begin{enumerate}
	\item Pro každou kompaktifikaci $bX$ prostoru $X$ je přírustek $bX
		\setminus X$ množina typu $F_\sigma$ v $bX$.
	\item Přírustek $\beta X \setminus X$ je množina typu $F_\sigma$ v \BX.
	\item Existuje kompaktifikace $cX$ prostoru $X$ taková, že $cX \setminus X$
		je $F_\sigma$ v $cX$.
\end{enumerate}

\dukaz
1 $\rightarrow$ 2 $\rightarrow$ 3: triviálně

2 $\rightarrow$ 1: Buď $bX$ libovolná kompaktifikace prostoru $X$, $id: X \to
	X$ se spojitě rozšiřuje na $f : \beta X \to bX$. Podle lemmatu $f[\BX
	\setminus X] = bX \setminus X$. Podle (2), $\BX \setminus X = \bigcup_{n
	\in \omega} F_n$, kde každá množina $F_n$ je uzavřená v $\BX \setminus X$.
	$f[F_n]$ je uzavřené v $bX$. $bX \setminus X = \bigcup_{n \in \omega}
	f[F_n]$ tedy $bX \setminus X$ je množina typu $F_\sigma$ v $bX$.

3 $\rightarrow$ 2: $cX$ buď kompaktifikace $X$, že $cX \setminus X = \bigcup_{n
	\in \omega} F_n$, že všechny $F_n$ jsou uzavřené v $cX$. $id : X \to X$ se
	spojitě rozšířuje na $f : \BX \to cX$. $f^{-1}[F_n]$ je (ze spojitosti)
	uzavřené v \BX. $f^{-1}[F_n] \subseteq \beta X \setminus X$ dle lemmatu,
	$\BX \setminus X = \bigcup_{n \in \omega} f^{-1}[F_n]$, tedy $\beta X
	\setminus X$ je typu $F_\sigma$ v \BX.
\qed

\definice \Tich~$X$ se nazývá {\bf Čechovsky úplný}, jestliže splňuje
	kteroukoliv z podmínek 1., 2., 3. předchozí věty.

\veta Tichonovův prostor $X$ je Čechovsky úplný $\iff$ existuje spočetný systém
	$\{\p G_n: n \in \omega\}$ otevřených pokrytí prostoru $X$ s následující
	vlastností:

	Kdykoliv $\p F$ soubor uzavřených množin, který má konečnou průnikovou
	vlastnost a takový, že $(\forall n \in \omega)~(\exists F \in \p F), F
	\subseteq G$, potom $\bigcap \p F \neq \emptyset$.
\dukaz
\noindent$\Leftarrow$: $X$ je \Tich,~$X \subseteq \beta X,~\{\p G_i : i \in
	\omega \}$ je soubor otevřených pokrytí s vlastnostmi z věty. Označme $\p
	G_i = \{G_{a,i} : a \in A_i\}$, každá množina $G_{a,i}$ je otevřená v $X$,
	máme $V_{a,i}$ otevřené množiny v \BX, aby platilo, že $G_{a,i} = X \cap
	V_{a,i}$. Tedy:
		$$\bigcap_{i \in \omega}~\bigcup_{a \in A_i} V_{a,i} \supseteq X$$
	
	Potřebujeme ukázat, že platí rovnost. (pak $\beta X \setminus X$ je
	$F_\sigma$ v \BX)

	Zvolme libovolně $x \in \bigcap_{i \in \omega}~\bigcup_{a \in A_i}
	V_{a,i}$, označme $\p B(x)$ bázi okolí bodu $x$ v prostoru \BX. Dále
	označme
		$$\p F = \{X \cap \betaClose B : B \in \p B(x)\}$$
	
	Platí, že $X \cap \betaClose B$ pro libovolné $B \in \p B(x)$ je
	uzavřené v $X$, neprázdné ($X$ je husté v \BX). $\p F$ má konečnou
	průnikovou vlastnost.

	Pro $i \in \omega$ libovolné: $x \in \bigcup_{a \in A_i} V_{a,i}$, tedy
	existuje $a_0 \in A_i$, $x \in V_{a_0,i}$, existuje tedy $B \in \p B(x), x
	\in B \subseteq \betaClose B \subseteq V_{a_0,i}$
		$$G_{a_0,i} = V_{a_0,i} \cap X,~\close B \cap X \in \p F,~
		  \text{tedy } \close B \cap X \subseteq G_{a_0,i}$$

	Tedy $\bigcap \p F \neq \emptyset$, ale $\bigcap \p F = \{x\}$, tedy $x
	\in X$ a tedy platí rovnost.


\medskip\noindent$\Rightarrow$: Nechť $X$ je Čechovsky úplný, $\beta X \supseteq X$,

	existují otevřené (v \BX) množiny $O_{x,i}$, tak že $X = \bigcap_{i \in
	\omega} G_i$.

	Pro $i \in \omega, x \in X, x \in G_i$, existuje otevřená množina (v \BX)
	$O_{x,i}$ tak, že:
		$$x \in O_{x,i} \subseteq \betaClose O_{x,i} \subseteq G_i$$
		$$\{X \cap O_{x,i} : x \in X\} = \p G_i,~\text{je otevřené pokrytí}$$

	Máme ukázat, že pro $\{\p G_i : i \in \omega \}$ platí tvrzení věty.

	Nechť $\p F$ je soubor uzavřených podmnožin prostoru $X$, splňující (1) má
	konečnou průnikovou vlastnost, (2) $i \in \omega (\exists G \in \p G_i)
	(\exists F \in \p F), F \subseteq G$
		$$\bigcap \{\betaClose F : F \in \p F\} \neq \emptyset \text{, z
		  kompaktnosti } \beta X$$

	Zvolme $x \in \bigcap \{\betaClose F : F \in \p F\}$, potřebujeme ukázat,
	že $x \in X$.

	Sporem: $x \in \beta X \setminus X$, tedy existuje $i \in \omega,~x \notin
	G_i,~\p G_i$ sestává z množin tvaru $O \cap X$, kde $O$ je otevřená v \BX a
	$\betaClose O \subseteq G_i$.

	$(\exists F \in \p F) (\exists O \cap X \in \p G_i)~F \subseteq O \cap X,~
	\betaClose F \subseteq O \subseteq G_i \not\ni x_i$, tedy $x \not\in
	\bigcap \{\betaClose F : F \in \p F\}$, spor.

	\qed

\veta[Baireova] Je-li $X$ Čechovsky úplný, pak každý soubor $\{U_n : n \in
	\omega\}$ otevřených hustých množin v prostoru $X$ má průnik hustý v X.
\dukaz
	Zvolme $\{\p G_i : i \in \omega\}$ jako v předchozí větě.

	Indukcí: $U_0$ je otevřené husté, zvolme $x \in U_0$ a okolí $O_0$ bodu
	$x_0$ tak, že $O_0 \subseteq U_0$ a součastně existuje $G \in \p G_0$, že
	$\close O_0 \subseteq G$.

	Zvolme $O_n$ otevřené, $U_{n+1}$ otevřené husté, existuje $x_{n+1} \in O_n
	\cap U_{n+1}$, bod $x_{n+1}$ má otevřené okolí $O_{n+1}$ tak, že $O_{n+1}
	\subseteq U_{n+1} \cap O_n$ a pro nějaké $G \in \p G_{n+1}$ je $\close
	O_{n+1} \subseteq G$. $\{\close O_n : n \in \omega\}$ $X$ je Čechovsky
	úplný, tedy $\bigcap \{\close O_n : n \in \omega\} \neq \emptyset$.

	Průnik je hustý, protože náš výsledek nezáleží na prvotní volbě $x$ a $U_0$
	(popřípadě pořadí dalších).
	\qed

\veta Platí:
\begin{itemize}
	\item Je-li $X$ Čechovsky úplný, $Y \subseteq X$ uzavřená množina, pak $Y$
		je Čechovsky úplný.
	\item Je-li $Y \subseteq X$ podmnožina typu $G_\delta$, pak podprostor $Y$
		je Čechovsky úplný.
\end{itemize}
\dukaz
	Průniková vlastnost $G_\delta$ v $G_\delta$ je $G_\delta$.
	\qed

\veta Suma $\Sigma_{a \in A}$ je Čechovsky úplná $\iff$ všechny $X_a$ jsou
	Čechovsky úplné.

\veta Kartézský součin spočetně mnoha Čechovsky úplných prostorů je Čechovsky úplný.
\dukaz $\{ X_n : n \in \omega \}$ soubor Čechovsky úplných prostorů, zvolme
	$bX_n$ kompaktifikaci $X_n$
		$$\prod_{n \in \omega} bX_n \supseteq \prod_{n \in \omega}
		  X_n,~\prod_{n \in \omega} bX_n \text{ je kompaktní prostor}$$
	Položme
		$$H_{n,m} = \prod_{j \neq n} bX_j \times F_{n,m},~H_{n,m} \text{ je
		  uzavřené v } \prod_{j \in \omega} bX_j$$
		$$H_{n,m} \cap \prod_{j \in \omega} bX_j \neq \emptyset~(F_{n,m}
		  \subseteq bX \setminus X_n). \bigcup_{n \in \omega} H_{n,m} =
		  \prod_{n \in \omega} bX_n \setminus \prod_{n \in \omega} X_n.$$
	Tedy $\prod X_n$ je Čechovsky úplný.
	\qed

%% 2. hodina
\section{Parakompaktní prostory}
\definice $X$ je topologický prostor, $\p M \subseteq \p P(X)$. $\p M$ se nazývá.
\begin{enumerate}[(a)]
	\item {\bf lokálně konečný}, jestliže ke každému bodu $x \in X$ existuje
	$O$ okolí bodu $x$ tak, že
		$$\{M \in \p M : M \cap O \neq \emptyset \}$$
	je konečný.

	\item {\bf diskrétní}, jestliže ke každému bodu $x \in X$ existuje $O$
	okolí bodu $x$ tak, že
		$$|\{ M \in \p M : M \cap O \neq \emptyset \}| \leq 1$$

	\item {\bf $\sigma$-lokálně konečný} (respektive {\bf $\sigma$-diskrétní}),
		pokud $\p M = \bigcup_{n \in \omega} \p M_n$ tak, že každý soubor $\p
		M_n$ je lokálně konečný (respektive diskrétní)
\end{enumerate}

\priklad $\{(a, a+1), a = p / 2, p \in \Z\}$ je lokálně konečný, $\{(a, a+1), a = p / 2^n, p \in \Z\}$ je $\sigma$-lokálně konečný

\poznamka
\begin{enumerate}[(a)]
	\item $\p M$ diskrétní $\implies$ $\p M$ lokálně konečný
	\item $\p M$ lokálně konečný: $\close{\bigcup \{M : M \in \p M\}} = \bigcup \{\close M : M \in \p M\}$ Tedy $\p M$ má vlastnost zachování uzávěru.
\end{enumerate}

\veta[A. H. Stone]
	Buď $X$ metrizovatelný prostor. Ke každému otevřenému pokrytí $\p U$
	prostoru $X$ existuje otevřené pokrytí $\p V$, které je lokálně konečné a
	$\sigma$-diskrétní, a které zjemňuje $\p U$.
\dukaz \qed

\head{Důsledky:}
\begin{enumerate}
	\item Nechť topologický prostor $X$ je metrizovatelný. Potom $X$ má
	$\sigma$-lokálně konečnou bázi (za $\p U=\{B_\rho(x, \frac{1}{n}) : x \in
	X\}$, $n \in \omega \setminus \{0\}$)

	\item Nechť $X$ je metrizovatelný. Potom $X$ má $\sigma$-diskrétní
		otevřenou bázi.
\end{enumerate}

\definice Topologický prostor $X$ se nazývá {\bf parakompaktní}, jestliže je
	\Hausd~a jestliže ke každému otevřenému pokrytí $\p U$ existuje lokálně
	konečné otevřené pokrytí $\p V$, které zjemňuje $\p U$.

\priklad Každý metrizovatelný prostor je kompaktní (Stoneova věta). Každý
	kompaktní prostor je parakompaktní.

(Příkladem nějakého nemetrizovatelného parakompaktu je například velký součin
kompaktů)

\definice Topologický prostor $X$ se nazývá {\bf kolektivně normální}, je-li
	$T_1$ a jestliže ke každému diskrétnímu souboru $\p F$ sestávajícího z
	uzavřených množin existuje soubor $\{\p U(F) : F \in \p F\}$, že pro každé
	$F \in \p F$, $F \subseteq \p U(F)$, každá množina $\p U(F)$ je otevřená,
	pro každou dvojici $F, F' \in \p F, F \neq F' \implies \p U(F) \cap \p
	U(F') = \emptyset$.

\veta Každý parakompaktní prostor je kolektivně normální.
\dukaz[nekontrolováno] Postupně:
\begin{enumerate}[(a)]
\item Každý parakopaktní prostor je regulární.

	Mějme $x \in X,~F \subseteq X$. $F$ uzavřené, $x \notin F$.

	$X$ je Hausdorfův, pro každé $y \in F$ zvolme dvě disjunktní otevřené
	množiny, $U_y \ni x$, $V_y \ni y$.
		$$\p U = \{V_y : y \in F\} \cup \{X \setminus F\}.$$
	$\p U$ je otevřené pokrytí $X$, $X$ je parakompaktní, buď $\p V$ otevřené
	lokálně konečné zjemnění $\p U$. Systém $\p V$ je lokálně konečný, tedy
	existuje $W$ okolí $x$ tak, že $\p V_0 := \{V \in \p V : V \cap W \neq
	\emptyset\}$ je konečné.

	Je-li $V \in \p V$: Pokud $V \cap F \neq \emptyset$, pak $V \subseteq
	V_y$ pro nějaké $y \in F$. Označme toto $y$ jako $y(V)$.

	Množina $W \cap \bigcap \{U_{y(V)} : V \in \p V_0\}$ je otevřená množina
	($\p V_0$ je konečné), obsahuje bod $x$ a současně neprotíná žádné $V \in
	\p V$, takové, že $V \cap F \neq \emptyset$.
	
	Položíme-li tedy $G = \{V \in \p V : V \cap F \neq \emptyset \}$ máme
	disjunktní otevřené množiny obsahující $x$, $F$.

\item Parakompaktní prostor je kolektivně normální.
	
	Zvolme libovolně diskrétní soubor $\p F$ sestávající z uzavřených množin.
	Pro každé $F \in \p F$ a každé $x \in F$ zvolme otevřené okolí $O_x$ bodu
	$x$ tak, že:
		$$\close O_x \cap \bigcup \{F' : F' \in \p F, F' \neq F\} = \emptyset.$$
	Položme $\p U = \{O_x : x \in \bigcup \p F\} \cup \{X \setminus \bigcup \p F$. $\p U$ je otevřené pokrytí prostoru $X$, $X$ je parakompaktní, nechť $\p V$ je lokálně konečné pokrytí $X$, které zjemňuje $\p U$.

	Máme-li $F \in \p F$, $F \neq F'$, $x \in F'$, $x \in V \in \p V$. Protože
	$V \subseteq O_x: \close V \cap F = \emptyset$. $\p V$ je lokálně konečný:
		$$\emptyset = F \cap \close{\bigcup \{V \in \p V: \exists F' \neq F, V
		\cap F' \neq \emptyset\}}$$

	Tedy stačí položit:
		$$U(F) = \bigcup \{V \in \p V: V \cap F \neq \emptyset\} \setminus
		\close{\bigcup \{V \in \p V : V \cap \bigcup (\p F \setminus \{F\}) \neq
		\emptyset \}}$$

	Zřejmě $U(F)$ je otevřené, $U(F) \supseteq F$, pro $F \neq F': U(F) \cap
	U(F') = \emptyset$.

\end{enumerate}
\qed

\lemma[1] Každé otevřené $\sigma$-lokálně konečné pokrytí topologického
	prostoru $X$ má lokální konečné zjemnění (ne nutně z otevřených množin).
\dukaz \qed

%% 3. hodina
\lemma[2] Buď $X$ regulární prostor. Má-li každé otevřené pokrytí X lokálně
	konečné zjemnění, pak každé otevřené pokrytí prostoru X má lokálně konečné
	uzavřené zjemnění.
\dukaz \qed

\lemma[3] Buď $X$ topologický prostor. Má-li každé otevřené pokrytí prostoru
	$X$ uzavřené lokálně konečné zjemnění, pak každé otevřené pokrytí prostoru
	$X$ má otevřené lokálně konečné zjemnění.
\dukaz \qed

\begin{veta}
Pro každý \Hausd~prostor $X$ jsou následující výroky ekvivalentní.
	\begin{enumerate}[(a)]
		\item $X$ je parakompaktní.
		\item $X$ je regulární a každé otevřené pokrytí prostoru $X$ má
			$\sigma$-lokálně konečné otevřené zjemnění.
		\item $X$ je regulární a každé otevřené pokrytí prostoru $X$ má lokálně
			konečné zjemnění (sestávající z libovolných množin).
		\item Každé otevřené pokrytí prostoru $X$ má lokálně konečné uzavřené
			zjemnění.
	\end{enumerate}
\end{veta}

\head {Důsledek:} Každý Lindelöfův prostor je parakompaktní.

\definice Nechť $\p U$ je pokrytí prostoru $X$, pokrytí $\p V$ se nazývá {\bf
	hvězdicovitým zjemněním} pokrytí $\p U$, značíme $\p V \prec^\star \p U$,
	$\{ V \in \p V : x \in V\}$ je hvězda bodu vůči $\p V$.

	Analogicky pro $M \subseteq X$:
		$$st(M, \p V) = \bigcup \{ V \in \p V : M \cap V \neq \emptyset \}$$
		$$st^n(x, \p V) = st(st^{n-1}(x, \p V), \p V),~\text{pro } n > 1$$

\head {Poznámka:} Uniformita $\Omega$ indukuje topologii:
	$$U \subseteq X~\text{je otevřená} \leftrightarrow \forall x \in U~\exists V \text{okolí } \triangle,~V \in \Omega,~V[x] \subseteq U,~V[x] = \{ y :~<x,y> \in V\}$$
	$$U \subseteq X~\text{je otevřená} \leftrightarrow \forall x \in U~\text{existuje uniformní pokrytí } \p V \in \Omega,\text{ že } st(x, \p V) \subseteq U.$$
	Topologický prostor $X$ je uniformizovatelný $\iff$ $X$ je úplně regulární.

\veta Hausdorfův prostor $X$ je parakompaktní $\iff$ ke každému otevřenému
	pokrytí prostoru $X$ existuje otevřené hvězdicovíté zjěmnění.
\dukaz \qed

%% 4. hodina
\veta Je-li prostor $X$ parakompaktní, $Y \subseteq X$ množina typu $F_\sigma$,
	pak podprostor $Y$ je parakompaktní.
\dukaz \qed

\head {Důsledek:} Parakompaktnost je dědičná na uzavřeném podprostoru.

\section{Metrizační věty}
\lemma Buď $\p C_n$ posloupnost pokrytí množiny $X$ takové, že $\p C_0 =
	\{X\}$, pro $n \geq 0$, že\\$\p C_{n+1} \prec^{\star\star\star} \p C_n$. Pak
	existuje nezáporná reálná funkce $d : X \times X \to \m R$ taková, že:

	\begin{enumerate}[(a)]
		\item $d(x,y) = d(y,x) \geq 0$
		\item $d(x,y) \leq d(x,z) + d(z,y)$
	\end{enumerate}
	A platí: označíme-li $B_d(x,\epsilon) = \{y \in X : d(x,y) < \epsilon \}$, potom
	\begin{enumerate}[(a)]
		\setcounter{enumi}{2}
		\item $\p C_{n+1} \prec^{\star} \{ B_d(x,2^{-n}) : x \in X\} \prec^{\star} \p C_n$
	\end{enumerate}
\dukaz \qed

\veta [P. S. Uryshon] Buď $X$ uniformizovatelný prostor. Pak $X$ je
	metrizovatelný $\iff$ jeho topologie je indukována uniformitou se spočetnou
	bází.

\definice [Diagonální zobrazení]
	Prostor $Y_\alpha (\alpha \in A)$, zobrazení $f_\alpha : X \to Y_\alpha (\alpha \in A)$,
	diagonální součin je:
		$$\Delta_{\alpha \in A} f_\alpha : X \to \prod_{\alpha \in A} Y_\alpha;~((\triangle_{\alpha \in A} f_\alpha(x))_\alpha = f_\alpha(x))$$
\begin{equation}
  \Delta_{\substack{
            1\le i \le n\\
            1\le j \le m}}
     M_{i,j}
\end{equation}
 $$
 \substack{ \Delta \\ a \in A } f_a
 $$

 $$
 \substack{ \triangle \\ a \in A } f_a
 $$

\veta [Bing, Nagata, Smirnov] Pro regulární prostor $X$ jsou následující výroky
	ekvivalentní:
	\begin{enumerate}[(a)]
		\item $X$ je metrizovatelný
		\item $X$ má $\sigma$-lokálně konečnou otevřenou bázi
		\item $X$ má $\sigma$-diskrétní otevřenou bázi
	\end{enumerate}
\dukaz (a) $\rightarrow$ (c): Stoneova věta

(c) $\rightarrow$ (b): triviální

(b) $\rightarrow$ (a): (zatím chybí)
\qed

\definice $H$ je metrický ježek s $A$ ostny. Pokud $H = [0,1] \times A / q$, $(0, \alpha)~q~(0, \beta);~\alpha,~\beta \in A$.

\head {Důsledek:}
	Každý metrizovatelný prostor lze vnořit do spočetného součinu metrických
	ježků.

\end{document}
