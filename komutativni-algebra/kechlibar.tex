\documentclass[11pt,a4paper]{article}
\usepackage[utf8]{inputenc}
\usepackage[czech]{babel}
\usepackage{amsfonts,amsthm,amsfonts,amssymb,amsmath}
\usepackage{a4wide}
\usepackage{enumerate}
\usepackage[IL2]{fontenc}

\newcommand\m[1]{\mathbb { #1 }} % tucne pismeno/text
\newcommand\p[1]{\mathcal{ #1 }} % psaci pismeno/text
\newcommand\IFF{\ensuremath{\iff}}
\newcommand\N{\m N}

\newcounter{numb}

\theoremstyle{definition}
\newtheorem*{definice}{Definice}
\newtheorem*{pozorovani}{Pozorování}
\newtheorem{poznamka}[numb]{Poznámka}

\theoremstyle{plain}
\newtheorem{veta}[numb]{Věta}
\newtheorem{lemma}[numb]{Lemma}
\newtheorem{tvrzeni}[numb]{Tvrzení}
\newtheorem{dusledek}[numb]{Důsledek}

\title{Konstrukce noetherovského okruhu, jehož celistvý uzávěr\\ není
	noetherovský}
\author{Adam Bartoš, Tomáš Jakl, Jan Kosina, Martin Raška, Jiří Vančura}

\begin{document}
\maketitle
\section{Úvod}

\section{Základní vztahy}
V našem textu budeme pracovat pouze s komutativními okruhy.

\definice Nechť $R$ je okruh, $X$ nějaká neurčitá a $I$ nějaký ideál $R$.
\begin{itemize}
	\item Potom okruh $R[[X]]$ je okruh formálních mocninných řad jedné
	neurčité $X$.
	\item Krullova dimenze okruhu je $$alt(R) = sup \{ |\p P| : \p P~\text{je
	ostře rostoucí posloupost prvoideálů} \} - 1.$$
	\item Označme $gen(I)$ minimální kardinalitu počtu generátorů ideálu $I$.
	\item Pokud je okruh $R$ lokální, má jediný maximální ideál. Označme tento
	maximální ideál $M_R$.
\end{itemize}

\veta Nechť $T$ je těleso a $R = T[[x]]$ je obor formálních mocninných řad. Pak
$f \in R$ je invertibilní \IFF má nenulový absolutní člen.

\textbf{TODO} kontrola: Důkaz pouze opsán z Kechlibara.

\begin{proof}
	Zřejmě každá jednotka oboru $R$ musí mít nenulový absolutní člen. Naopak,
	nechť $f\in R$ má nenulový absolutní člen. Bez újmy na~obecnosti
	lze~před\-po\-klá\-dat, že~tento absolutní člen je~roven jedné. Položme $g=
	1-f$ a~nechť $h =  \sum_{i= 0}^\infty g^i$, čili formálně, je-li $g=
	\sum_{i= 1}^\infty g_ix^i$, pak $$h= 1+\sum_{i= 1}^\infty
	\sum_{(k_1,\dots,k_l\in \Bbb N_0, k_1+\dots+k_l= i)} g_{k(1)}\cdots
	g_{k(l)}x^i$$ Vezměme nyní $n\in \Bbb N$. Pak $f\sum_{j= 0}^n g^j =
	(1-g)\sum_{j= 0}^n g^j =  1 - g^{n+1}$. Ovšem $g^{n+1}$ má~koeficienty
	u~$x,x^2,\dots,x^n$ nulové, takže $h$ je~inverzní prvek k~$f$ v~$R$, a~$f$
	je tedy jednotka.
\end{proof}

\tvrzeni[Zobecněná Hilbertova věta] Buď $R$ okruh, následující je ekvivalentní.
\begin{enumerate}
	\item $R$ je noetherovský.
	\item $R[X]$ je noetherovský.
	\item $R[[X]]$ je noetherovský.
\end{enumerate}

\tvrzeni[Věta o Krullově dimenzi] Buď $R$ lokální noetherovský okruh. Platí
	$$gen(M_R) \geq alt(R).$$

\definice Buď $R$ lokální noetherovský obor. Řekneme, že $R$ je regulární, když
	$$gen(M_R) = alt(R) < \omega.$$
	
\section{Vlastnosti okruhů $A$ a $B$}
Následující definované struktury budeme volně používat po zbytek textu se
stejným značením/pojmenováním.

\definice ~\\[-1.5em]
\begin{itemize}
	\item $K_0$ buď perfektní těleso charakteristiky $p$
	\item $K$ buď $K_0(X_i : i < \omega)$ (podílové těleso okruhu $K[X_i : i <
	      \omega]$), kde $X_i$ jsou nové různé proměnné
	\item $A$ buď $K[[Z_1,Z_2,Z_3]]$
	\item $B$ buď $K^p[[Z_1,Z_2,Z_3]][K]$
\end{itemize}
\textbf{TODO} Stejně značíme definici vlastností a definici konkrétních
objektů. Nějak očesat?


\veta $A$ je lokální, noetherovský, $alt(A) = 3$ a je regulární.
\begin{proof}
	$A$ je lokální, protože jeho maximální ideál $Z_1A + Z_2A + Z_3A$ obsahuje
	právě všechny neinvertibilní prvky okruhu. Ze zobecněné Hilbertovy věty
	plyne, že je noetherovský.

	Protože $3 \leq alt(A) \leq gen(M_A) \leq 3$, kde první nerovnost plyne z
	toho, že $0 \subsetneq Z_1A \subsetneq Z_1A + Z_2A \subsetneq Z_1A + Z_2A +
	Z_3A$ je řetězec ideálů, druhá nerovnost z Věty o Krullovy dimenzi.

	Tedy $alt(A) = gen(M_A) = 3$ a protože $A$ je lokální noetherovský, je i
	regulární.
\end{proof}

\lemma Buď $S$ noetherovský okruh a Buď $R \subseteq S$ jeho podokruh. Pokud
pro každý konečně generovaný ideál $I \subseteq R$ platí, že $IS \cap R = I$
pak $R$ je noetherovský.
\begin{proof}
	Budeme postupovat sporem. Ať $R$ není noetherovský okruh, pak tedy existuje
	$I \subseteq R$, který není konečně generován. Existuje nekonečná
	posloupnost ideálů $I_i$ tak, že:

	$$I_1 \subsetneq I_2 \subsetneq \quad \ldots \quad \subseteq I$$

	\noindent Potom, ale pro posloupnost ideálů $I_1S, I_2S, \dots$ v $S$ z
	noetherovskosti existuje $n \in \N$ tak, že:

	$$I_1S \subseteq I_2S \subseteq \quad \ldots \quad \subseteq I_nS =
	I_{n+1}S = \quad \ldots \quad = IS$$

	\noindent Podle předpokladu je $I_n = I_nS \cap R = I_{n+1}S \cap R =
	I_{n+1}$, ale $I_n \not= I_{n+1}$, což je spor.
\end{proof}


\lemma Buď $T \subseteq U$ tělesa, $\{X_i: i < n\}$ neurčité, pak $T[[X_i: i <
n]][U]$ je noetherovský obor.
\begin{proof}
	Označme $R = T[[X_i: i < n]][U]$, $S = U[[X_i: i < n]]$. Platí, že $R
	\subseteq S$ a $S$ je podle zobecněné Hilbertovy věty noetherovský obor. K
	použití předchozího lemmatu stačí ukázat, že pro každý konečně generovaný
	ideál $I \subseteq R$ platí, že $IS \cap R \subseteq I$. Buď $I = \sum_{i =
	1}^m r_i R$, konečně generovaný ideál v $R$ pro nějaká $r_i \in R$. Buď $r
	\in IS \cap R$, ukážeme, že $r \in I$.

	Z $r \in IS = \sum_{i = 1}^m r_i RS = \sum_{i = 1}^m r_i S$, víme, že $r =
	\sum_{i = 1}^m r_i f_i$, pro nějaká $f_i \in S$. Platí, že $s \in S$ je v
	$R$, právě tehdy když existuje $U_s = \{u_i: i = 1, \dots, l\} \subseteq U$
	a $\{s_i: i = 1, \dots, l\} \subseteq R$, že $s = \sum_{i = 1}^l u_i s_i$.

	Uvažme těleso $W = T(U_r \cup \bigcup_{i = 1}^m U_{r_i}) \subseteq U$. $W$ je
	z definice konečného lineárního stupně nad $T$. Máme tedy $r$, $r_i \in
	W[[X_i: i < n]] \subseteq R$. Uvažme rozklad $U = W \oplus U'$. Ten
	indukuje rozklad $S = W[[X_i: i < n]] \oplus U'[[X_i: i < n]]$. Proto
	můžeme rozepsat každý z prvků $f_i$ jako $f_i = g_i + h_i$, kde $g_i \in
	W[[X_i: i < n]]$, $h_i \in U'[[X_i: i < n]]$.

	Potom $r = \sum_{i = 1}^m r_i f_i = \sum_{i = 1}^m r_i g_i + \sum_{i = 1}^m
	r_i h_i$, přičemž $r$ i $\sum_{i = 1}^m r_i g_i \in W[[X_i: i < n]]$ a
	$\sum_{i = 1}^m r_i h_i \in U'[[X_i: i < n]]$, a tedy $\sum_{i = 1}^m r_i h_i
	= 0$. Celkem $r = \sum_{i = 1}^m r_i g_i \in \sum_{i = 1}^m r_i W[[X_i: i <
	n]] \subseteq \sum_{i = 1}^m r_i R = I$. Tedy $r \in I$.
\end{proof}

\veta $B$ je lokální, noetherovský, $alt(B) = 3$ a je regulární.
\begin{proof}
Protože $K^p \subseteq K$ je těleso. Je užitím předchozí věty $B$ noetherovský.

\dots (TODO lokalita)
\end{proof}

\tvrzeni[Fakt] Každý regulární obor je Gaussův.

\tvrzeni[Fakt] Každý Pseudobezoutův obor (a tím spíše Gaussův obor) je celistvě
uzavřen.
\dusledek Okruhy $A$ a $B$ jsou celistvě uzavřeny.

\textbf{TODO} Je potřeba v dalším textu? + Přidat důkaz?

\section{Okruhy $B[b_0]$ a $B[b_i : i < \omega]$}

\definice Označme formální mocninné řady $b_j$, $c_j$ a $d_j \in A$ pro všechna
$j \in \N$, dané předpisem:
\begin{itemize}
	\item $c_j = \sum_{i < \omega} X_{2(i + j)} Z^i_3$
	\item $d_j = \sum_{i < \omega} X_{2(i + j) + 1} Z^i_3$
	\item $b_j = Z_1c_j + Z_2d_j$
	\item $\p M = \{ m : m \text{ je monický polynom o proměnných } Z_1, Z_2,
	Z_3 \text{ v } B[b_j : j < \omega] \}$
\end{itemize}

\pozorovani $c_k = X_{2k} + Z_3 c_{k+1}$, $d_k = X_{2k+1} + Z_3 d_{k+1}$

\veta $B[b_0]$ je noetherovský.
\begin{proof}
	 Protože $B[b_0] \simeq B[Y] / \{ f \in B[Y] : f(b_0) = 0 \}$. Díky
	 zobecněné Hilbertově větě je $B[Y]$ noetherovský a proto je také $B[b_0]$
	 noetherovský.
\end{proof}

\lemma $\forall i < \omega\colon b_i \in B(b_0)$
\begin{proof}
	Protože $c_k = X_{2k} + Z_3\,c_{k+1}$, $d_k = X_{2k+1} + Z_3\,d_{k+1}$ a $b_k
	= Z_1 c_k + Z_2 d_k = Z_1 X_{2k} + Z_3 X_{2k + 1} + Z_3 b_{k + 1}$. Můžeme
	$b_{k+1}$ vyjádřit jako $\frac{b_k - Z_1 X_{2k} - Z_2 X_{2k+1}}{Z_3}$ (jsme
	v podílovém tělese). Dále indukcí.
\end{proof}

\lemma\label{celistvostA} $A$ je celistvé nad $B$.
\begin{proof}
	Dokážeme si pozorování: $\forall f \in A~\exists g$, že $f^p = f^p_0 + X^p
	g^p$, kde $f_0$ je absolutní člen $f$. Protože $f = f_0 + (f - f_0)$, pak
	$f^p = \sum_{i = 0}^p {p\choose i} f^i_0\,(f-f_0)^{p-i} = f^i_0 + (f-f_0)^p$
	(rozšíření oboru zachovává jeho charakteristiku, takže obor je
	charakteristiky $p$).

	Pro libovolné $f \in A$, je $f = \sum_{m \in \p M} f_m\,m$ pro $f_m \in K$
	a platí, že $f^p = \sum_{m\in\p M} f^p_m\,m^p \in B$.
\end{proof}

\lemma[VII.7, jen strana nepožadující ($p=2$)] $B[b_i : i < \omega] \supseteq
\bigcup_{k < \omega} B[b_i : i < k]$
\begin{proof}
	Víme, že $\bigcup_{k < \omega} B[b_i : i < k] = \bigcup_{k < \omega}
	B[b_k]$.

	Nechť $a = \frac{\beta_1 + \beta_2 b_0}{Z_3r^l} \in A$, pro nějaká
	$\beta_1, \beta_2 \in B$. Označme $\beth_l$ prvních $l$ členů mocninného
	rozvoje $Z_3$ v $b_0$:
	
	\[
	b_0 = \underbrace{\sum_{i < l} (X_{2i} Z_1 + X_{2i+1} Z_2)
	Z_3^i}_{\beth_l} + Z_3^l b_l.
	\]

	Potom z $b_0 = \beth_l + Z_3^l\,b_l$ a $\beta_1 + \beta_2\,b_0 = Z_3^l a$
	platí, že $Z_3^l a - Z_3^l\,\beta_2\,b_l = \beta_1 + \beta_2\,\beth_l$.
	Protože $\beta_1 + \beta_2\,\beth_l \in B$, je i $Z_3^l (a - \beta_2\,b_l)
	\in B$. Libovolná řada náleží $B$ pouze v závislosti na jejích
	koeficientech u $Z_1, Z_2, Z_3$. Proto pokud $Z_3 v \in B$, pak už i $v \in
	B$. A tedy i $(a - \beta_2\,b_l) \in B$.

	Z toho dostáváme požadované $a = (a - \beta_2\,b_l) + \beta_2\,b_l \in B[b_i
	: i < \omega]$.
\end{proof}

\veta $B[b_i : i < \omega]$ je celistvým uzávěrem $B[b_0]$ v $A$.
\begin{proof}
	Že $B[b_i : i < \omega]$ je obsaženo v celistvém uzávěru $B[b_0]$ plyne z
	Lemmatu \ref{celistvostA}, protože každý prvek z $A$ je celistvý v $B$
	(tedy i všechny $b_i$) a o to spíše je celistvý v $B[b_0]$.

	\dots (\textbf{TODO})
\end{proof}


\veta $B[b_i : i < \omega]$ není noetherovský.
\begin{proof}
	Označme $\tilde{B} = B[b_i : i < \omega]$. Dokážeme, že posloupnost
	ideálů $I_n = \sum_{i < n} b_i \tilde{B}$ je ostře rostoucí. K tomu
	stačí ukázat, že $b_n \notin I_n$, $\forall n < \omega$.

	Budeme postupovat sporem. Ať $n \in \N$ je nejmenší, že $b_n \in I_n$.
	Existuje tedy poslouponost $\{e_i\}_{i < n} \subset \tilde B$ tak, že
	$b_n = \sum_{i < n} e_i b_i$. Každý z $e_i \in \tilde B$, a proto jde
	vyjádřit jako $e_i = \sum_{m \in \p M} f^i_m m$, pro $f^i_m \in B$, kde
	pouze konečně mnoho $f^i_m$ je nenulových.

	Potom
	\[
	b_n = \sum_{i < n} \sum_{m \in \p M} f^i_m m b_i = \sum_{i < n} \sum_{m \in
	\p M \setminus \{1\}} f^i_m m b_i + \sum_{i < n} f^i_m m b_i \;.
	\]
	A protože $f^i_1 = g_i + Z_1 g_{i,1} + Z_2 g_{i,2}$ pro nějaká $g_i \in
	K^p[[Z_3]][K]$ a $g_{i,1}, g_{i,2} \in B$. Dostáváme
	\[
	b_n = \sum_{i < n} \sum_{m \in \p M \setminus \{1\}} f^i_m m b_i +
	\sum_{i<n} \left(Z_1 g_{i,1} + Z_2 g_{i,2}\right)b_i + \sum_{i<n} g_i b_i
	\;.
	\]

	První dvě sumy obsahují ve svých členech vždy některé z $Z_1^2, Z_2^2$ nebo
	$Z_1 Z_2$, ale $b_n$ neobsahuje žádní z nich. Proto $b_n = \sum_{i<n} g_i
	b_i$.

	Porovnáním koeficientů u mocnin $Z_1, Z_2, Z_3$ v rovnosti pro $b_n$
	dostáváme, že
	\[
	c_n = \sum_{i<n} g_i c_i \quad \& \quad d_n = \sum_{i<n} g_i d_i
	\]
	protože $g_i \in K^p[[Z_3]][K]$.
	\\

	Dále \dots (nemám zbytek)
\end{proof}

\textbf{TODO} Střídá se značení $i < \omega$ a $n \in \N$, udělat jednotně?
Jednotně taky psaní řad, tzn. jestli je člen $X$ či $Z_i$ před nebo za
koeficientem.

\end{document}
