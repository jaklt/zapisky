\chapter{Used optimisations in search engines}

In order to write strong playing program in given game, it is necessary to
enhance chosen algorithm with various extensions. In this chapter we describe
the most used extensions for AlphaBeta or MCTS algorithm.

\section{AlphaBeta}
\subsection{Transposition table}
COMPLETELY REWRITE: If we find in TT occurs result from previous search that is
shallower than we need, we prefer steps from Principal Variation of given entry
in succeeding search and use also its bounds.

If we find satisfying result in TT we use its full PV and score as ours.

\subsection{Iterative deepening framework}
\subsection{Aspiration Windows}
% TODO how to mention it?
% \subsection{Quiescent search}
%   Is nice way, how to reduce horizont effect ...
%   % [Related to arimaa: http://arimaa.com/arimaa/forum/cgi/YaBB.cgi?board=devTalk;action=display;num=1122418533] <- TODO: Maybe implement this way
%   % [http://mediocrechess.sourceforge.net/guides/quiescentsearch.html]
\subsection{Move ordering}
The following methods change order in which branches are selected and then
inspected. It is very important for AlphaBeta search to have nodes well
ordered. Because the earlier we find pruning child of the node the shorter time
we spend in it.
	\subsubsection{History heuristics}
	The main idea behind this extension is that if some step is good enough
	to cause so many pruning anywhere in the search tree and if it is valid in
	given position it could be also good here.

	It is implemented in way that we create table with statistic for every
	combination of players, piece, position and direction.
	  % [http://webdocs.cs.ualberta.ca/~jonathan/PREVIOUS/Courses/657/index.html]

	\subsubsection{Killer moves}
	When a step prunes branches in some position it is very natural to ask if
	the same step could cause pruning in another branch and the same depth of
	the tree. To go even further we take two last steps caused pruning to be
	preferred in the search. As Zhong found out three or more Killer moves would help. [ZHONG]

	\subsubsection{Null move}

\subsection{Heuristics}
All preceding optimisations do not change result given by the algorithm. They
could only significantly decrease amount of time needed to obtain such result.

The consequent optimisations are rather heuristics, because they try to
give you approximately good result in much shorter time. In this work, we
will not use those optimisations.


- Negascout/PVS
- MTD-f
% TODO XXX: Are really heuristics?


\section{MCTS}
\subsection{Transposition table}
If we look at Arimaa, Chess or Go game tree there is so many repetitions in
nodes of the tree for almost every board position. The Transposition tables
are used to reduce the number of repetitions in tree.

To do so we bind nodes consideret the same to one. When we Transposition
tables the game tree looks more like Direct Acyclic Graph than tree.

\subsection{Progressive bias}
- add to uct formula $+ {H_B \over n_i}$. Where $H_B$ is progressive bias
coeficient as described in 
\subsection{History heuristics}
- add to uct formula $+ {hh_i \over \sqrt n_i}$. Where $hh_i$ is history
heuristics coeficiend as is described in Kozeleks thesis\cite{KOZELEK}.

Similar approach is All-Moves-As-First Heuristics however it updates statistics only for all steps that were used in this simulation (in each node) (TODO)

\subsection{Best-of-N}
\subsection{Children caching}
\subsection{Virtual visits}
Initialise with $v$ visits.
\subsection{Maturity threshold}

- In MonteCarlo simulation, it should (as Kozelek wrote) significantly improve
  strength of program by choosing random moves rather with some heuristic.
  % TODO: really significantly??
	- One way how to choose step is to generate all steps and chose one with
	  the best value given by some incrementally precomputed function.
	  % TODO check word order
	- Another similar way how to choose step is from all possible generated
	  steps choose at random $r$ of them and from these $r$ choose one with the
	  highest value given by the same function as above.

\section{General}
The following extensions are considered must have in every successful Arimaa
bot and do not depend on used algorithm.

	\subsection{Zobrist keys}
	Motivation ...

	For every element of $Piece\times Player\times Position$ the random 64 bit
	number is generated. ...

	How nodes are stored in Transposition table entries

	\subsection{Bitboards}
	Because Arimaa could be played with standart chess set it is also possible
	to use from world of chess well known board representation called
	Bitboards. One may say that bitboards suits even better for Arimaa than for
	Chess.

\section{Comparison of used optimisations}

\subsection{History heuristic}
In MCTS we prefer moves that were often chosen in going through tree. In
AlphaBeta value of move is increasing when branch using selected step is
pruned. Every time we order branches of the search tree by value given by
history heuristics. This approach should increase amount of pruning.

\subsection{Transposition table}

