%% Verze pro jednostranný tisk:
% Okraje: levý 40mm, pravý 25mm, horní a dolní 25mm
% (ale pozor, LaTeX si sám přidává 1in)
\documentclass[12pt,a4paper,fleqn]{report}
\setlength\textwidth{145mm}
\setlength\textheight{247mm}
\setlength\oddsidemargin{15mm}
\setlength\evensidemargin{15mm}
\setlength\topmargin{0mm}
\setlength\headsep{0mm}
\setlength\headheight{0mm}
\let\openright=\clearpage

\usepackage[utf8]{inputenc}
\usepackage{graphicx}
\usepackage{amsthm}
% \usepackage{indentfirst}
\usepackage{amsfonts}
\usepackage{a4wide}
\usepackage[footnote]{acronym}

\usepackage[unicode]{hyperref}
\hypersetup{pdftitle=Arimaa challenge - comparission study of MCTS versus alpha-beta methods}
\hypersetup{pdfauthor=Tomáš Jakl}

\acrodef{MCTS}{Monte Carlo Tree Search}


%%% Drobné úpravy stylu

% Tato makra přesvědčují mírně ošklivým trikem LaTeX, aby hlavičky kapitol
% sázel příčetněji a nevynechával nad nimi spoustu místa. Směle ignorujte.
\makeatletter
\def\@makechapterhead#1{
  {\parindent \z@ \raggedright \normalfont
   \Huge\bfseries \thechapter. #1
   \par\nobreak
   \vskip 20\p@
}}
\def\@makeschapterhead#1{
  {\parindent \z@ \raggedright \normalfont
   \Huge\bfseries #1
   \par\nobreak
   \vskip 20\p@
}}
\makeatother

% Toto makro definuje kapitolu, která není očíslovaná, ale je uvedena v obsahu.
\def\chapwithtoc#1{
\chapter*{#1}
\addcontentsline{toc}{chapter}{#1}
}

\begin{document}

% Trochu volnější nastavení dělení slov, než je default.
\lefthyphenmin=2
\righthyphenmin=2

%%% Titulní strana práce

\pagestyle{empty}
\begin{center}

\large

Charles University in Prague

\medskip

Faculty of Mathematics and Physics

\vfill

{\bf\Large BACHELOR THESIS}

\vfill

\centerline{\mbox{\includegraphics[width=60mm]{logo.eps}}}

\vfill
\vspace{5mm}

{\LARGE Tomáš Jakl}

\vspace{15mm}

% Název práce přesně podle zadání
{\LARGE\bfseries Arimaa challenge -- comparission study of MCTS versus alpha-beta methods}

\vfill

Department of Theoretical Computer Science and Mathematical Logic

\vfill

\begin{tabular}{rl}

Supervisor of the bachelor thesis: & Mgr. Vladan Majerech Dr. \\
\noalign{\vspace{2mm}}
Study programme: & Computer Science \\
\noalign{\vspace{2mm}}
Specialization: & General Computer Science \\
\end{tabular}

\vfill

Prague 2011

\end{center}

\newpage

%%% TODO Na tomto místě mohou být napsána případná poděkování (vedoucímu práce,
%%% konzultantovi, tomu, kdo zapůjčil software, literaturu apod.)

\openright

\noindent
Poděkování.


%%% Strana s čestným prohlášením k bakalářské práci

\vglue 0pt plus 1fill

\noindent
I~declare that I~carried out this bachelor thesis independently, and only with
the cited sources, literature and other professional sources.

\medskip\noindent
I~understand that my work relates to the rights and obligations under the Act
No. 121/2000 Coll., the Copyright Act, as amended, in particular the fact that
the Charles University in Prague has the right to conclude a license agreement
on the use of this work as a school work pursuant to Section 60 paragraph 1 of
the Copyright Act.

\vspace{10mm}

\hbox{\hbox to 0.5\hsize{%
In Prague date \today
\hss}\hbox to 0.5\hsize{%
Tomáš Jakl
\hss}}

\vspace{20mm}
\newpage

%%% Povinná informační strana bakalářské práce (TODO)

\vbox to 0.5\vsize{
\setlength\parindent{0mm}
\setlength\parskip{5mm}

Title:
Arimaa challenge -- comparission study of MCTS versus alpha-beta methods

Author:
Tomáš Jakl

Department:
Department of Theoretical Computer Science and Mathematical Logic

Supervisor:
Mgr. Vladan Majerech Dr., Department of Theoretical Computer Science and Mathematical Logic

Abstract:
In the world of chess programming the most successful algorithm for game tree
search is considered AlphaBeta search, however in game of Go it is Monte
Carlo Tree Search. Arimaa has similarities with both Go and Chess, but there
has been no successful program using Monte Carlo Tree Search so far. The main
goal of this thesis is to compare capabilites given by Monte Carlo Tree Search
algorithm and AlphaBeta search, both having the same evaluation function, in
game of Arimaa.

Keywords: Arimaa, Monte Carlo Tree Search, Alpha Beta search

\vss}\nobreak\vbox to 0.49\vsize{
\setlength\parindent{0mm}
\setlength\parskip{5mm}

Název práce:
Arimaa challenge -- srovnávací studie metod MCTS a alfa-beta

Autor:
Tomáš Jakl

Katedra:
Katedra teoretické informatiky a matematické logiky

Vedoucí bakalářské práce:
Mgr. Vladan Majerech Dr., Katedra teoretické informatiky a matematické logiky
Jméno a příjmení s tituly

Abstrakt:
% TODO abstrakt v rozsahu 80-200 slov; nejedná se však o opis zadání bakalářské práce

Klíčová slova: Arimaa, Monte Carlo Tree Search, Alpha Beta search


\vss}

\newpage

\openright
\pagestyle{plain}
\setcounter{page}{1}
\tableofcontents

%%% Chapters
% \chapter*{Introduction}
% \addcontentsline{toc}{chapter}{Introduction}

\chapter{Introduction}
After Gasparov was defeated, new challenges has come. Defeat men in Go. Until
now, there were no significant success on standart 13$\times$13 board. But many
useful new way of playing games was invented. In ??? Monte Carlo methods were
successfully used to defeat all other computer players of Go. After that day
all successful programs were using Monte Carlo methods.
% TODO so many ???s

Tomáš Kozelek has shown in his work, that building Arimaa playing program using
MCTS is possible. In this work we will focus on comparing capabilities and
perspectives to the future given by AlphaBeta search and MCTS to game of
Arimaa.

After first computers were created it was always in human target to fight
with/compete human mind in every occasion. First significant result was shown
in 1997. IBM constructed/built computer with just one purpose, to defeat
the best human player in the game of chess. It took few years of development
and ??? milions of dolars to build computer. Fist time they wasn't successful
[???], but after some time and more effort they defeated Gasparov with computer
named Deep blue.
% TODO preformulovat a upresnit/rozsirit
[1] and [???]

We will introduce the game of Arimaa and describe two algorithms \ac{MCTS} and AlphaBeta search ... TODO


\section{Terminology}
\begin{description}
\item[Game bot] is game playing program.
\item[Game tree] for arbitrary game is tree with starting positions as root and
   with children of nodes as all possible consequent positions.
\item[Evaluation function] is function which estimates value of given position of the game. It can be used for example to compare which of two given positions is better.
\item[Minimax tree] for two player game is game tree limited to some depth with
added values in all nodes. Values are defined recursively. In leaf of the tree
is value calculated by evaluation function. In nonleaf node is the value
defined as the best value from nodes children from nodes active player point
of view.
\item[Principal variation] is best sequence of moves for actual player leading
from root of the tree to leaf in minimax if we presume both players play their
best.
\item[Transposition] is 
\item[Branching factor] 
\end{description}

\section{The Game of Arimaa}
The game of Arimaa is pretty new game. It is carefully designed to be hard to
play for computers, but easy to play for humans. Creator of the game Omar
Syed [1]

The game was carefully designed in order not to be possible to use methods
well known from Chess as game-ending tables or opening tables. (TODO
REWRITE:) Also to be significantly harder to precompute moves for huge
number moves to future and to efficiently decide which of two given position
is better.

(TODO REWRITE:) Omar says that gasparov was not oversmarted but overcomputed ...
In Arimaa it is significantly harder to precompute moves ??because?? and to
efficiently decide which of two given position is better ??because??.

Why we cannot use standart methods widely used in chess? (Section 2 in Kozeleks thesis)

\section{Rules of the game~\cite{arimaa.com}}
Arimaa is two-player zero-sum game with perfect information. It is designed to
be possible to play it using the board and pieces from chess set. The starting
player has Gold color, second is Silver. In the zeroth turn Gold and then
Silver player each place all their pieces into first two (for Gold) or last two
(for Silver) lines of the board any way they consider appropriate. Piece set
consist of eight Rabbits, two Cats, two Dogs, two Horses, Camel and Elephant in
order from weakest to strongest.

%% TODO example picture of the starting position

Players are taking turns starting with Gold. In each turn player makes move,
which consist of one to four steps. If less than four steps were made it is
said that player passed. After move ends, position of the board must be
different from position before turn started.

Figure is considered frozen if on one of its neighbouring squares is opponents
stronger piece and on no figure sharing color with it. Adjacent or neighbouring
squares are those squares lying in one position to left, right, front or
backwards.

%% TODO example picture of the frozen piece and possibility to push and pull

To make step player chooses one of its non frozen figures and move it to one of
the free adjacent squares, with one exception -- rabbits cannot step backwards.
Instead of making simple step player can provide pull or push. Pulling is
almost identical to simple step, but after step is done player choose one of
the weaker piece which stayed on neighbouring of the square from which figure
moved, and move it to position from which our piece stepped. Pushing is a kind
of opposite to pulling. At first neighbouring weaker piece is moved to its free
adjacent square and then our piece is placed to weaker pieces former place.
Push and pull counts as 2 steps and can not be combined together.

On the board are four special squares called traps in positions \texttt{c3},
\texttt{c6}, \texttt{f3} and \texttt{f6}. After each step ends if any piece is
in trap and has no adjacent piece sharing color with him, it is trapped and
therefore removed from the board.

At end of the move is checked if game ended and it happens if one of the
following conditions is fulfilled:
\begin{enumerate}
\item one of the players rabbit has reached opponents front line (goalline), we
call it goal,
\item one of the players has no rabbit left, we call it elimination,
\item next active player cannot move, we call it immobilisation,
\item or the same position is repeated in third time in the row.
\end{enumerate}

Player wins if he scores goal or opponent is eliminated, is immobilised or
repeated position for third time.

%TODO figures with examples of pushing/pulling, immobilisations, trapping, ...

\section{Comparison to Go and Chess}
Because Arimaa is played with full chess set it is very natural to ask about
similarities with the game of Chess. As in Chess is in Arimaa also so easy to
ruin good position with just one bad move. For example stepping out of trap an
therefore let another piece to be trapped or unfreezing rabbit near goalline.
Unlike in Chess starting position is not predefined and there are about
$4.207\times10^{15}$ different possible openings which makes it hard for arimaa
bot programmer to use any kind of opening tables~\cite{COX}.

In Arimaa it is also very hard to build good static evaluation function. Even
human players often do not know for example if in certain position it is better
to sacrifice Camel for trapping opponents Horse and Cat and often it depends on
many other factors. Building good evaluation function is very hard in game of
Go also, because it requires a lot of local searching to find territories and
determine their potential owners. On the other hand destroying good position by
making wrong step is in Go a lot harder.

In game of Go if we omit filling eye any random playing sequence leads to end,
which is not so easily achievable in Arimaa and Chess. Christ Cox showed in his
work that in Arimaa branching factor of average position is around 20,000. With
comparison in Chess it is only 35 and in Go 200~\cite{COX}.

% TODO? position in arimaa vs tactic in chess ~\cite{COX}

\section{Challenge}
Omar Syed decided to left a prize 10,000 USD for programmer or group of
programmers who develop Arimaa playing program which win Arimaa Computer
Championship and then defeat three chosen top human players before the year
2020. Omar Syed believes that this motivation will help further improvements in
area of AI game programming~\cite{syed}.

So far computers were not even close to defeat one of the chosen human
players~\cite{arimaa.com}. ... (TODO add facts)

\section{Object of research}
The main part of this work is to develop well documented Arimaa playing program
and try to answer the following questions:

\begin{enumerate}
\item Is MCTS competitive alpha beta search at all?
\item Is MCTS more promising engine than AlphaBeta search in the future with
      increasing number of cpus?
\item How important is evaluation function in MCTS compared to AlphaBeta?
\end{enumerate}
% TODO: which area should be compared? who is more perspective in future?


\chapter{Algorithms}
In this chapter we introduce two algorithms for searching minimax trees
AlphaBeta search and Monte Carlo Tree Search (MCTS). Both algorithms became
successful and dominating in some field. AlphaBeta search in game of Chess and
MCTS in game of Go. (TODO reformulate)

The trivial algorithm to search game tree could be Minimax search. It is
searching minimax tree in depth first search order and therefore the search do
not ends until all nodes at certain depth are explored. It is quite
noneffective with huge branching factor which Arimaa, Chess or Go have.

\section{Description of the AlphaBeta search}
AlphaBeta search has grown in popularity in game of Chess. Until now all
successful arimaa bots were using it with various enchants. AlphaBeta search is
natural optimisation of Minimax search. The main purpose of algorithm is to
reduce number of branches and nodes to be visited.

During the search we are trimming bounds (window) of the best minimax value.
The pseudocode of the algorithm is shown in Listing~\ref{alphabeta}.

\lstset{language=Python, caption=Pseudocode of the AlphaBeta search, label=alphabeta}
\begin{lstlisting}
alphabeta (node, depth, alpha, beta):
    if  depth = 0 or node is a terminal node:
        return evaluate(node)
    if  is maximizing node:
        for each child of node
            score = alphabeta(child, depth-1, alpha, beta,
                              oponent(Player))
            alpha = max(alpha, score)

            if beta <= alpha: break  # Beta cut-off
        return alpha
    else:
        for each child of node:
            score = alphabeta(child, depth-1, alpha, beta,
                              oponent(Player))
            beta = min(beta, score)

            if beta <= alpha: break  # Alpha cut-off
        return beta
\end{lstlisting}

In maximizing nodes we are improving lower estimate (alpha) of the minimax
value and in minimizing nodes the upper estimate (beta). If value from a child
forces these bounds to meet we know that better approximation of this node is
not possible and so we return our estimate, we say we pruned the search in
node.

As can be proven, if AlphaBeta finds solution for depth $n$, then it is the best
solution in MiniMax tree for depth $n$~\cite{knuth:alphabeta}. In optimal case
instead of examining $\mathcal O(b^d)$ nodes is only examined $\mathcal
O(b^{d/2})$ of them, where $b$ is branching factor of the game and $d$ is
searched depth~\cite{ZHONG}.

A lot of time is spared if the pruning children of nodes are listed as first.
However in the worst case if children are sorted in opposite order to the
optimal, the whole minimax tree is searched. Consequently it is very important
to have nodes well ordered.


\section{Description of the Monte Carlo Tree search}
Monte Carlo methods was formerly used for approximation of mathematical,
physical, biological and others processes where full calculation would be
difficult or even impossible. For example in mathematics it is used for numeric
integration or estimate the $\pi$. In games without perfect information, such
as poker, backgammon, or scrabble, using randomness has shown to be beneficial
too~\cite{MonteCarloMethod, MonteCarloGo}.

In game of Go was always a huge troublesome to build efficient evaluation
function. Unlike poker or backgammon, Go is a game with perfect information and
hence it is not so natural to use Monte Carlo methods. The first attempts to
use random approach as evaluating approximation of the Go position were in
1993. We present generalised Bernd's algorithm~\cite{BERND,KOZELEK}:

\begin{enumerate}
\item Play random game from given position to the end, with one exception,
	choose only steps not filling eyes. At the end of simulation count in the
	result of simulation for the first step played.
\item If there is time left go to 1.
\item Choose move with highest ratio between number times the move won when it
	  was played and the number of times it was played.
\end{enumerate}

However Bernd showed that after some time his algorithm indicates no further
improvement. It was necessary to realize that the main problem was that it
searched all branches with the same probability. (TODO) .... It should give more
promising branches more attention.


\subsection{Bandit Problem}
A $K$-armed bandit, is slot machine with $K$ arms. When arm is drawn it
produces reward. Distribution of each arm reward is independent on other arms
and previous draws of this arm. The task is to choose best strategy to maximize
sum of reward through iterative plays.~\cite{MoGo,MultiarmedBandit}

In~\cite{MultiarmedBandit} is presented thee UCB1 algorithm (where UCB stands for Upper Confidence Bounds) for Bandit Problem:

\begin{enumerate}
\item Play each arm of the bandit once.
\item Play arm maximizing the formula $\overline X_i + \sqrt{2 \log n \over n_i}$,
	  where $\overline X_i$ is average value of the arm $i$, $n$ is number
	  of games that were played by parent of the $i$ and $n_i$ is number of
	  games played with arm $i$.
\end{enumerate}


\subsection{UCT algorithm}
UCT is Upper Confidence bound to Trees or in other words UCB applied to minimax
trees. The main idea is to consider each node of minimax tree as multiarmed
bandit problem and each child of the node as independent arm.

In MoGo they used UCB1 algorithm extended to searching minimax tree called UCT.
MoGo \cite{MoGo}

UCT is algorithm traversing built multi-armed bandit tree using UCB1 formula.

Playing random simulation is also called playout.

We presume the reader is familiar with rules of the game Go. (TODO ???)
	
Standalone UCT needs set of carefully chosen extensions to be competitive. The
MCTS is some variant of UCT algorithm with those extensions.

UCB1 formula itself presumes the random involved variables are identically distributed and independent which is not true in UCT.
As MCTS we call UCT algorithm with various of extensions.

MCTS algorithm starts with a tree containing only the root node representing starting position.
MCTS consist of four steps, which we repeat until time is up:
\begin{enumerate}
\item Selection
\item Expansion
\item Simulation
\item Backpropagation
\end{enumerate}
\cite{progressive-strategies}

\lstset{language=Python, caption=Pseudocode of the MonteCarlo Tree Search, label=mcts:alg}
\begin{lstlisting}
playOneSequence(rootNode):
    node[0] = rootNode
    i = 0
    while node[i] is not visited for first time:
        node[i+1] = descendByUCB1(node[i])
        i = i + 1

    createNode(node[i])
    node[i].value = getValueByMC(node[i])

    for j in {0, ..., i-1}:
        updateValue(node[j],-node[i].value)
\end{lstlisting}

We use Monte Carlo Tree Search algorithm as is described in~\cite{MoGo} and also
modified by Kozelek~\cite{KOZELEK}.

We modified Kozelek's UCB1 exploration formula to form:
	$$
	\overline X_i + c \sqrt{\frac{log~n}{n_i}} + \frac{h_i}{n_i} + \frac{hh_i}{\sqrt{n_i}}
	$$
where $\overline X_i + c \sqrt{\frac{log~n}{n_i}}$ is a generalised UCB1
formula, $h_i$ is heuristics evaluated for step leading to this position, and
$hh_i$ is history heuristic value.

\chapter{Used optimisations in search engines}

In order to write a strong playing program in given game, it is necessary to
enhance chosen algorithm with various extensions. In this chapter we describe
the most used extensions for AlphaBeta or MCTS algorithm.

\section{AlphaBeta}
\subsection{Transposition Table}\label{AlphaBeta:TT}
If we look at the game tree of Arimaa, Chess or Go there is so many repetitions in
nodes of the tree for almost every position on the board. The Transposition table
is used to reduce the number of repetitions in tree.

COMPLETELY REWRITE: If in TT occurs result from previous search that is
shallower than we need, we prefer steps from Principal Variation of given entry
in succeeding search and use also its bounds.

If we find a satisfying result in TT we use its full PV and its score as ours.

\subsection{Iterative Deepening Framework}
Because there is no way how to determine how long the the AlphaBeta search to given
depth will take we need to find some time management tool. This is a method in
which we are iteratively starting new deeper and deeper searches. Thanks to
the exponential growth of the minimax tree with increasing depth we know that
a searching one level more will cost us approximately a lot more than previous
shallower searching.

Iterative deepening used with other optimisation methods such as History
Heuristic or Transposition Table often cause a time save if we compare it to
just searching to a maximal possible depth, because we can sort moves using
information gained from previous shallower searches~\cite{COX}. (TODO rewrite)

\subsection{Aspiration Windows}
% TODO how to mention it?
% \subsection{Quiescent search}
%   Is nice way, how to reduce horizont effect ...
%   % [Related to arimaa: http://arimaa.com/arimaa/forum/cgi/YaBB.cgi?board=devTalk;action=display;num=1122418533] <- TODO: Maybe implement this way
%   % [http://mediocrechess.sourceforge.net/guides/quiescentsearch.html]

\subsection{Move ordering}
The following methods change the order in which branches are selected and then
inspected. It is very important for the AlphaBeta search to have nodes well
ordered. Because earlier we find a pruning child of a node the shorter time
we spend in it.

	\subsubsection{History heuristics}
	The main idea behind this extension is that if some step is good enough
	to cause so many pruning anywhere in the search tree and if it is valid in
	given position it could be also good here.

	It is implemented in way that we create a table with a score for every
	element from combination of players, piece, position and direction. During
	the AlphaBeta search we increase the score for element every time such
	combination causes pruning or finds the best score. It is believed that
	the deeper cut-off happens the more relevant it is and therefore the score
	is incremented by $d^2$ or $2^d$ where $d$ is a actual searched depth.

	During searching are nodes of the minimax tree sorted by score from the
	History Heuristic table in decreasing order~\cite{COX}.

	\subsubsection{Killer moves}
	When a step prunes branches in some position it is very natural to ask if
	the same step could cause pruning in another branch and the same depth of
	the tree. To go even further we take two last steps caused pruning to be
	preferred in the search. As Zhong found out three or more Killer moves would help~\cite{ZHONG}.

	\subsubsection{Null move}


\subsection{Heuristics}
All preceding optimisations do not change result given by the algorithm. They
could only significantly decrease amount of time needed to obtain such result.

The consequent optimisations are rather heuristics, because they try to
give you approximately good result in much shorter time. In this work, we
will not use those optimisations.


- Negascout/PVS
- MTD-f
% TODO XXX: Are really heuristics?


\section{Monte Carlo Tree Search}
In AlphaBeta search we made great effort in sorting nodes of the searched tree
properly. In MCTS we need to use optimisations which helps us to gain better
and more informations from each iteration of algorithm on top of that.
(TODO too complicated, rewrite)

We chose to implement only heuristic Tomáš Kozelek described as
useful~\cite{KOZELEK}.

\subsection{Transposition table}
The motivation is the same as is in the AlphaBeta algorithm
(see~\ref{AlphaBeta:TT}). However in MCTS we are increasingly building game
tree instead of just exploring branches to some depth. A natural use of
Transposition table for MCTS is to share statistics for the same transpositions
in built game tree.

To do so we bind nodes considered the same to one. Bound nodes share their
children nodes, visit count and score statistic. We say that two nodes are the
same if they are in the same depth in minimax tree and if they represent the
same transposition. The game tree became Directed Acyclic Graph.

In implementation during the computation we keep table of all transpositions
and when any node is expanded we bind it to transposition in table if exists.

In Kozelek's work is regarded to be dangerous to bound visit count and score
statistic for children nodes~\cite{KOZELEK}. Nevertheless we believe that if
some step leads us to a position which is proven to be not worth trying in some
branch the same stands for all its transpositions.

\subsection{Progressive bias}
Progressive bias technique is a nice way how to combine offline learned
informations with online learned informations. The more we go through a node
the importance of offline learned information goes down and the importance of
online learned information became superior.

To do so ${H_B \over n_i}$ is added to the UCB formula. Where $H_B$ is
progressive bias coefficient computed by step-evaluation
function~\cite{progressive-strategies}.

In Arimaa such step-evaluation function should appreciate steps with Elephant
moving, killing an opponents piece, around previous steps, which are pushing or
pulling or making goal. Such function should also handicap players own piece
sacrifice steps or inverse steps~\cite{KOZELEK}.

\subsection{History heuristics}
- add to uct formula $+ {hh_i \over n_i}$. Where $hh_i$ is history
heuristics coeficiend as is described in Kozeleks thesis~\cite{KOZELEK}.

Similar approach is All-Moves-As-First Heuristics however it updates statistics only for all steps that were used in this simulation (in each node) (TODO)

\subsection{Best-of-$N$}
In random simulations, we may want to sacrifice true randomness for gaining
more objective results from playouts~\cite{HeavyPlayouts}. $N$ random steps are
generated and a step with the best value given by the same step-evaluation
function as was used in Progressive bias is chosen.

As a consequence, the number of playouts decreases, but the quality of
information learned in playouts improve and therefore the strength of the
program improve also.

\subsection{Children caching}
Kozelek introduced natural method how to decrease an amount of time spent in
node during selection part of the MCTS.

After some number of visits of a node its children caching is switched on. Then
a few best children are chosen and cached and every time the selection part of
the MCTS goes through this node it chooses a descend node from earlier cached
children. After some time it is necessary to discard and fill cache
again~\cite{KOZELEK}.

Using this optimisations should improve the speed of algorithm without negative
impact on quality.

\subsection{Maturity threshold}
TODO: Is another technique how to shorted time used in node selection. We
expand only nodes which had at least $threshold_maternity + depth\_of(node)$
visit count.

This creates 

\subsection{Virtual visits}
In node expansion initialise new node with $v$ virtual visits. This small
change increases the power of the algorithm significantly. Kozelek
experimentally determined the best performance of algorithm for $v \in
[4,5]$~\cite{KOZELEK}.

\section{Independent optimisations}
The following extensions are considered must have in every successful Arimaa
bot and do not depend on used algorithm.

	\subsection{Zobrist keys}
	Motivation ...

	For every element of $Piece\times Player\times Position$ the random 64 bit
	number is generated. ...

	How nodes are stored in Transposition table entries

	\subsection{Bitboards}
	Because Arimaa could be played with standart chess set it is also possible
	to use from world of chess well known board representation called
	Bitboards. One may say that bitboards suits even better for Arimaa than for
	Chess.

\section{Comparison of used optimisations}

\subsection{History heuristic}
In MCTS we prefer moves that were often chosen in going through tree. In
AlphaBeta value of move is increasing when branch using selected step is
pruned. Every time we order branches of the search tree by value given by
history heuristics. This approach should increase amount of pruning.

\subsection{Transposition table}


\chapter{Implementation}
We used Haskell as programming language. We developed set of libraries for play
Arimaa to be used with both MCTS and AlphaBeta algorithm. On top of those
libraries we built mentioned algorithms. The critical parts like bit operations
and evaluate function are written in C.

In Haskell is possible to program in much higher level than it is in most other
programming languages. However it is also harder to reason about performance.

We are pretty sure that more experienced Haskell programmer would write both
engines more efficiently.


We tried/made huge effort to keep our program rather simple and modular as much
as possible. Therefore one can switch on or off almost every mentioned search
extension.

In step generator we let the possibility to generate Pass step switched off by
default.

Lazily generated list of steps in AlphaBeta vs. lazily sorted children in MCTS.
(lazily generated list of steps was important)

Aspiration window gave sometimes strange results so we left it unused by default.
Using history heuristic in Alpha Beta program tends to decrease quality of our
program and we believe that causing list of steps from given position to be
evaluated is limiting in comparison to have them evaluated lazily. However it
is known that importance of history heuristic grows when the depth increases
[ZHONG] and hence for mor efficient programs HH is much more important.

Our AlphaBeta algorithm lack of standart Quiescence algorithm with trap control
or Goal check. We believe that both trap control and Goal check could be also
included in MCTS. However we know that true power of the algorithm is to
explore those situations. (TODO REWRITE and more optimistic)

\section{Parallelization}
In Monte Carlo Tree Search, do parallelization is much more natural than is in
AlphaBeta search.

Haskell gives us easy threading and data structure locking capabilities.
Local mutexes are rather simple thanks to MVar structure.

\section{Evaluation function}
We wrote evaluation function completely in C to be as fast as possible.

In our work we do not focus on creating strong evaluation function even though
in order to have arimaa playing program, we need to have one. Therefore we
developed one not so strong and with possibility to easily replace it.

Our evaluation function lack of a very important part of standart evaluation
function -- the goal check.
Also it do not have frame and hostage detection.

\chapter{Methodology}

At the begging we decided to follow the consequent schedule:

\begin{enumerate}% {0}
\item Study possible and most used method used in Arimaa, Chess and Go.
\item Develop two arimaa playing programs using two earlier described
	  algorithms and all listed optimisatoin methods for them.
\item Compare those engines playing offline matches.
\item Analise engines time cost centres. Where are bottlenecks.
\end{enumerate}

\noindent By comparing engines we mean:

\begin{enumerate}
\item playing 3s/10s/30s per move time limit,
\item playing with transposition tables of size 100MB, 200MB, or 400MB,
\item using one, two, four, eight cores,
% \item run AB to same level of search and give MCTS the same amount of time to play
\end{enumerate}

Work on bot is never ending so we will try to develop two comparable bots and
try to measure how changes in settings affect their win:loss rate.

\chapter{Conclusion}
Writing two programs at once turns out to be more difficult than we thought. In
our development we have not focussed on high time optimisations and
sophisticated goal check and trap control in position evaluation. These three
thinks are considered the most difficult in Arimaa bot development.

Using MCTS as is or as was provided by Kozelek is not enought for bot
programming in Arimaa, but UCB formula or UCT algorithm itself can be valuable
in yet unknown hybrid algorithm using both AlphaBeta and MCTS algorithms.

??? Combine previous and consequent paragraphs ???

The most prommising variant of the Arimaa playing program using MCTS algorithm
could be some combination of MCTS and AlphaBeta. Using full four depth
AlphaBeta search while expanding leafs of UCT. (after node expansion make
tactiacal lookahead)

REWRITE: We also believe that using more sophisticated board representation
would help MCTS algorithm against AlphaBeta, because the cost of managing
clever representation returns with more times you use the clever
representation, which dominates in MCTS. (From one position MCTS often
generates possible moves more than once in contrast of AlphaBeta).

What was done. What we achieved. What we get from it. ...

\section{Further work}
% TODO it should be further work from our point of view (to attack another similar topics to ours)

\begin{enumerate}
\item Find way how to implement some kind of Quiescence search to MCTS and compare with similar approach in AlphaBeta.
\item ...
\item Optimise playouts using more sophisticated way how to generate steps. Interesting ideas are described in Zhong's work~\cite{ZHONG}.
\item try patterns heuristics in playout generation~\cite{PatternsGo,PatternsArimaa}.
\end{enumerate}
Goal check??

The quality of bots can be improved by:
\begin{enumerate}
\item create better evaluation function,
\item try progressive pruning  methods~\cite{progressive-strategies,
MonteCarloGo},
\item use more optimised data structures and random number generator,
\item We learned from \ref{pic:generalTuning} that 
\end{enumerate}


\appendix
\def\bibname{Literature}
\begin{thebibliography}{99}
\addcontentsline{toc}{chapter}{\bibname}

\bibitem{KOZELEK}
	{\sc KOZELEK,} Tomáš.
	\emph{MCTS methods in the game of Arimaa}.
	Master's thesis.
	Charles University in Prague, 2009.

\bibitem{COX}
	{\sc COX,} Christ-Jan.
	\emph{Analysis and Implementation of the Game Arimaa}.
	Master Thesis.
	Universiteit Maastricht, 2006.

\bibitem{arimaa.com}
	Arimaa homepage.
	\texttt{http://arimaa.com/}.

\bibitem{syed}
	Omar {\sc SYED} and Aamir {\sc SYED}.
	\emph{Arimaa: A New Game Designed to be Difficult for Computers}.
	Journal of the International Computer Games Association, 26(2):138–139, 2003.

\bibitem{knuth:alphabeta}
	Donald E. {\sc KNUTH} and R. W. {\sc MOORE} (1975).
	\emph{An Analysis of Alpha-Beta Pruning}.
	Artificial Intelligence Vol. 6, No. 4: 293-326. Reprinted in Selected Papers on Analysis of Algorithms by Donald E. Knuth, Published 2000, by Addison-Wesley
	ISBN: 1-57585-211-5.

\bibitem{ZHONG}
	{\sc Zhong,} Haizhi.
	\emph{Building a Strong Arimaa-playing Program}.
	Master Thesis.
	University of Alberta, 2005.

\bibitem{progressive-strategies}
	Guillaume {\sc CHASLOT}, Mark {\sc WINANDS}, Jaap H. van den {\sc HERIK},
	Jos {\sc UITERWIJK}, and Bruno {\sc BOUZY}.
	\emph{Progressive strategies for Monte-Carlo tree search}.
	In Joint Conference on Information Sciences, Salt Lake City 2007, Heuristic
	Search and Computer Game Playing Session, 2007.


\end{thebibliography}

[2] Peter Auer, Nicol` Cesa-Bianchi, and Paul Fischer. Finite-time analysis of
    the o multiarmed bandit problem. Mach. Learn., 47(2-3):235–256, 2002.
[BERND] Bernd Brugmann. Monte Carlo go. Technical report, 1993.
[15] Chaslot, Winonds, ..., Progressive strategies for Monte-Carlo Tree Search, 2007.


\chapwithtoc{List of Abbreviations/Glossary}

\chapter{Appendix 1}
- tables that weren't placed into text.

\chapter{Appendix 2 - User documentation}
We provide source codes of bots and all small tools we developed during our
work. All is available in \url{http://github.com/JackeLee/rabbocop} or on
included CD.

\section{Compiling options}
Source codes of programs are saved on CD in \texttt{rabbocop} dictionary.
We use the make for building our programs. Simple writing \texttt{make
IterativeAB} or \texttt{make MCTS} compiles either AlphaBeta or MCTS in its
standart version of the bot.


\begin{description}
\item[VERBOSE=$n$]{
	For IterativeAB program, this will print actual best move and score anytime
	a depth is fully explored.
	For MCTS it will print actual best move and score after each $n$ iterations of
	UCT algorithm.
	}

\item[EVAL=fairy]{
	Builds program using Fairy's evaluation function.}

\item[NULL\_MOVE]{
	Enables Null moves for AlphaBeta search.}

\item[canPass=1]{
	Adds possibility to generate less than 4 moves.}

\item[CORES=$X$]{
	Compiles program with the number of available CPU's set to $X$. (Default is 1)
	When running program compiled with CORES=X it is necessary to add +RTS -N$X$
	as additional arguments.}

\item[WINDOW=$X$]{
	It switches on aspiration window option for AlphaBeta algorithm with Window
	set to $X$.}

\item[noHH=1]{
	Disables history heuristics optimisations for MCTS.}

\item[abHH=1]{
	Enables history heuristics optimisations for AlphaBeta.}

\item[noHeavyPlayout=1]{
	This disables heuristic in playouts.}

\item[PROF=1]{
	Enables profiling for bots.}

\item[playAB, playMCTS, or playMatch]{
	Starts new game to play by human vs AlphaBeta search, human versus MCTS or
	AlphaBeta search versus MCT.
	For the first time it downloads testing and graphical environments for
	offline matches.
	}
\end{description}


\openright
\end{document}
