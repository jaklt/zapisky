\documentclass{js}

\newcommand\m[1]{\mathbb { #1 }}
\newcommand\p[1]{\mathcal{ #1 }}
\newcommand\N{\ensuremath{\m N}}
\let \forces \Vdash

\DeclareMathOperator{\gen}{gen}

\newenvironment{items}{
	\itemize
	\itemsep = 0pt%
}{%
	\enditemize
}
\newenvironment{itemsn}{
	\enumerate%
	\itemsep = 0pt%
}{%
	\endenumerate%
}

\begin{document}
\title{The Set of Arithmetical Sets is not Arithmetical}
\author{Tom\'a\v s Jakl (KAM MFF)}
\maketitle

Cohen introduced forcing in 1963 to prove independence of the Axiom of choice
and the Continuum hypothesis from Zermelo--Fraenkel set theory. From a
topological point of view the construction of generic filter in the method of
forcing is nothing more than Baire category theorem.

In this talk I will show how arithmetical forcing (one of the most transparent
kinds of forcing) can be used for recursion theoretic proof of theorem that the
set of all arithmetical sets is not arithmetical.

\section{Preliminaries}

\begin{definition}
    Let $\mathfrak L$ be the first-order language with a constant $\bar n$ for each $n \in \N$, an unary relation `$\cdot \in X$' and function and relation  symbols $+$, $\times$,  $<$, $=$.

    \vspace{0.2em}
    Given a formula $\varphi$ of $\mathfrak L$, $\varphi$ \emph{is true for} $A \subseteq \N$ ($A \models \varphi$) is inductively defined as follows:

    \begin{tabular}{l c l}
        $A \models \varphi \text{ is atomic without `} \cdot \in X$' & $\iff$ & $\varphi$ is true in \N \\
        $A \models \bar n \in X$ & $\iff$ & $n \in A$ \\
        $A \models \neg \psi$ & $\iff$ & not $A \models \psi$ \\
        $A \models \psi_0 \lor \psi_1$ & $\iff$ &  $A \models \psi_0$ or $A \models \psi_1$ \\
        $A \models (\exists x)\psi(x)$ & $\iff$ &for some $n\in\N$ $A \models \psi(\bar{n})$. \\
    \end{tabular}
\end{definition}

Cohen's idea was based on intuition that the truth can be approximated by a
finite information. For example if some finite string $\sigma$ of 0s and 1s
tells us that $\varphi$ is true, then $A \models \varphi$ for all $A \supseteq
\sigma$ (we identify a set and its characteristic function).

\begin{definition}
    Given a formula $\varphi$ of $\mathfrak L$, $\sigma$ \emph{forces} $\varphi$ ($\sigma \forces \varphi$) is inductively defined as follows:

    \begin{tabular}{l c l}
        $\sigma \forces \varphi \text{ is atomic without `} \cdot \in X$' & $\iff$ & $\varphi$ is true in \N \\
        $\sigma \forces \bar n \in X$ & $\iff$ & $\sigma(n) = 1$ \\
        $\sigma \forces \neg \psi$ & $\iff$ & $(\forall \tau \supseteq \sigma)(\tau \not\forces \psi$) \\
        $\sigma \forces \psi_0 \lor \psi_1$ & $\iff$ &  $\sigma \forces \psi_0$ or $\sigma \forces \psi_1$ \\
        $\sigma \forces (\exists x)\psi(x)$ & $\iff$ &for some $n\in\N$ $\sigma \forces \psi(\bar{n})$. \\
    \end{tabular}
\end{definition}

\begin{definition}
    For given $A \subseteq \N$ we say $A$ \emph{forces} $\varphi$ ($A \forces \varphi$) if for some finite $\sigma \subseteq A$, $\sigma \forces \varphi$.
\end{definition}

% lemma: Basic properties of \forces

\begin{definition}
    $A$ is \emph{$n$-generic} if for all sentences $\varphi \in \Sigma_n$ either $A \forces \varphi$ or $A \forces \neg\varphi$. $A$ is \emph{$\omega$-generic} if the same happens for all sentences of $\mathfrak L$.
\end{definition}

\begin{lemma}[Forcing = Truth for generic sets]
    $A$ is $n$-generic iff for any sentence $\varphi \in \Sigma_n \cup \Pi_n$, $A \models \varphi$ iff $A \forces \varphi$.
\end{lemma}

% lemma: Definability of forcing

\begin{definition}
    A set $A \subseteq \N$ \emph{is defined by} a formula $\varphi$ if $n \in A \iff \emptyset \models \varphi(\bar{n})$. A set $A$ is said to be \emph{arithmetical} if it is defined by some formula of $\mathfrak L$.
    % ... definition for classes
\end{definition}

\section{Demonstration of the method}

\begin{theorem}
    There is a $n$-generic set defined by $\Sigma_{n+1}$ formula.
\end{theorem}

\begin{proof}[Proof idea]
    \begin{enumerate}
        \item We have $\Sigma_0$ enumeration of $\mathfrak L$ sentences: $\psi_0, \psi_1, \dots$.
        \item For each $\psi_i$ we $\Sigma_{n+1}$-find a string $\sigma_i$ extending $\sigma_{i-1}$ such that $\sigma_i \forces \psi_i$ or $\sigma_i \forces \neg\psi_i$.
        \item $\bigcup_i \sigma_i$ is a characteristic function of a $n$-generic.
    \end{enumerate}
\end{proof}

The proof goes the same way as the proof of Baire category theorem (theorem says:
in a complete metric/locally compact Hausdorff space a countable intersection
of open dense sets is dense).

The sets $[\sigma] = \{ A \subseteq \N: A \supseteq \sigma \}$ form a basis of
product topology of $2^\omega$. Since $2^\omega$ is locally compact Hausdorff
space and the fact that the set $\p O_\varphi = \{A : A \forces \varphi \text {
or } A \forces \neg\varphi\}$ is open and dense in $2^\omega$ there is a
$n$-generic set in $\bigcap_{\varphi \in \Sigma_n} \p O_\varphi$.

% theorem: There are comeager many $\omega$-generic sets in $2^\omega$.

\begin{theorem}
    The set of arithmetical sets is not arithmetical.
\end{theorem}

\section{Conclusion}
In Recursion theory, in order to force desired properties it often suffices to
find some generic set, on the other hand in Set theory one has to construct a
model ``along'' constructed generic filter which will then have desired
properties. Model construction then makes the method much more difficult.

\end{document}
